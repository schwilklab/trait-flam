%% reftex-default-bibliography: ("/home/schwilk/write/bib/schwilk.bib")

% LaTeX support: latex@mdpi.com 
%  In case you need support, please attach all files that are necessary for compiling as well as the log file, and specify the details of your LaTeX setup (which operating system and LaTeX version / tools you are using).

%  =================================================================
\documentclass[fire,article,submit,moreauthors,pdftex]{Definitions/mdpi} 


%\usepackage[utf8]{inputenc}

% If you would like to post an early version of this manuscript as a preprint, you may use preprint as the journal and change 'submit' to 'accept'. The document class line would be, e.g., \documentclass[preprints,article,accept,moreauthors,pdftex]{mdpi}. This is especially recommended for submission to arXiv, where line numbers should be removed before posting. For preprints.org, the editorial staff will make this change immediately prior to posting.

%--------------------
% Class Options:
%--------------------
%----------
% journal
%----------
% Choose between the following MDPI journals:
% acoustics, actuators, addictions, admsci, aerospace, agriculture, agriengineering, agronomy, algorithms, animals, antibiotics, antibodies, antioxidants, applsci, arts, asc, asi, atmosphere, atoms, axioms, batteries, bdcc, behavsci , beverages, bioengineering, biology, biomedicines, biomimetics, biomolecules, biosensors, brainsci , buildings, cancers, carbon , catalysts, cells, ceramics, challenges, chemengineering, chemistry, chemosensors, children, cleantechnol, climate, clockssleep, cmd, coatings, colloids, computation, computers, condensedmatter, cosmetics, cryptography, crystals, dairy, data, dentistry, designs , diagnostics, diseases, diversity, drones, econometrics, economies, education, electrochem, electronics, energies, entropy, environments, epigenomes, est, fermentation, fibers, fire, fishes, fluids, foods, forecasting, forests, fractalfract, futureinternet, futurephys, galaxies, games, gastrointestdisord, gels, genealogy, genes, geohazards, geosciences, geriatrics, hazardousmatters, healthcare, heritage, highthroughput, horticulturae, humanities, hydrology, ijerph, ijfs, ijgi, ijms, ijns, ijtpp, informatics, information, infrastructures, inorganics, insects, instruments, inventions, iot, j, jcdd, jcm, jcp, jcs, jdb, jfb, jfmk, jimaging, jintelligence, jlpea, jmmp, jmse, jnt, jof, joitmc, jpm, jrfm, jsan, land, languages, laws, life, literature, logistics, lubricants, machines, magnetochemistry, make, marinedrugs, materials, mathematics, mca, medicina, medicines, medsci, membranes, metabolites, metals, microarrays, micromachines, microorganisms, minerals, modelling, molbank, molecules, mps, mti, nanomaterials, ncrna, neuroglia, nitrogen, notspecified, nutrients, ohbm, particles, pathogens, pharmaceuticals, pharmaceutics, pharmacy, philosophies, photonics, physics, plants, plasma, polymers, polysaccharides, preprints , proceedings, processes, proteomes, psych, publications, quantumrep, quaternary, qubs, reactions, recycling, religions, remotesensing, reports, resources, risks, robotics, safety, sci, scipharm, sensors, separations, sexes, signals, sinusitis, smartcities, sna, societies, socsci, soilsystems, sports, standards, stats, surfaces, surgeries, sustainability, symmetry, systems, technologies, test, toxics, toxins, tropicalmed, universe, urbansci, vaccines, vehicles, vetsci, vibration, viruses, vision, water, wem, wevj

%---------
% article
%---------
% The default type of manuscript is "article", but can be replaced by: 
% abstract, addendum, article, benchmark, book, bookreview, briefreport, casereport, changes, comment, commentary, communication, conceptpaper, conferenceproceedings, correction, conferencereport, expressionofconcern, extendedabstract, meetingreport, creative, datadescriptor, discussion, editorial, essay, erratum, hypothesis, interestingimages, letter, meetingreport, newbookreceived, obituary, opinion, projectreport, reply, retraction, review, perspective, protocol, shortnote, supfile, technicalnote, viewpoint
% supfile = supplementary materials

%----------
% submit
%----------
% The class option "submit" will be changed to "accept" by the Editorial Office when the paper is accepted. This will only make changes to the frontpage (e.g., the logo of the journal will get visible), the headings, and the copyright information. Also, line numbering will be removed. Journal info and pagination for accepted papers will also be assigned by the Editorial Office.

%------------------
% moreauthors
%------------------
% If there is only one author the class option oneauthor should be used. Otherwise use the class option moreauthors.

%---------
% pdftex
%---------
% The option pdftex is for use with pdfLaTeX. If eps figures are used, remove the option pdftex and use LaTeX and dvi2pdf.

%=================================================================
\firstpage{1} 
\makeatletter 
\setcounter{page}{\@firstpage} 
\makeatother
\pubvolume{xx}
\issuenum{1}
\articlenumber{5}
\pubyear{2019}
\copyrightyear{2019}
%\externaleditor{Academic Editor: name}
\history{Received: date; Accepted: date; Published: date}
%\updates{yes} % If there is an update available, un-comment this line

%% MDPI internal command: uncomment if new journal that already uses continuous page numbers 
%\continuouspages{yes}

%------------------------------------------------------------------
% The following line should be uncommented if the LaTeX file is uploaded to arXiv.org
%\pdfoutput=1

%=================================================================
% Add packages and commands here. The following packages are loaded in our class file: fontenc, calc, indentfirst, fancyhdr, graphicx, lastpage, ifthen, lineno, float, amsmath, setspace, enumitem, mathpazo, booktabs, titlesec, etoolbox, amsthm, hyphenat, natbib, hyperref, footmisc, geometry, caption, url, mdframed, tabto, soul, multirow, microtype, tikz


%\bibliography{/home/schwilk/write/bib/schwilk.bib}

%=================================================================
%% Please use the following mathematics environments: Theorem, Lemma, Corollary, Proposition, Characterization, Property, Problem, Example, ExamplesandDefinitions, Hypothesis, Remark, Definition
%% For proofs, please use the proof environment (the amsthm package is loaded by the MDPI class).

%=================================================================
% Full title of the paper (Capitalized)
\Title{Moisture effects on leaf litter are species-specific and result in non-additive flammability in mixed conifer forest}

% Author Orchid ID: enter ID or remove command
\newcommand{\orcidauthorA}{0000-0000-000-000X} % Add \orcidA{} behind the author's name
%\newcommand{\orcidauthorB}{0000-0000-000-000X} % Add \orcidB{} behind the author's name

% Authors, for the paper (add full first names)
\Author{Rita Quinones de Magalhaes $^{\dagger}$ and Dylan W. Schwilk}

% Authors, for metadata in PDF
\AuthorNames{Rita Magalh\~{a}es, Dylan W. Schwilk}

% Affiliations / Addresses (Add [1] after \address if there is only one affiliation.)
\address[1]{Texas Tech University}

% Contact information of the corresponding author
\corres{Correspondence: dylan.schwilk@ttu.edu, DWS}

% Current address and/or shared authorship
\firstnote{Current address: Rochester Institute of Technology, Rochester, NY 14623} 
%\secondnote{These authors contributed equally to this work.}
% The commands \thirdnote{} till \eighthnote{} are available for further notes

%\simplesumm{} % Simple summary

%\conference{} % An extended version of a conference paper

% Abstract (Do not insert blank lines, i.e. \\) 
\abstract{ Moisture content is a strong determinant of forest fuel
  flammability. But less is known about the mechanisms by which litter traits
  influence litter moisture dynamics, particularly the processes of moisture
  absorption and desorption. Characterizing how species’ leaf litter retain and
  release moisture through time is critical to the assessment of fuel’s
  availability to burn. Natural forests are often comprised of mixed stands,
  leading to mixtures of leaf litter on the surface fuelling the fires. Multiple
  studies have established the existence of non-additive effects in the
  flammability of dry litter, but we lack a mechanistic understanding of
  moisture synergies in mixed litter and of moisture effects on the
  flammability of mixed litter. The litter from eight species of a
  mixed-conifer forest was saturated and allowed to dry to determine moisture
  absorption and desorption. Burn trials were performed in moist litter beds of
  single species and of mixtures, to establish flammability response to the
  dry-down process. Moisture absorption is determined by traits pertaining to
  leaf geometry while moisture desorption is influenced by emergent properties
  of the litterbed. Species that produce more aerated litter beds (such as \emph{Q.
  kelloggii} and \emph{Pinus} species) have faster rates of litter desiccation. My results
  show a relationship between time since wetting and time to ignition and
  spread rate, with significant interaction with genus. This indicates a
  genus-specific behavior of flammability and time since wetting.
  We found non-additive effects in litter mixtures for both moisture content and
  flammability. Mixtures containing \emph{Q. kelloggii} and \emph{P. jeffreyi}, species that
  dry faster, produced more flammable behavior (lower time to ignition, faster
  spread rate, taller flames). Fire behavior of mixtures depended on moisture
  content, although mixtures without Q. kelloggii had always higher moisture
  content than predicted by the average fire behavior of the species in the
  mixture. At lower moisture contents (<75\%) the mixtures exhibit positive
  non-additive behavior and at higher moisture contents (> 200\%) they exhibit
  negative non-additive behavior.}

% A single paragraph of about 200 words maximum. For research articles,
% abstracts should give a pertinent overview of the work. We strongly encourage
% authors to use the following style of structured abstracts, but without
% headings: (1) Background: Place the question addressed in a broad context and
% highlight the purpose of the study; (2) Methods: Describe briefly the main
% methods or treatments applied; (3) Results: Summarize the article's main
% findings; and (4) Conclusion: Indicate the main conclusions or
% interpretations. The abstract should be an objective representation of the
% article, it must not contain results which are not presented and
% substantiated in the main text and should not exaggerate the main
% conclusions.


% Keywords
\keyword{flammability, litter, flame spread rate}

% The fields PACS, MSC, and JEL may be left empty or commented out if not applicable
%\PACS{J0101}
%\MSC{}
%\JEL{}

%%%%%%%%%%%%%%%%%%%%%%%%%%%%%%%%%%%%%%%%%%
% Only for the journal Diversity
%\LSID{\url{http://}}

%%%%%%%%%%%%%%%%%%%%%%%%%%%%%%%%%%%%%%%%%%
% Only for the journal Applied Sciences:
%\featuredapplication{Authors are encouraged to provide a concise description of the specific application or a potential application of the work. This section is not mandatory.}
%%%%%%%%%%%%%%%%%%%%%%%%%%%%%%%%%%%%%%%%%%

%%%%%%%%%%%%%%%%%%%%%%%%%%%%%%%%%%%%%%%%%%
% Only for the journal Data:
%\dataset{DOI number or link to the deposited data set in cases where the data set is published or set to be published separately. If the data set is submitted and will be published as a supplement to this paper in the journal Data, this field will be filled by the editors of the journal. In this case, please make sure to submit the data set as a supplement when entering your manuscript into our manuscript editorial system.}

%\datasetlicense{license under which the data set is made available (CC0, CC-BY, CC-BY-SA, CC-BY-NC, etc.)}

%%%%%%%%%%%%%%%%%%%%%%%%%%%%%%%%%%%%%%%%%%
% Only for the journal Toxins
%\keycontribution{The breakthroughs or highlights of the manuscript. Authors can write one or two sentences to describe the most important part of the paper.}

%\setcounter{secnumdepth}{4}
%%%%%%%%%%%%%%%%%%%%%%%%%%%%%%%%%%%%%%%%%%
\begin{document}
%%%%%%%%%%%%%%%%%%%%%%%%%%%%%%%%%%%%%%%%%%

%%%%%%%%%%%%%%%%%%%%%%%%%%%%%%%%%%%%%%%%%%
% \setcounter{section}{-1} %% Remove this when starting to work on the template.
% \section{How to Use this Template}
% The template details the sections that can be used in a manuscript. Note that the order and names of article sections may differ from the requirements of the journal (e.g., the positioning of the Materials and Methods section). Please check the instructions for authors page of the journal to verify the correct order and names. For any questions, please contact the editorial office of the journal or support@mdpi.com. For LaTeX related questions please contact latex@mdpi.com.
%The order of the section titles is: Introduction, Materials and Methods, Results, Discussion, Conclusions for these journals: aerospace,algorithms,antibodies,antioxidants,atmosphere,axioms,biomedicines,carbon,crystals,designs,diagnostics,environments,fermentation,fluids,forests,fractalfract,informatics,information,inventions,jfmk,jrfm,lubricants,neonatalscreening,neuroglia,particles,pharmaceutics,polymers,processes,technologies,viruses,vision

\section{Introduction}
% The introduction should briefly place the study in a broad context and highlight why it is important. It should define the purpose of the work and its significance. The current state of the research field should be reviewed carefully and key publications cited. Please highlight controversial and diverging hypotheses when necessary. Finally, briefly mention the main aim of the work and highlight the principal conclusions. As far as possible, please keep the introduction comprehensible to scientists outside your particular field of research. Citing a journal paper \cite{ref-journal}. And now citing a book reference \cite{ref-book}. Please use the command \citep{ref-journal} for the following MDPI journals, which use author-date citation: Administrative Sciences, Arts, Econometrics, Economies, Genealogy, Humanities, IJFS, JRFM, Languages, Laws, Religions, Risks, Social Sciences.

Fuel moisture content is an important, and often driving, determinant of surface fire behavior. Environmental conditions such as temperature, relative humidity, and wind, drive the moisture content of the fuel available to burn [CITE]. In particular, fine dead fuels respond very readily to environmental changes [CITE]. In mixed-conifer forest of the western USA, where the fire regime is characterized by surface fires fuelled by plant litter, leaf traits can influence landscape scale fire behavior \cite{Schwilk+Caprio-2011}. Litter packing (measured as litterbed density or packing ratio) controls the flammability of dry litter fuels in seasonally dry climates (higher litter density reduces flammability) and leaf litter packing is influenced by leaf size and shape \cite{Fonda+Belanger+etal-1998, Kane+Varner+etal-2008, Schwilk+Caprio-2011, Kreye+Varner+etal-2013}. Fine fuel moisture content reduces ignition probability and flame spread rate \cite{Gisborne-1936, Fons-1946, Anderson+Rothermal-1965}. However, the mechanism by which both monospecific and mixed litterbeds absorb and desorb moisture is poorly known, and could be characterized by leaf traits, litter traits, or both.

Fuel moisture influences flammability and fire behavior through effects on ignition, spread rate, flame height, fuel consumption, and heat release, due to the high specific heat content of water \cite{Rothermel-1972}. Specifically, high moisture contents decrease the likelihood of ignition thereby reducing fire spread \cite{Nelson-2011}. Moisture acts as a heat-sink resulting in less energy available for propagation, and the resulting water vapor will both cool the flaming front and dilute flammable gases \cite{Albini-1976, Shafizadeh-1977}. Variation in moisture of living fuels has strong effects on flammability, but moisture of dead excised litter fuels can also vary. Moisture of litter fuels varies with climate, with seasons and with short term weather within a season. Therefore, the rate of drying and flammability at any particular moisture content are key parameters that influence the fuel flammability throughout the fire season \cite{Viney-1991}. If species differ in the rate of drying following a precipitation event, this may strongly influence relatively flammability across species at any particular time post-wetting.  Such differential moisture content across time, may contribute to the often high spatial heterogeneity of surface fires. Fuel moisture dynamics of leaf litter are dependent upon forest structure (shading and rain throughfall effects), litter physical structure, and chemical composition of the fuelbeds \cite{Nelson+Hiers-2008, Matthews-2014, Kreye_Hiers_etal-2018}.

Fuel moisture content through time is controlled by two main processes, absorptive capacity and rate of moisture loss \citep{Kreye-2013}. Absorptive capacity represents the potential maximum fuel moisture and is evaluated by the moisture content of litter when saturated. After saturation, the differences in moisture content indicate different abilities to retain moisture. The ability of litter to absorb and desorb water should be dependent on leaf and/or litter geometry, however, which traits control these processes are largely unknown. We propose two different sets of traits responsible for variation in the two processes mentioned: leaf particle geometry and litter bulk density. Litter characteristics such as surface-area-to-volume ratio, litter bulk density, mass density, particle size and shape, and chemical traits influence moisture diffusion through litter. Litter packing has two pathways of effect on fire behavior: directly by its effect on oxygen availability for combustion, and indirectly by its effect on dry-down rate, and both these pathways are affected by leaf litter traits (such as leaf size, litter bulk density). Fuelbeds composed of larger leaves with lower moisture retainability and decomposition rates favor drying, thus increasing the probability of ignition and fire spread rate.

Species traits may interact such that the flammability of multi-species mixtures is not the average of the constituent species.  Such non-additive effects have been shown to be synergistic in dry litter fuels where the flammability of a mixture is driven by the most flammable species  \cite{VanAltena+Logtestjin+etal-2012, Magalhaes+Schwilk-2012}. In most forests, leaf litter will be a mixture representative of the species in the stand. It is possible that similar non-additive effects may influence litter moisture content, especially if different leaf traits influence adsorptive capacity and drying rate. Blauw et al.\cite{Blauw+Wensink+etal-2015} suggests that moisture content can contribute to non-additive effects on heathland flammability, but the mechanism of this effect is not known.

% I investigated non-additive effects on both moisture content and the flammability of litter mixtures varying in time since wetting and in moisture content. I expect that litter mixtures will differ in their bulk density and thus, will dry at different rates.

We investigated species trait effects on litter moisture dynamics and consequent litter flammability across eight tree species in a temperate fire-prone forest in California. We determined the rate of moisture loss, the moisture of extinction (the threshold representing the moisture value above which fire will not ignite or spread) and the influence of moisture on the flammability of eight species from a mixed-conifer forest in California. Experimental burn trials of leaf litter at various moisture levels determined how time since wetting influenced time to ignition and flame spread rate. The burn trials were also performed with multi-species litter mixtures to evaluate if species contribute equally to litter moisture dynamics or if there are non-additive effects. We compared leaf litter traits with the measured moisture absorption and desorption rates. Finally, we tested if the non-additive effects on flammability reported in previous studies occur under ecologically-relevant moisture levels and whether fuel moisture effects strengthen or mask such non-additivity.


%%%%%%%%%%%%%%%%%%%%%%%%%%%%%%%%%%%%%%%%%%
\section{Materials and Methods}

\subsection{Site description and species selection}

Field sites for this study were located in Sequoia and Kings Canyon National Parks, California, USA (36 36N, 118 42 W) between 1600 and 2400 m elevation in mixed-conifer forest. I chose eight tree species that are representative of this type of forest: \emph{Pinus jeffreyi} Grev. \& Balf., \emph{Pinus lambertiana} Dougl., \emph{Pinus ponderosa} Dougl. ex Laws., \emph{Abies concolor} (Gord. \& Glend.) Lindl. ex Hildebr., \emph{Abies magnifica} A. Murr., \emph{Calocedrus decurrens} (Torr.) Florin, \emph{Quercus kelloggii} Newb., and \emph{Sequoiadendron giganteum} (Lindl.) J. Buchholz. In this study, I consider litter the top layer of leaves and small twigs less than 0.625 cm diameter (= 1-hour fuel) that have fallen in the previous year (mostly undecomposed). I collected the leaf litter in the summer (mid-June to mid-July) in 2011 and in 2012 from 21 separate sites across the study area. Effort was made to collect from different populations (minimum of four, each population separated from the others by 9 to 32 km) for each species to control for plastic and ecotypic effects. An exception to this involved the litter from A. magnifica which is restricted to the higher elevations and could only be collected from one area, due to logistical constraints. Within that area, \emph{A. magnifica} litter was collected from over 10 different individuals. For the remaining species, the collection involved 2-4 individual trees at each site, separated by at least 10 m. The litter was collect about 2 m away from the trunk to obtain a more uniform sample, because bark and heavier branches tend to fall closer to the tree trunk.

\subsection{Moisture absorption and desorption}


We produced dry-down curves to determine maximum water retainability and the rate of moisture content loss (dessication rate) for each species and for four distinct mixtures. These curves allowed us to  determine the rate at which a species loses moisture (approximated as
exponential decay), and determine maximum water content following saturation
and gravity draining. Samples from the eight species (six replicates per
species) and four mixtures (5 replicates per mixture type) were placed in
baskets (approximately 45 x 45 cm by 15 cm tall) constructed of 1 cm aluminum
screen. Samples were 450 g of litter (dry weight) per trial, and litter depth
was standardized to 10 cm by changing the horizontal dimensions of the basket.
Weight was measured by using a balance sensitive to 0.1 g (model XS16001L,
Mettler Toledo, Columbus, OH). Samples were immersed in plastic storage bins
filled with water for 24 hours to fully saturate them. After saturation, the
baskets were removed from the water, allowed to drain for 3 minutes, and
samples weighed to measure maximum moisture content. The samples were then
allowed to dry in a partially controlled environment kept at 21 C temperature
and 30\% relative humidity, assessed via an iButton sensor (Maxim Integrated)
placed over the samples. During the dry-down process three subsamples were
taken at 24 hour intervals and weighed. These subsamples were then oven dried
for 24 hours at 100 C and re-weighed to assess moisture content. This was
repeated until the final sample from a basket was less than10\% moisture.


The dry-down curves indicated that the eight species varied in their moisture behavior [FIG 2?]. To determine a subset of species to use in the mixture trials, we performed analysis of contrasts using lsmeans package in R  (TODO Lenth 2016) to determine similarity between the species, which fell into four groups. I then selected one species from each group to include in the mixtures resulting in four distinct 3-species mixtures. 

 Leaf trait data
 
We investigated the influence of leaf thinness and litter bulk density on the moisture dynamics of the species in this study. Leaf thinness data (leaf length over thickness) was obtained for subsamples of the litter collected for this project.  Bulk density of the litter was obtained for each basket during the dry-down experiment by dividing the dry mass (450 g) by the estimated litter volume (basket width x length x litter depth).

\subsection{Flammability assessment}

We performed flammability tests across multiple moisture levels on monospecific litter beds and on mixed litter beds with eaach species or mixture replicated 5 times at each moisture level. Flammability was assessed using a 150 cm long burn table, 15 cm wide and 15 cm tall, in which leaf litter was placed and the table gently shaken to allow settling. This design allows fire behaviour to reach constant flame spread rates \cite{Magalhaes+Schwilk-2012}. Two graduated metal rulers   equally spaced along the apparatus allowed visual determination of maximum flame height.
Litter was ignited with a propane torch; maximum adiabatic flame temperature of 1899 C, http://www.benzomatic.com/products/fuel.aspx) at one end of the apparatus and allowed to burn until extinction. If a species at a specific moisture level failed to ignite after three attempts, no further attempts were made.  A timer was used to record time to ignition and duration of flaming combustion. The flammability trials followed the methodology detailed in de Magalhães and Schwilk \cite{Magalhaes+Schwilk-2012}. 

% and nine type-K thermocouples attached to data loggers measured temperature every second at three distances from the ignition point and at three different depths below the litter surface.
 
During flammability trials, we recorded time to ignition (s), duration of flaming combustion (s), spread rate (mm/s), calculated by dividing the length of the burned surface by the time it takes the flaming front to reach the end of the apparatus, maximal flame height (mm), and mass loss (\%), calculated as the difference between fuel mass before and after flame extinction. I also measured duration above 100 C, or the length of time temperature exceeded 100 C, and temperature integration, which integrates duration of combustion and temperature above 100 C and serves as a proxy for heat release (de Magalhães and Schwilk 2012, Cornwell et al. 2015). By performing the flammability tests at several moisture levels I were able to determine moisture of extinction (a threshold representing the moisture value above which litter will not ignite or spread) for all species in this study, using a logistic regression.

Burn trials occurred in a cement structure used to simulate house fires at the Fire Department of the City of Lubbock.This structure minimizes wind and helps regulate temperature and relative humidity. Temperature and relative humidity were measured before every trial using a Kestrel 3000 (Nielsen-Kellerman, Boothwyn, PA). To minimize temperature and relative humidity variation, all trials were conducted on clear days between 10 am and 3 pm. 

\subsection{Statistical analyses}


Dry-down curves were created based on the drying experiment, fitted with an exponential decay curve to each species.:

\begin{equation}
m = m_{max} e^{-\lambda t}
\end{equation}

Where $m$ is moisture content at time $t$, and $\lambda$ is the exponential decay coefficient. To estimate these c oefficents we fit a linear mixed-effects model by first taking the natural logarithm of moisture content, with time since wetting and species as fixed effects nested within replicate basket. We extracted the coefficients of the fitted model to describe maximum water retainability ($m_{max}$; g of water per g of dry mass) and desiccation rate ($\lambda$) for each species. All analyses were conducted in R \cite{RCoreTeam-2019}. We fit models with R packages `lme4' \cite{Bates_Machler_etal-2015} and `afex' \cite{Singmann_Bolker_etal-2017}.  Approximate degrees of freedom and p-values were calculated by the Kenward-Roger approximation \cite{Kenward_Roger-1997} which is recommended by Luke \cite{Luke-2017} as producing acceptable Type 1 error rates.

We tested if one leaf trait (specific leaf area) and one litter trait (litter
bulk density) influenced moisture dynamics (maximum water retainability and
desiccation rate) by fitting linear models with species means as observations.
For each response variable we fit a single linear model with two predictors and
an interaction term.

The effect of time since wetting on flammability was modelled using linear regression with the interaction between time since wetting and genus as fixed effects and average vapor pressure deficit in the burning room during that trial as a nuisance covariate. Differences across species were tested with post-hoc Tukey tests. A binomial logistic regression was fitted to the probability of ignition to obtain moisture of extinction.

We investigated non-additive effects on moisture and on flammability by testing if moisture content and flammability of mixtures were predicted by the average of the constituent species. To choose mixtures we first grouped the species according to their maximum water retainability and dessication rate according to estimated marginal means by species using the R package `emmeans' \cite{Lenth-2019}. This resulted in four broad groups of species and we then selected one representative species from each litter type group and produced all four possible three-species litter mixtures using those four species. For each mixture and time since wetting, we predicted moisture content and flame spread rate using the models fit to single species data described above and then calculated an expected moisture content and spread rate as the average of these three values for the mixture. This procedure was done for both moisture content and flammability parameter values of the constituent species. The expectation from the null model was that the difference between observed and predicted values of flammability will be zero. We tested if there were significant departures from the null (zero) with a mixed effects linear model.


%%%%%%%%%%%%%%%%%%%%%%%%%%%%%%%%%%%%%%%%%%
\section{Results}

\subsection{Species specific moisture dynamics and leaf traits}

There are marked differences in the extent of moisture species retain initially following wetting and draining by gravity (all p < 0.0001, Fig. \ref{fig1-drydown} and Table \ref{tab1-drydown}). \emph{Q. kelloggii}, the only broadleaf, is the species that retains the most moisture but has had a rapid dessication rate (. Calculating estimated marginal means following the linear mixed-effects model discriminated the species into three different maximum water retention groups: 1) \emph{Q. kelloggii}; 2) \emph{P. lambertiana} and \emph{C. decurrens}; and 3) \emph{A. magnifica}, \emph{S. giganteum}, \emph{A. concolor}, \emph{P. jeffreyi} and \emph{P. ponderosa}.  There are also differences in desiccation rates across the eight species and estimated mixed linear model slopes discriminated species into three groups from lowest to most rapid: 1) (\emph{C.decurrens}, 2) \emph{Abies spp} and \emph{S. gianteum}, and 3) \emph{Q. kelloggii} and \emph{Pinus spp}.

\begin{figure}[H]
  \centering
  \label{fig1-drydown}
\includegraphics[width=8cm]{figs_tables/fig1_drydown-curves.pdf}
\caption[Dry down curves for eight litter types.]{Lines show best-fit exponential decay curves fit to moisture content on a dry mass basis. Litter samples we saturated by immersion for 24 hours and then allowed to drain by gravity for 3 min before initial weighing. Colors indicate genera and there were significant differences in both }
\end{figure}


\begin{table}[H]
  \caption{Mixed model results for moisture content as a function of time since wetting. Approximate degrees of freedom, pseudo F statistics and p-values were calculated by the Kenward-Roger approximation \cite{Kenward_Roger-1997}.}
  \label{tab1-drydown}
\centering
%% \tablesize{} %% You can specify the fontsize here, e.g., \tablesize{\footnotesize}. If commented out \small will be used.
\input{figs_tables/drydown-tab.ltx}
\end{table}

Across these eight species specific leaf area had a positive effect on maximum water retention (Fig. \ref{maxmc}, linear model p = 0.003 and litter bulk density had a negative effect on desiccation rate (Fig. \ref{fig-bd-di}, linear model p = 0.029). Anova tables are shown in in Fig. \ref{tab-mc-anova} and Fig. \ref{tab-di-anova}.

\begin{figure}[H]
  \centering
  \label{fig-maxmc-di}
\includegraphics[width=16cm]{figs_tables/fig3-SLA-maxMC.pdf}
\caption[Maximum water retention by specific leaf area]{Maximum water retention by specific leaf area (SLA) across eight species.  TODO }
\end{figure}

\begin{table}[H]
  \caption{Linear model results for maximum moisture content as a function of specific leaf area and litter bulk density.}
  \label{tab2-mc-anova}
\centering
%% \tablesize{} %% You can specify the fontsize here, e.g., \tablesize{\footnotesize}. If commented out \small will be used.
\input{figs_tables/mc-anova.ltx}
\end{table}


\begin{figure}[H]
  \centering
  \label{fig-bd-di}
\includegraphics[width=16cm]{figs_tables/fig4_di_by_bd.pdf}
\caption[Litter density effect on dessication rate.]{  TODO }
\end{figure}

\begin{table}[H]
  \caption{Linear model results for estimated dessication rate ($hr^{-1}$) as a function of specific leaf area and litter bulk density.}
  \label{tab2-di-anova}
\centering
%% \tablesize{} %% You can specify the fontsize here, e.g., \tablesize{\footnotesize}. If commented out \small will be used.
\input{figs_tables/di-anova.ltx}
\end{table}


\subsection{Time since wetting effect on flammability}


Time to ignition and spread rate were modeled against time since wetting and using genus as a fixed effect.


[TODO]?  redo in terms of groups.

of time since wetting and genus on time to ignition and spread rate (F10, 171=24.51, p < 2.2e-16 and F10, 171=55. 85, p<2.2e-16 respectively; Figures 2.3 and 2.4). There were also significant interactions of time since wetting and genus on the response variables (time to ignition and spread rate; both p < 2.2e-16). For time since ignition, the negative relationship is driven by Sequoiadendron and Calocedrus, as evidenced by the steeper slopes of the curves. However, if I removed Sequoiadendron and Calocedrus from the models, the interaction of time since wetting and genus disappears for time to ignition (p=0.1643) but is still significant for spread rate (p = 2.05e-6). After drying for 72 hours, Q. kelloggi is always more flammable and more responsive to changes in moisture content, while C. decurrens and S. giganteum are not (Figures 2.3 and 2.4).


We used binomial logistic regression to model the relationship between moisture content and the probability of ignition to determine moisture of extinction for the eight species. Q. kelloggii ignited at all moisture levels tested (10--65\%) while C. decurrens and A. concolor failed to do so across most moisture levels. I assumed the inflection point on the curve at 50\% to allow for the determination of moisture of extinction for the remaining species (15\% for A. concolor and S. giganteum, 20\% for A. magnifica, 35\% for P. lambertiana, and 60\% for P. ponderosa and P. jeffreyi).

\subsection{Effects of moisture and flammability in litter mixtures}

Moisture effects on litter mixtures

Based on visual inspection of the dry down curves and the estimated marginal means for maximum water retainability and desiccation rate, selected four species to represent the full range of litter moisture dynamics for use in the litter mixtures: \emph{Abies concolor}, \

The mixture that does not contain Q. kelloggii (AbCaPi; Fig. 2.5) absorbs less moisture. This result is to be expected considering the result obtained from the single species trials, and reveals that Q. kelloggii, as the species that absorbs more moisture, drives the water retainability behavior of the mixtures it appears in. In terms of maximum water retainability, the mixtures discriminate into three groups: 1) mixture without Q. kelloggii (AbCaPi), 2) mixtures with both A. concolor and Q. kelloggii (AbCaQu and AbPiQu), and 3) mixture without A. concolor (CaPiQu). There is no discrimination between the mixtures in terms of slopes. Though the mixtures with both C. decurrens and P. jeffreyi (AbCaPi and CaPiQu) are significantly different from one another, the remaining two mixtures (AbCaQu and AbPiQu) cannot be distinguished from one another nor from AbCaPi and CaPiQu. The moisture content of a mixture at any given time was also dependent upon the bulk density of the mixture (anova analysis resulted in a better model if bulk density was included as a fixed effect, p=0.01135).


Time since wetting effect on mixture flammability

In general, the flammability of mixtures was higher as time since wetting went up as expected (Table 2.2). The absence of Q. kelloggii in a mixture, in general, lead to slower flame spread rates, longer flaming combustion, and longer ignition times. The combination of both Q. kelloggii and P. jeffreyi in a mixture generated taller flames and shorter flaming combustion.

Non-additivity in moisture content

We examined the relationship between observed moisture content of a mixture, and the moisture content predicted based on the constituent species to determine the existence of non-additive effects in moisture content (Fig. 2.6). The mixture that does not contain Q. kelloggii (AbCaPi) consistently retains more moisture that what would be predicted by the average of its constituent species. In general, mixtures start the drying process drier than expected and after 48 hours are wetter than expected, indicating lower moisture absorption and slower desiccation rates. The non-additive behavior is dependent on the moisture content of the mixture. When moisture content is above 200\%, I observe negative non-additivity for all mixtures containing Q. kelloggii. There is additive behavior when moisture content is between 75 and 150\%, and positive non-additivity below 75\% moisture content.

Non-additivity in flammability

We determined the presence of non-additivity in the flammability of litter mixtures by comparing the flammability of the mixtures to a null model created by the arithmetic average of the predicted individual flammability of each species in the mixture (Figs. 2.7 and 2.8). Both spread rate and time to ignition exhibited non-additivity, mediated by time since wetting. Observed spread rate values for the mixtures were higher than predicted by the null model, though the non-additivity was stronger for mixtures that contained Q. kelloggii. This result is likely driven by the bulk density of the mixtures, which was lower than predicted, making the mixtures’ litter beds more aerated. Non-additivity in time to ignition is strongly influenced by the presence of C. decurrens in the mixture. This species failed to ignite several times in the individual trials and was assigned the maximum number of seconds for the ignition attempt, 180.



This section may be divided by subheadings. It should provide a concise and precise description of the experimental results, their interpretation as well as the experimental conclusions that can be drawn.
\begin{quote}
This section may be divided by subheadings. It should provide a concise and precise description of the experimental results, their interpretation as well as the experimental conclusions that can be drawn.
\end{quote}

%\begin{listing}[H]
%\caption{Title of the listing}
%\rule{\textwidth}{1pt}
%\raggedright Text of the listing. In font size footnotesize, small, or normalsize. Preferred format: left aligned and single spaced. Preferred border format: top border line and bottom border line.
%\rule{\textwidth}{1pt}
%\end{listing}


%%%%%%%%%%%%%%%%%%%%%%%%%%%%%%%%%%%%%%%%%%
\section{Discussion}

\subsection{Leaf traits influence on moisture dynamics }

The species in this study displayed different patterns of both moisture absorption and desorption (with Q. kelloggii and the pines exhibiting higher values for both), which were controlled by different litter traits (leaf thinness and litter bulk density, respectively). Q. kelloggii was the species that absorbed the most moisture, followed by the pines. The species that have a greater absorptive capacity are also the species that are usually more flammable when dry. This indicates that following a rain event, the most flammable species in my study would not be available as fuel immediately, but cannot be discounted as potential fuel. A species capacity to absorb moisture is controlled by leaf particle geometry, specifically the ratio between leaf length and leaf thickness, and not litter bed arrangement. The eight species differed in their desorption capacity. Q. kelloggii and the pines had higher desiccation rates, while C. decurrens exhibited lower desiccation rates. The ANOVA results indicate that leaf thinness best explains variation in maximum water retention, while litter bulk density best explains variation in desiccation rates.

Bulk densities measured here closely match those for these and other species \cite{van_Wagtendonk+Sydoriak+etal-1998,Stephens+Finney+etal-2004}

The overall dry-down process of litter is being controlled by both traits intrinsic to the leaf particles (leaf geometry) and traits intrinsic to the litter bed (bulk density). Decomposition can also affect moisture content in litter. Van Wagner (1969), investigating fuel weathering, found that pine needles had a variation in moisture content 3 to 6 times greater when they lacked their protective wax coating when compared to intact needles.


TODO

With eight species, power to detect species-specific trait effects on these moisture patterns was weak, but we examined one litter trait known to directly influence flammability: litter bulk density. Litter density is largely driven by leaf size in this system \cite{Magalhaes+Schwilk-2012}. 

\subsection{Time since wetting and flammability}

After a rain event, the litter of different species will exhibit different moisture contents. These results show that both time to ignition and spread rate were affected by time since wetting, and the interaction of time since wetting and genus was also significant.

This indicates that there is a species-specific response to flammability based on time since wetting, and the differences are due to different moisture contents at each time interval. Therefore, mere moisture content will not inform on the fire behavior measures of a litter bed. Species composition of the litter must be considered, and time since the last rain event should allow the determination of the potential flammability of an area. 

Time since wetting, as expected, lowered the flammability of the eight species in this study and its effect overwhelms the effect of volatile content. Q. kelloggii and the pines are more flammable at higher moisture contents when compared to volatile-rich species such as C. decurrens, Abies spp. or S. giganteum. This effect is influenced by litter bulk density and is likely to persist in fuel mixtures where the constituent species have different desiccation indices and therefore will have different moisture contents. I have determined moisture of extinction for the eight species in this study. As expected, species that have higher flammability values (Pinus spp.) had higher moisture of extinction than species with lower flammability (Abies spp. and S. giganteum). I used a higher cut-off moisture for the burn trials than the moisture of extinction found in the literature (65\% vs. 27-30\%; \citep{ Rothermel-1972} [TODO Dickinson et al. 2016] in order to be conservative, as moisture content studies as sparse. However, Q. kelloggii was still flammable at ~80-90\% moisture content (values at 72 hours used for the species mixtures), which I did not anticipate. Due to its high moisture content, Q. kelloggii was not burnt at 24 and 48 hours since wetting, which explains the gap shown in the Figures. Though I cannot comment on the exact behavior of Q. kelloggii at 24 and 48 hours, it is unlikely to be more flammable than the pines. This result indicates that though Q. kelloggii might be too wet to start a fire right after a rain event, Q. kelloggii litter dries faster than the other species in this study and will become very flammable within two days.

\subsection{Effects of moisture and flammability in litter mixtures}

We tested the dry-down process and flammability of four unique three-species litter mixtures. These mixtures contained species representative of all dry-down behavior types observed in single species trials, as assessed by the Tukey test. As expected from the single species results, mixtures that contained Q. kelloggii absorbed more water during the saturation period, but there was no difference in desiccation rate among the mixtures. However, when investigating the non-additivity in moisture content, the mixture that did not contain Q. kelloggii retained more moisture than would be predicted by the null model (arithmetic average of the three constituent species in the mixture). In general, non-additivity varied with moisture content, with positive non-additive behavior at lower moisture contents (<75\%, 72-96 hours since wetting occurred) and negative non-additive behavior at very high moisture contents (>200\%, 0 hours since wetting occurred). This result indicates that immediately after a rain event, the moisture content of a mixture will be lower than expected, however these high values for moisture content are unlikely to be found in natural litters. As the mixture dries, the dry-down behavior will approximate the null model, and three to four days after the rain event the dry-down behavior will become non-additive again. The positive non-additivity seems influenced by litter bed characteristics, since the strongest non-additivity was observed for the mixture with two species that have less dense litter: Q. kelloggii and P. jeffreyi (CaPiQu).

The flammability results for the mixtures followed closely the results obtained for the individual species. Mixtures containing Q. kelloggii and P. jeffreyi had faster spread rates and lower times to ignition, resulting in higher overall flammability. These results are influenced by the lower bulk densities of litter mixtures containing those two species. There was a non-additive effect of moisture on flammability for the four mixtures in this study, reinforcing results shown for dry litter \citep{Magalhaes+Schwilk-2012, VanAltena+Logtestjin+etal-2012}. Spread rate and time to ignition had significant non-additivity, and were strongly influenced by the species present in the mixture. Presence of C. decurrens in a mixture marked a big difference between the null model and the prediction. This was due to several failed ignitions for this species when burned individually, which conditioned the predicted values, while all the mixtures ignited. There were marked differences in the non-additivity results from this study and the results from a previous study investigating dry litter samples. The presence of non-additive effects in moisture content leads to non-additive effects in the flammability of moist litter beds. When accounting for this effect, the non-additivity on the flammability of moist litter beds is subtler, when compared to species effects. This result, together with previously published work, suggests that litter flammability is prone to non-additive effects, reinforcing the idea that we cannot approach flammability by lumping species into fuel categories, as is common in fire behavior models. Community flammability is dependent on and particular to its constituent species and cannot be addressed by simply averaging the flammability of all species it comprises. This dynamic view is likely to be increasingly relevant considering ongoing climate change, which predicts changes in community assembly.

% I have found non-additive effects (both positive and negative) in moisture content and in the flammability of wet litter mixtures of mixed-conifer forest. These results complement the efforts of Blauw et al. (2015) for moist litter and add to the literature investigating non-additivity in litter flammability (de Magalhães and Schwilk 2012, van Altena et al. 2012). However, these studies are limited in the scope of the species studied and suggest that non-additivity needs to be investigated at a larger scale, within different systems and comparatively across a larger number of species. Moisture content is considered the single most important driver of litter flammability and the results from my study indicate how leaf and litter traits affect moisture dynamics, and how moisture effects on litter flammability are species-specific. A trait-based perspective provides a better mechanistic understanding of moisture dynamics of litter fuel flammability and fire behavior and is especially important for predicting and understanding fire in plant communities that are changing.


% Authors should discuss the results and how they can be interpreted in perspective of previous studies and of the working hypotheses. The findings and their implications should be discussed in the broadest context possible. Future research directions may also be highlighted.



% %%%%%%%%%%%%%%%%%%%%%%%%%%%%%%%%%%%%%%%%%%
% \section{Conclusions}

% This section is not mandatory, but can be added to the manuscript if the discussion is unusually long or complex.

%%%%%%%%%%%%%%%%%%%%%%%%%%%%%%%%%%%%%%%%%%
% \section{Patents}
% This section is not mandatory, but may be added if there are patents resulting from the work reported in this manuscript.

%%%%%%%%%%%%%%%%%%%%%%%%%%%%%%%%%%%%%%%%%%
\vspace{6pt} 

%%%%%%%%%%%%%%%%%%%%%%%%%%%%%%%%%%%%%%%%%%
%% optional
%\supplementary{The following are available online at \linksupplementary{s1}, Figure S1: title, Table S1: title, Video S1: title.}

% Only for the journal Methods and Protocols:
% If you wish to submit a video article, please do so with any other supplementary material.
% \supplementary{The following are available at \linksupplementary{s1}, Figure S1: title, Table S1: title, Video S1: title. A supporting video article is available at doi: link.}

%%%%%%%%%%%%%%%%%%%%%%%%%%%%%%%%%%%%%%%%%%
\authorcontributions{For research articles with several authors, a short paragraph specifying their individual contributions must be provided. The following statements should be used ``conceptualization, X.X. and Y.Y.; methodology, X.X.; software, X.X.; validation, X.X., Y.Y. and Z.Z.; formal analysis, X.X.; investigation, X.X.; resources, X.X.; data curation, X.X.; writing--original draft preparation, X.X.; writing--review and editing, X.X.; visualization, X.X.; supervision, X.X.; project administration, X.X.; funding acquisition, Y.Y.'', please turn to the  \href{http://img.mdpi.org/data/contributor-role-instruction.pdf}{CRediT taxonomy} for the term explanation. Authorship must be limited to those who have contributed substantially to the work reported.}

%%%%%%%%%%%%%%%%%%%%%%%%%%%%%%%%%%%%%%%%%%
\funding{Please add: ``This research received no external funding'' or ``This research was funded by NAME OF FUNDER grant number XXX.'' and  and ``The APC was funded by XXX''. Check carefully that the details given are accurate and use the standard spelling of funding agency names at \url{https://search.crossref.org/funding}, any errors may affect your future funding.}

%%%%%%%%%%%%%%%%%%%%%%%%%%%%%%%%%%%%%%%%%%
\acknowledgments{In this section you can acknowledge any support given which is not covered by the author contribution or funding sections. This may include administrative and technical support, or donations in kind (e.g., materials used for experiments).}

%%%%%%%%%%%%%%%%%%%%%%%%%%%%%%%%%%%%%%%%%%
\conflictsofinterest{Declare conflicts of interest or state ``The authors declare no conflict of interest.'' Authors must identify and declare any personal circumstances or interest that may be perceived as inappropriately influencing the representation or interpretation of reported research results. Any role of the funders in the design of the study; in the collection, analyses or interpretation of data; in the writing of the manuscript, or in the decision to publish the results must be declared in this section. If there is no role, please state ``The funders had no role in the design of the study; in the collection, analyses, or interpretation of data; in the writing of the manuscript, or in the decision to publish the results''.} 

%%%%%%%%%%%%%%%%%%%%%%%%%%%%%%%%%%%%%%%%%%
%% optional
\abbreviations{The following abbreviations are used in this manuscript:\\

\noindent 
\begin{tabular}{@{}ll}
MDPI & Multidisciplinary Digital Publishing Institute\\
DOAJ & Directory of open access journals\\
TLA & Three letter acronym\\
LD & linear dichroism
\end{tabular}}

%%%%%%%%%%%%%%%%%%%%%%%%%%%%%%%%%%%%%%%%%%
%% optional
\appendixtitles{no} %Leave argument "no" if all appendix headings stay EMPTY (then no dot is printed after "Appendix A"). If the appendix sections contain a heading then change the argument to "yes".
\appendix
\section{}
\unskip
\subsection{}
The appendix is an optional section that can contain details and data supplemental to the main text. For example, explanations of experimental details that would disrupt the flow of the main text, but nonetheless remain crucial to understanding and reproducing the research shown; figures of replicates for experiments of which representative data is shown in the main text can be added here if brief, or as Supplementary data. Mathematical proofs of results not central to the paper can be added as an appendix.

\section{}
All appendix sections must be cited in the main text. In the appendixes, Figures, Tables, etc. should be labeled starting with `A', e.g., Figure A1, Figure A2, etc. 

%%%%%%%%%%%%%%%%%%%%%%%%%%%%%%%%%%%%%%%%%%
% Citations and References in Supplementary files are permitted provided that they also appear in the reference list here. 

%=====================================
% References, variant A: internal bibliography
%=====================================
% \reftitle{References}
% \begin{thebibliography}{999}
% % Reference 1
% \bibitem[Author1(year)]{ref-journal}
% Author1, T. The title of the cited article. {\em Journal Abbreviation} {\bf 2008}, {\em 10}, 142--149.
% % Reference 2
% \bibitem[Author2(year)]{ref-book}
% Author2, L. The title of the cited contribution. In {\em The Book Title}; Editor1, F., Editor2, A., Eds.; Publishing House: City, Country, 2007; pp. 32--58.
% \end{thebibliography}

% The following MDPI journals use author-date citation: Arts, Econometrics, Economies, Genealogy, Humanities, IJFS, JRFM, Laws, Religions, Risks, Social Sciences. For those journals, please follow the formatting guidelines on http://www.mdpi.com/authors/references
% To cite two works by the same author: \citeauthor{ref-journal-1a} (\citeyear{ref-journal-1a}, \citeyear{ref-journal-1b}). This produces: Whittaker (1967, 1975)
% To cite two works by the same author with specific pages: \citeauthor{ref-journal-3a} (\citeyear{ref-journal-3a}, p. 328; \citeyear{ref-journal-3b}, p.475). This produces: Wong (1999, p. 328; 2000, p. 475)

%=====================================
% References, variant B: external bibliography
% =====================================
\externalbibliography{yes}
\bibliography{/home/schwilk/write/bib/schwilk}

%%%%%%%%%%%%%%%%%%%%%%%%%%%%%%%%%%%%%%%%%%
%% optional
%\sampleavailability{Samples of the compounds ...... are available from the authors.}

%% for journal Sci
%\reviewreports{\\
%Reviewer 1 comments and authors’ response\\
%Reviewer 2 comments and authors’ response\\
%Reviewer 3 comments and authors’ response
%}

%%%%%%%%%%%%%%%%%%%%%%%%%%%%%%%%%%%%%%%%%%
\end{document}
