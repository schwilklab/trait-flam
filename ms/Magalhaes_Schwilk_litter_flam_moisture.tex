%% reftex-default-bibliography: ("/home/schwilk/write/bib/schwilk.bib")

% LaTeX support: latex@mdpi.com 
%  In case you need support, please attach all files that are necessary for compiling as well as the log file, and specify the details of your LaTeX setup (which operating system and LaTeX version  tools you are using).

%  =================================================================
\documentclass[fire,article,submit,moreauthors,pdftex]{Definitions/mdpi} 


%\usepackage[utf8]{inputenc}

% If you would like to post an early version of this manuscript as a preprint, you may use preprint as the journal and change 'submit' to 'accept'. The document class line would be, e.g., \documentclass[preprints,article,accept,moreauthors,pdftex]{mdpi}. This is especially recommended for submission to arXiv, where line numbers should be removed before posting. For preprints.org, the editorial staff will make this change immediately prior to posting.

%--------------------
% Class Options:
%--------------------
%----------
% journal
%----------
%=================================================================
\firstpage{1} 
\makeatletter 
\setcounter{page}{\@firstpage} 
\makeatother
\pubvolume{xx}
\issuenum{1}
\articlenumber{5}
\pubyear{2019}
\copyrightyear{2019}
%\externaleditor{Academic Editor: name}
\history{Received: date; Accepted: date; Published: date}
%\updates{yes} % If there is an update available, un-comment this line

%% MDPI internal command: uncomment if new journal that already uses continuous page numbers 
%\continuouspages{yes}

%------------------------------------------------------------------
% The following line should be uncommented if the LaTeX file is uploaded to arXiv.org
%\pdfoutput=1

%=================================================================
% Add packages and commands here. The following packages are loaded in our class file: fontenc, calc, indentfirst, fancyhdr, graphicx, lastpage, ifthen, lineno, float, amsmath, setspace, enumitem, mathpazo, booktabs, titlesec, etoolbox, amsthm, hyphenat, natbib, hyperref, footmisc, geometry, caption, url, mdframed, tabto, soul, multirow, microtype, tikz


%\bibliography{/home/schwilk/write/bib/schwilk.bib}

%=================================================================
%% Please use the following mathematics environments: Theorem, Lemma, Corollary, Proposition, Characterization, Property, Problem, Example, ExamplesandDefinitions, Hypothesis, Remark, Definition
%% For proofs, please use the proof environment (the amsthm package is loaded by the MDPI class).

%=================================================================
% Full title of the paper (Capitalized)
\Title{Moisture effects on leaf litter are species-specific and result in non-additive flammability in mixed conifer forest}

% Author Orchid ID: enter ID or remove command
\newcommand{\orcidauthorA}{0000-0000-000-000X} % Add \orcidA{} behind the author's name
%\newcommand{\orcidauthorB}{0000-0000-000-000X} % Add \orcidB{} behind the author's name

% Authors, for the paper (add full first names)
\Author{Rita Quinones de Magalhaes $^{\dagger}$ and Dylan W. Schwilk}

% Authors, for metadata in PDF
\AuthorNames{Rita Magalh\~{a}es, Dylan W. Schwilk}

% Affiliations / Addresses (Add [1] after \address if there is only one affiliation.)
\address[1]{Texas Tech University}

% Contact information of the corresponding author
\corres{Correspondence: dylan.schwilk@ttu.edu, DWS}

% Current address and/or shared authorship
\firstnote{Current address: Rochester Institute of Technology, Rochester, NY 14623} 
%\secondnote{These authors contributed equally to this work.}
% The commands \thirdnote{} till \eighthnote{} are available for further notes

%\simplesumm{} % Simple summary

%\conference{} % An extended version of a conference paper

% Abstract (Do not insert blank lines, i.e. \\) 
\abstract{ Moisture content is a strong determinant of forest fuel flammability
  and varies on small spatial scales and over short time periods. Emperical
  modeling of equilibrium dead fine fuel moisture is well established, but less
  is known about the mechanisms by which species-specific litter traits
  influence litter moisture dynamics, particularly the processes of moisture
  absorption and desorption. Characterizing how species’ leaf litter retain and
  release moisture through time is critical to the assessment of fuel’s
  availability to burn. Furthermore, Natural forests are often comprised of
  mixed stands, leading to mixtures of leaf litter on the surface fuelling the
  fires. Multiple studies have established the existence of non-additive
  effects in the flammability of dry litter, but we lack a mechanistic
  understanding of moisture synergies in mixed litter and of moisture effects
  on the flammability of mixed litter. The litter from eight species of a
  mixed-conifer forest was saturated and allowed to dry to determine moisture
  absorption capacity and desorption rates. Burn trials were performed in moist
  litter beds of single species and of mixtures, to establish flammability
  response to the dry-down process. Moisture dynamics vary across species with
  higher specific leaf area being associated with higher maximum absorption and
  lower litter bulk density being associated with faster drying rates. Species
  that produce more aerated litter beds (such as \emph{Q. kelloggii} and
  \emph{Pinus} species) have faster rates of litter drying. We examined two
  axes of flammability (spread rate and fuel consumption) and found that these
  differing moisture dynamics result in time since wetting having strongly
  differing effects on flammability across the major litter types typified by
  different tree genera. Finally, we found that litter mixtures exhibited
  consistent non additivity in flame spread rate in fuel consumption. This non
  additivity became more positive with fuel drying. Litter mixtures burn with
  fire behavior more similar to that of the most flammable constituent species
  and this effect increases as fuels dry.}
  
% Keywords
\keyword{flammability, litter, flame spread rate}

%\setcounter{secnumdepth}{4}
%%%%%%%%%%%%%%%%%%%%%%%%%%%%%%%%%%%%%%%%%%
\begin{document}
%%%%%%%%%%%%%%%%%%%%%%%%%%%%%%%%%%%%%%%%%%

\section{Introduction}

Fuel moisture content is an important, and often driving, determinant of
surface fire behavior. Environmental conditions such as temperature, relative
humidity, and wind drive the moisture content of the fuel available to burn
\cite{Kreye+Varner+etal-2018}. In particular, fine dead fuels respond very
readily to environmental changes \cite{Nelson-2001}. Fuel moisture influences
flammability and fire behavior through effects on ignition, spread rate, flame
height, fuel consumption, and heat release, due to the high specific heat
content of water \cite{Rothermel-1972, Nelson-2011}. Moisture acts as a
heat-sink resulting in less energy available for propagation, and the resulting
water vapor will both cool the flaming front and dilute flammable gases
\cite{Albini-1976, Shafizadeh-1977}. Variation in moisture of living fuels has
strong effects on flammability, but moisture of dead excised litter fuels can
also vary. Moisture of litter fuels varies with climate, with seasons and with
short term weather within a season. Therefore, the rate of drying and
flammability at any particular moisture content are key parameters that
influence fuel flammability throughout the fire season \cite{Nelson-2001}.

Fire scientists often categorize fuel by size classes that correspond to moisture exchange rates, although such simplified categories have long been recognized as not sufficient to describe species specific and decomposition effects on moisture dynamics \cite{Anderson-1985}. Fuel moisture dynamics of leaf litter are dependent upon forest structure (shading and rain throughfall effects), litter physical structure, and chemical composition of the fuelbeds \cite{Nelson+Hiers-2008, Matthews-2014, Kreye_Hiers_etal-2018}. A variety of models describing moisture dynamics of fine fuels are used by fire scientists \cite{Viney-1991, Nelson-2000, Catchpole+Catchpole+etal-2001}. These models usually aim to describe equilibrium conditions and at least  include temperature and relative humidity although some include precipitation and solar radiation effects as well. Explicit models of fire behavior as a function of time since wetting could improve fire prediction, however. In many litter fuel driven systems, natural lightning ignitions are concurrent with precipitation and, therefore, fire behavior can depend on rate of drying following a wetting event. If species differ in the rate of drying, this may strongly influence relatively flammability across species at any particular time post-wetting. Such differential moisture content across time, may contribute to the often high spatial heterogeneity of surface fires \cite{Knapp_Schwilk_etal-2006, Kreye_Hiers_etal-2018}.

Litter packing (measured as litterbed density or packing ratio) controls the flammability of dry litter fuels in seasonally dry climates (higher litter density reduces flammability) and leaf litter packing is influenced by leaf size and shape \cite{Fonda+Belanger+etal-1998, Kane+Varner+etal-2008, Schwilk+Caprio-2011, Kreye+Varner+etal-2013}. Fine fuel moisture content reduces ignition probability and flame spread rate \cite{Gisborne-1936, Fons-1946, Anderson+Rothermal-1965}. However, the mechanism by which both monospecific and mixed litterbeds absorb and desorb moisture is poorly known, and could be driven by by leaf traits, ermegent litter litter traits, or both. Species specific effects on litter flammability are recorded for the forests of the southeastern US \cite{Nowacki+Abrams-2008} where species-specific litter drying rates play a role in flammability feedbacks \cite{Kreye+Varner+etal-2013}. Similarly, in mixed-conifer forest of the western USA, where the fire regime is characterized by surface fires fueled by plant litter, overstory tree composition drives variation in local fire behavior \cite{Schwilk+Caprio-2011} as a result of differential litter flammability cross tree species and as a result of synergistic interactions in multi-species litter mixtures \cite{Magalhaes+Schwilk-2012}. These flammability feedbacks may help explain historical fire frequency patterns, but current estimates of litter flammability in mixed conifer forests are based largely on burning dry litter.

Fuel moisture content through time is controlled by two main processes, absorptive capacity and rate of moisture loss \citep{Kreye-2013}. Absorptive capacity represents the potential maximum fuel moisture and is evaluated by the moisture content of litter when saturated. After saturation, the differences in moisture content indicate different abilities to retain moisture. We propose two different sets of traits responsible for variation in the two processes mentioned: leaf particle geometry and litter bulk density. Litter characteristics such as surface-area-to-volume ratio, litter bulk density, mass density, particle size and shape, and chemical traits influence moisture diffusion through litter [CITE]. Litter packing has two pathways of effect on fire behavior: directly by its effect on oxygen availability for combustion, and indirectly by its effect on dry-down rate, and both these pathways are affected by leaf litter traits, especially leaf size \cite{Scarff+Westoby-2006}. Fuelbeds composed of larger leaves with lower moisture retainability and decomposition rates favor drying, thus increasing the probability of ignition and fire spread rate.

Flammability is a multi-dimensional characteristic of fuels \citep{Schwilk-2015, Pausas+Keeley+etal-2017} with different traits influencing different components of fire behavior. In litter driven fire, the two major axes of variation are described by measurements related to flame spread rate and measurements related to heat transfer. Flame spread rate is easy to measure in burning trials, but heat transfer can be more difficult and often proxies such as duration of heating are used \citep{Magalhaes+Schwilk-2012, Varner+Kane+etal-2015}. These different components of fire behavior have differing ecological effects. Species traits may interact such that the flammability of multi-species mixtures is not the average of the constituent species. Such non-additive effects have been shown to be synergistic in dry litter fuels where the flammability of a mixture is driven by the most flammable species  \cite{VanAltena+Logtestjin+etal-2012, Magalhaes+Schwilk-2012}. It is possible that similar non-additive effects may influence litter moisture content, especially if different leaf traits influence adsorptive capacity and drying rate. There is some evidence for non addiitvity of flammability in moist fuels and one study has found increasing non additivity with moisture \cite{Blauw+Wensink+etal-2015}.

We investigated species specific litter moisture dynamics across eight tree species in a temperate fire-prone forest in California. We determined the maximum water retention and rate of moisture loss and examined how these vary in relation to two leaf and litter bed traits.  We then examined the influence of moisture on the flammability across these species and how flame spread rate and fuel consumption varied with time since wetting and by taxon. Experimental burn trials of leaf litter at various moisture levels determined how time since wetting influenced two axes of flammability characterized by flame spread rate and total fuel consumed. Previous work described the flammability of these species in monocultures and in mixtures for oven-dry litter \cite{Magalhaes+Schwilk-2012} and found consistent positive non additivity of flammability where fire behavior of mixtures was driven by the most flammable species. Here, we examined if this positive non additivity was maintained in mixtures across varying levels of fuel moisture. The burn trials were also performed with multi-species litter mixtures to evaluate if species contribute equally to litter moisture dynamics or if there are non-additive effects as seen in oven-dried litter. We tested if the non-additive effects on  reported in previous studies occur under ecologically-relevant moisture levels and whether fuel moisture effects strengthen or mask such non-additivity.


%%%%%%%%%%%%%%%%%%%%%%%%%%%%%%%%%%%%%%%%%%
\section{Materials and Methods}

\subsection{Site description and species selection}

Field sites for this study were located in Sequoia and Kings Canyon National Parks, California, USA (36 36N, 118 42 W) between 1600 and 2400 m elevation in mixed-conifer forest. We chose eight dominant tree species that are representative of this type of forest: \emph{Pinus jeffreyi} Grev. \& Balf., \emph{Pinus lambertiana} Dougl., \emph{Pinus ponderosa} Dougl. ex Laws., \emph{Abies concolor} (Gord. \& Glend.) Lindl. ex Hildebr., \emph{Abies magnifica} A. Murr., \emph{Calocedrus decurrens} (Torr.) Florin, \emph{Quercus kelloggii} Newb., and \emph{Sequoiadendron giganteum} (Lindl.) J. Buchholz. In this study, we consider litter the top layer of leaves and small twigs less than 0.625 cm diameter (= 1-hour fuel) that have fallen in the previous year (mostly undecomposed). Leaf litter was collected in the summer (mid-June to mid-July) in 2011 and in 2012 from 21 separate sites across the study area. Effort was made to collect from different populations (minimum of four, each population separated from the others by 9 to 32\,km) for each species to control for plastic and ecotypic effects. An exception to this involved the litter from \emph{A. magnifica} which is restricted to the higher elevations and could only be collected from one area, due to logistical constraints. Within that area, \emph{A. magnifica} litter was collected from over 10 different individuals. For the remaining species, the collection involved 2-4 individual trees at each site, separated by at least 10\,m. The litter was collect about 2\,m away from the trunk to obtain a more uniform sample, because bark and heavier branches tend to fall closer to the tree trunk. All litter was air dried down to < 5\% fuel moisture for weighing and dividing into moisture trials. The moisture manipulations and subsequent burning trials were conducted in April-August 2015.

\subsection{Moisture absorption and desorption}


We produced dry-down curves to determine maximum water retainability and the
rate of moisture content loss for each species and for four distinct
three-species mixtures. These curves allowed us to determine the rate at which
a species loses moisture, approximated as exponential decay, and to determine
maximum water content following saturation and gravity draining. Samples from
the eight species (six replicates per species) and four mixtures (5 replicates
per mixture type) were placed in baskets (approximately 45 x 45\,cm by 15\,cm
tall) constructed of 1 cm aluminum screen. Samples were 450 g of litter (dry
weight) per trial, and litter depth was standardized to 10\,cm by changing the
horizontal dimensions of the basket. Weight was measured by using a balance
sensitive to 0.1\,g (model XS16001L, Mettler Toledo, Columbus, OH). Samples were
immersed in plastic storage bins filled with water for 24 hours to fully
saturate them. After saturation, the baskets were removed from the water,
allowed to drain for 3 minutes, and samples weighed to measure maximum moisture
content. The samples were then allowed to dry in a controlled environment kept
at 21\,C temperature and 30\% relative humidity, assessed via an iButton sensor
(Maxim Integrated) placed over the samples. Litter baskets were arranged in a
completely randomized design. During the dry-down process three subsamples were
taken at 24 hour intervals and weighed. These subsamples were then oven dried
for 24 hours at 100\,C and re-weighed to assess moisture content. This was
repeated until the final sample from a basket was less than 10\% moisture.


% The dry-down curves indicated that the eight species varied in their moisture behavior (Fig \ref{fig-drydown}). To determine a subset of species to use in the mixture trials, we performed analysis of contrasts using lsmeans package in R (TODO Lenth 2016) to guide choosing a subset of species for the litter mixtures. Based on the marginal means analysis, we chose four species to maximize variation in water retention and drying rate. Creating all possible three species mixtures from these four species resulted in four distinct three-species mixtures. 

% Leaf trait data

With only eight species in this study, our power to test leaf trait effects on
the observed moisture dynamics was low, so we decided \emph{a priori} to test
the effect of one leaf trait likely to influence moisture absoorption, specific
leaf area, and one litter trait known to directly effect flammability and
likely to influence drying rate, litter bulk density. Specific leaf area was
taken from leaf fresh leaf area and leaf dry mass from 5 individuals per
species and was collected as part of a previous study
\cite{Magalhaes+Schwilk-2012}. Litter bulk density was obtained in this
experiment for each basket by dividing the dry mass (450\,g) by the estimated
litter volume (basket width x length x litter depth).

\subsection{Flammability assessment}

We performed flammability tests across multiple moisture levels on monospecific
litter beds and on mixed litter beds with each species or mixture replicated 5
times at each moisture level. We used the initial dry-down experiment to guide
our timeing so as to attempt to burn samples over a range of moisture contents.
We aimed for moisture contents from 10\% to 100\% on a dry mass basis as
preliminary trials on damp fuels suggested that across these fuels ignition is
unlikely above 100\% fuel moisture. To expand our data on the dry end of this
gradient, we supplemented these data with data from an earlier experiment on
oven-dried litter \cite{Magalhaes+Schwilk-2012}. This supplemental data added
five replicates per species at an oven dried moisture content of $<$ 5\%. We
assigned these replicates a time since wetting value of 144 hours equivalent to
the longest time since drying for the wetted samples. Flammability was assessed
using a 150\,cm long burn table, 15\,cm wide and 15\,cm tall, in which leaf
litter was placed and the table gently shaken to allow settling. Burning trials
all used 450\,g (dry weight) of litter. For three-species mixtures, this was
150\,g per species. This design allows fire to reach constant flame spread
rates \cite{Magalhaes+Schwilk-2012}. Two graduated metal rulers equally spaced
along the apparatus allowed visual determination of maximum flame height.
Litter was ignited with a propane torch; maximum adiabatic flame temperature of
1899 C, \url{http://www.benzomatic.com/products/fuel.aspx}) at one end of the
apparatus and allowed to burn until extinction. If a sample failed to ignite after 30 seconds, no further attempt was made and if three samples of a species at a particular mositure level failed to ignite, no further attempts were made on that species/moisture combination and the replicate was allowed to continue drying. A timer was used to record total time to ignition and duration of flaming combustion. The flammability trials followed the methodology detailed in de
Magalhães and Schwilk \cite{Magalhaes+Schwilk-2012}.

% and nine type-K thermocouples attached to data loggers measured temperature every second at three distances from the ignition point and at three different depths below the litter surface.
 
During flammability trials, we recorded time to ignition (s), duration of
flaming combustion (s), flame spread rate (mm/s), calculated by dividing the
length of the burned surface by the time it takes the flaming front to reach
the end of the apparatus, maximal flame height (mm), and fuel consumption (\%).
Some trials resulted only in smoldering combustion; for those trials, a time to
ignition and mass loss were recorded, but flame spread rate and flame height
were recorded as zero. Our thermocouple system failed or data was lost for two
days of burning trials, therefore we have omitted those data from this
analysis. Our past work \cite{Magalhaes+Schwilk-2012} demonstrated that flame
spread rate captures one important axis of flammability variation
\citep{Schwilk-2015, Pausas+Keeley+etal-2017} and percent fuel consumed
captures a portion of the second major axis of flammability variation (although
not as well as integrating temperature over time does). Our analysis,
therefore, focused on these two relatively orthogonal measures of flammability
to avoid examining multiple redundant dependent variables.

Burn trials occurred in a cement structure used to simulate house fires at the
Fire Department of the City of Lubbock. This structure minimized wind and helped
regulate temperature and relative humidity. Temperature and relative humidity
were measured before every trial using a Kestrel 3000 (Nielsen-Kellerman,
Boothwyn, PA). All trials were conducted on clear days between 10 am and 3 pm.

\subsection{Statistical analyses}

Dry-down curves were created based on the drying experiment, fitted with an exponential decay curve to each species.:

\begin{equation}
m = m_{max} e^{-\lambda t}
\end{equation}

Where $m$ is moisture content at time $t$, and $\lambda$ is the exponential
decay coefficient. To estimate these c oefficents we fit a linear mixed-effects
model by first taking the natural logarithm of moisture content, with time
since wetting and species as fixed effects nested within replicate basket. We
extracted the coefficients of the fitted model to describe maximum water
retainability ($m_{max}$; g of water per g of dry mass) and desiccation rate
($\lambda$) for each species. All analyses were conducted in R
\cite{RCoreTeam-2019}. We fit models with R packages `lme4'
\cite{Bates_Machler_etal-2015} and `afex' \cite{Singmann_Bolker_etal-2017}.
Approximate degrees of freedom and p-values were calculated by the
Kenward-Roger approximation \cite{Kenward_Roger-1997} which is recommended by
Luke \cite{Luke-2017} as producing acceptable Type 1 error rates. We tested if
one leaf trait (specific leaf area) and one litter trait (litter bulk density)
influenced moisture dynamics by fitting linear models with species means as
observations. For each response variable we fit a single linear model with two
predictors and an interaction term.

Our past work \cite{Magalhaes+Schwilk-2012} demonstrated that dry litter
flammability across these eight species was driven by leaf traits and litter
packing. These previously reported flammability trials demonstrated that species within a genus tended to behave similarly and the two scale-leaved genera in the Cupressaceae,
\emph{Calocedrus} and \emph{Sequoiadendron}, also behaved similarly to one
another. Therefore, to preserve power and focus on analyses we used taxonomic groups (\emph{Abies}, Cupressaceae, \emph{Quercus}, and \emph{Pinus}) as our flammability functional groups and treated these as fixed effects in our models. The effect of fuel moisture and
time since wetting on flammability parameters were modeled using linear
models with fuel moisture or time since wetting and taxon as fixed effects and
average vapor pressure deficit in the burning room during that trial as a
nuisance covariate. We used linear mixed models to predict spread rate, but
fuel consumption tends to exhibit threshold effects with few intermediate
values. Therefore, we used mixed generalized linear models with a beta
regression link with the `glmmADMB` package \cite{Skaug_Fournier_etal-2016} to
model percent fuel consumed as a function moisture content and of time since
wetting.

We investigated possible non-additive effects on moisture and on flammability
by testing if moisture content and flammability of mixtures were predicted by
the average of the constituent species. To choose mixtures, we first grouped the
species according to their maximum water retainability and dessication rate
according to estimated marginal means by species using the R package `emmeans'
\cite{Lenth-2019}. This resulted in four broad groups of species and we then
selected one representative species from each litter type group and produced
all four possible three-species litter mixtures using those four species. For
each mixture and time since wetting, we predicted moisture content based on the
mean content of the three consituent species at that time since wetting. We
predicted flammability parameters based on the models fit to single species
data described above and then calculated an expected moisture content and
spread rate as the average of these three values for the mixture. The
expectation from the null model was that the difference between observed and
predicted values of flammability will be zero. We tested if there were
significant departures from the null (zero) with a mixed effects linear model.
For flammability parameters, we fit linear mixed effect models with the
observed-predicted flammability value as the dependent variable and used Wald
tests to test if the centered intercept was different than zero (overall non
additivity) and if the slope with time since wetting was different from zero
(non additivity changing with drying).

%%%%%%%%%%%%%%%%%%%%%%%%%%%%%%%%%%%%%%%%%%
\section{Results}

\subsection{Species specific moisture dynamics and leaf traits}

There were marked differences in the extent of moisture species retain initially
following wetting and draining by gravity (all p < 0.0001, Fig.
\ref{fig:drydown} and Table \ref{tab:drydown}). \emph{Q. kelloggii}, the only
broadleaf, is the species that retained the most moisture initially.
\emph{Quercus}, like the pines, had a rapid drying rate (Fig.
\ref{fig:drydown} and SI Fig. 1). Calculating estimated marginal means following the
linear mixed-effects model discriminated the species into three different
maximum water retention groups: 1) \emph{Q. kelloggii}; 2) \emph{P.
  lambertiana} and \emph{C. decurrens}; and 3) \emph{A. magnifica}, \emph{S.
  giganteum}, \emph{A. concolor}, \emph{P. jeffreyi} and \emph{P. ponderosa}
(SI Fig. 1). There are also differences in desiccation rates across the eight
species and estimated mixed linear model slopes discriminated species into
three groups from lowest to most rapid: 1) (\emph{C.decurrens}, 2) \emph{Abies
  spp} and \emph{S. gianteum}, and 3) \emph{Q. kelloggii} and \emph{Pinus spp}
(SI Fig. 2).

\begin{figure}[H]
  \centering
\includegraphics[width=8cm]{figs_tables/fig1_drydown-curves.pdf}
\caption{Dry down curves for eight litter types. Lines show best-fit
  exponential decay curves fit to moisture content on a dry mass basis. Litter
  samples we saturated by immersion for 24 hours and then allowed to drain by
  gravity for 3 min before initial weighing. Colors indicate genera and there
  were significant differences in both maximum water retention and in rate of drying (based linear model on log of water content). A small amount of horizontal jitter
  (0.5 hr) was added to this plot to aid in visualizing because, although
  collections were staggered, there were often multiple trays collected at a
  single time point.}
 \label{fig:drydown}
\end{figure}


\begin{table}[H]
  \caption{Linear mixed model results for moisture content as a function of time since
    wetting. Approximate degrees of freedom, pseudo F statistics and p-values
    were calculated by the Kenward-Roger approximation.
    \cite{Kenward_Roger-1997}.}
  \label{tab:drydown}
\centering
%% \tablesize{} %% You can specify the fontsize here, e.g., \tablesize{\footnotesize}. If commented out \small will be used.
\input{figs_tables/tab1_drydown_anova.ltx}
\end{table}


Across these eight species specific leaf area had a positive effect on maximum
water retention (Fig. \ref{fig:maxmc-di}, linear model p = 0.003) and litter
bulk density had a negative effect on desiccation rate (Fig. \ref{fig:bd-di},
linear model p = 0.029). Anova results are shown in Rables \ref{tab:mc_di_anova}.

\begin{figure}[H]
  \centering
\includegraphics[width=16cm]{figs_tables/fig2_SLA_maxMC.pdf}
\caption{Maximum water retention by specific leaf area (SLA) across eight species. ANOVA table in Table \ref{tab:mc_di_anova}}
  \label{fig:maxmc-di}
\end{figure}


\begin{figure}[H]
  \centering
\includegraphics[width=16cm]{figs_tables/fig3_di_bd.pdf}
\caption{Linear model predicting dessication rate as a function of specific leaf area and litter bulk density across eight species. ANOVA table in Table \ref{tab:mc_di_anova}}.
  \label{fig:bd-di}
\end{figure}



\begin{table}[H]
  \caption{Linear model results for maximum water retention (\% moisture content on dry mass basis) and for estimated desiccation rate (hr$^{-1}$) as functions of specific leaf area and litter bulk density across eight species. ANOVA tables shown with degrees of freedom, sum of squares, mean squares, F, and p for each effect.}
  \label{tab:mc_di_anova}
\centering
%% \tablesize{} %% You can specify the fontsize here, e.g., \tablesize{\footnotesize}. If commented out \small will be used.

\begin{tabular}{ccrrrrr}
  \toprule
Ind. variable & Effect & df & Sum of Squares & Mean squares & F & p value \\ 
  \midrule
Moisture content & SLA & 1 & 250834.5 & 250834.51 & 130.13 & 0.000 \\ 
  & density & 1 & 538.5 & 538.54 & 0.28 & 0.625 \\ 
  & SLA:density & 1 & 5135.2 & 5135.24 & 2.66 & 0.178 \\ 
  & Residuals & 4 & 7710.4 & 1927.61 &  &  \\ 
  \midrule
  
Desiccation rate & density & 1 & $3.92 \times 10^{-4}$ & $3.92 \times 10^{-4}$ & 11.01 & 0.029 \\ 
 &  SLA & 1 & $1.88 \times 10^{-5}$ & $1.88 \times 10^{-5}$ & 0.53 & 0.507 \\ 
 &  density:SLA & 1 & $6.89 \times 10^{-6}$ & $6.89 \times 10^{-6}$ & 0.19 & 0.683 \\ 
 &  Residuals & 4 & $1.42 \times 10^{-4}$ & $3.56 \times 10^{-5}$ &  &  \\ 
   \bottomrule

\end{tabular}
\end{table}


% \begin{table}[H]
%   \caption{Linear model results for estimated dessication rate ($hr^{-1}$) as a function of specific leaf area and litter bulk density. Anova table shown with degrees of freedom, sum of squares, mean squares, F, and p for each effect}
%   \label{tab:di_anova}
% \centering
% %% \tablesize{} %% You can specify the fontsize here, e.g., \tablesize{\footnotesize}. If commented out \small will be used.
% \input{figs_tables/tab3_di_anova.ltx}
% \end{table}


\subsection{Moisture effects on flammability}


Flame spread rate declined with fuel moisture (Fig. \ref{fig:spread_moist}). \emph{Quercus} had higher spread rates across all fuel moistures and had a steeper decline in response to moisture than did the other groups (taxon by time since wetting interaction p = 0.022, Table \ref{tab:spreadrate_anova}). Fuel moisture negatively influenced fuel consumption. Overall fuel consumption and the rate of change with moisture content varied across the taxonomic groups (Fig. \ref{fig:consume_moist}, Table \ref{tab:consume_moist_anova}). Although the overall ranking of groups was similar for fuel consumption and for flame spread rate, in contrast to the pattern with spread rate, for fuel consumption, \emph{Pinus} and \emph{Abies} has a steeper decline in flammability with fuel moisture than did the other groups.

%% spread rate by moisture
\begin{figure}[H]
  \centering
\includegraphics[width=8cm]{figs_tables/fig4_spread_actualMC.pdf}
\caption{Flame spread rate by moisture content (model fit F= 61.15, d.f = 9, 148,  p < 0.0001). Model ANOVA table in Table \ref{tab:spreadrate_anova} and coefficients in SI Table 2.}
  \label{fig:spread_moist}
\end{figure}

\begin{table}[H]
  \caption{Generalized linear mixed model results for flame spread rate as a function of fuel moisture with taxonomic group as a fixed effect and species as a nested random effect. Approximate degrees of freedom, pseudo F statistics and p-values were calculated by the Kenward-Roger approximation \cite{Kenward_Roger-1997}. Coefficient estimates are in SI Table 2.}
  \label{tab:spreadrate_anova}
\centering
\input{figs_tables/tab4_spread_moist_anova.ltx}
\end{table}

%% fuel consumption by moisture:
\begin{figure}[H]
  \centering
\includegraphics[width=8cm]{figs_tables/fig5_consume_actualMC.pdf}
\caption{Percent fuel consumed by moisture content. Lines indicate best fit beta regression lines. Model deviance table in Table \ref{tab:consume_moist_anova} and coefficients in SI Table 3.}
  \label{fig:consume_moist}
\end{figure}


\begin{table}[H]
  \caption{Generalized linear mixed model results for fuel consumption as a function of fuel moisture with taxonomic group as a fixed effect and species as a nested random effect.  The analysis of deviance table below shows deviance comparisons for the simplest null model with only intercept and random effect (model 1), for a taxon effect only (model 2), for a taxon and mosture content effect (model 3) and for the full model including the taxon by moisture interaction (model 4). Coefficient estimates are in SI Table 3.}
  \label{tab:consume_moist_anova}
\centering
\input{figs_tables/tab5_consume_moist_anova.ltx}
\end{table}

As a consequence of the relationships shown above, time since wetting had a strong effect on spread rate (p < 0.0001) and the slope of this effect varied by taxonomic group (Fig. \ref{fig:spread_time}, Table \ref{tab:spreadrate_anova}). The pines were able to ignite and burn much sooner after wetting than the other genera, none of which had measurable spread rates until burning was attempted 72 hours after wetting (note that trials that failed to even ignite smoldering combustion are ommitted completely from the model and figure). At that point, the \emph{Abies}, \emph{Calocedrus}, and \emph{Sequiadendron} had dryer litter than did the \emph{Quercus}, yet the spread rates were slow. \emph{Quercus} had the steepest response to time since wetting (Fig. \ref{fig:spread_time}, SI Table 4)).


\begin{figure}[H]
  \centering
\includegraphics[width=8cm]{figs_tables/fig6_spread_time.pdf}
\caption{Flame spread rate by time since wetting for litter of eight species.
  Lines indicate best fit linear model fits. Model ANOVA table in Table
  \ref{tab:spread_time_anova} and coefficients in SI Table 4. Horizontal
  jitter (up to 2 hr) was added to the plot for visualization and to avoid
  overplotting. All collections were at the times labeled on the x-axis.. A
  lack of data at earlier time points for a taxon indicates no ignitions
  occurred at those time points.}
  \label{fig:spread_time}
\end{figure}


\begin{table}[H]
  \caption[Mixed linear model of flame spread rate.]{Mixed model results for flame spread rate  as a function of time since wetting with genus as a fixed effect and species as a nested random effect. Approximate degrees of freedom, pseudo F statistics and p-values were calculated by the Kenward-Roger approximation \cite{Kenward_Roger-1997}. Estimated coefficients are in SI Table 4}
  \label{tab:spread_time_anova}
\centering
%% \tablesize{} %% You can specify the fontsize here, e.g., \tablesize{\footnotesize}. If commented out \small will be used.
\input{figs_tables/tab6_spread_time_anova.ltx}
\end{table}

Fuel consumption was highly variable across trials and intermediate values were less common than very low are moderately high consumption (Fig. \ref{fig:consume_time}). All parameters significantly contributed to the generalized linear mixed model as indicated by the nested model comparisons in Table \ref{tab:consume_time_anova}. As with flame spread rate some pines had the most rapid response to drying. However, unlike in the other flammability analyses, here the pines did not all behave similarly: \emph{P. lambertiana} had lower fuel consumption than did \emph{P. ponderosa} and \emph{P. jeffreyii}. We noticed this during trials and in model fitting we found that placing \emph{P. lambertiana} in its own group allowed model convergence that was otherwise impossible. This breaks our intent to use only \emph{a priori} taxonomic groups, but we could not obtain model convergence otherwise.


\begin{figure}[H]
  \centering
\includegraphics[width=8cm]{figs_tables/fig7_consume_time.pdf}
\caption{Percent fuel consumed by time since wetting for litter of eight
  species. Curves shown indicate best fit beta regression lines. Model ANOVA
  table in Table \ref{tab:consume_time_anova} and coefficients in SI Table
  4. Horizontal jitter (up to 2 hr) was added to the plot for visualization and
  to avoid overplotting. All collections were at the times labeled on the
  x-axis.. A lack of data at earlier time points for a taxon indicates no
  ignitions occurred at those time points.}
  \label{fig:consume_time}
\end{figure}

\begin{table}[H]
  \caption{Generalized linear mixed model results for fuel consumption as a function of time since wetting with taxonomic group as a fixed effet and species as a nested random effect. The table shows sequential chi-square tests on nested beta regression models. The analysis of deviance table below shows deviance comparisons for the simplest null model with only intercept and random effect (model 1), for a taxon effect only (model 2), for a taxon and mosture content effect (model 3) and for the full model including the taxon by moisture interaction (model 4). Coefficient estimates for the full model are in SI Table 5.}
  \label{tab:consume_time_anova}
\centering
\input{figs_tables/tab7_consume_time_anova.ltx}
\end{table}

% \begin{table}[H]
%   \caption{Generalized linear mixed model results for fuel consumption as a function of time since wetting with taxonomic group as a fixed effect and species as a nested random effect. These are beta regression models. The analysis of deviance table below shows deviance comparisons for the simplest null model with only intercept and random effect (model 1), for a taxon effect only (model 2), for a taxon and mosture content effect (model 3) and for the full model including the taxon by moisture interaction (model 4). Coefficient estimates are in SI Table 5.}
%   \label{tab:consume_time_anova}
% \centering
% \input{figs_tables/tab6_consume_time_anova.ltx}
% \end{table}

\subsection{Non additivity in  litter mixtures}

Based on visual inspection of the dry down curves and the estimated marginal means for maximum water retainability and desiccation rate (SI Figs 2-3), we selected four species to represent the full range of litter moisture dynamics for use in the litter mixtures and examined all four three-species combinations of those four. The four species were \emph{Abies concolor}, \emph{Calocedrus decurrens}, \emph{Pinus jeffreyi} and \emph{Quercus kellogii}.

\begin{figure}[H]
  \centering
\includegraphics[width=8cm]{figs_tables/fig8_mixture_drydown-curves.pdf}
\caption{Fuel moisture by time since saturation for four litter mixtures.  Codes for species in mixtures: Ab = \emph{Abies concolor}, Ca = \emph{Calocedrus decurrens}, Pi = \emph{Pinus jeffreyi} and Qu = \emph{Quercus kellogii}. Moistures were measured every 24 hours but data are shown with slight horizontal jitter to reduce point overlap. Mixed linear model results shown in Table \ref{tab:mixtures_drydown}.}
  \label{fig:mixture_dry_down}
\end{figure}

Mixture dry down rates and maximum water retention showed some relationship to the dry down curves for individual constituent species (Fig. \ref{fig:mixture_dry_down}. For example, the mixture that does not contain \emph{Q. kelloggii} (AbCaPi in Fig. Fig. \ref{fig:mixture_dry_down}) has the lowest initial retention. Initial water retention and rates differed across mixtures (linear model based on log of moisture content, Table \ref{tab:mixtures_drydown}).

\begin{table}[H]
  \caption{Linear mixed model results for moisture content as a function of time since
    wetting for four different litter mixture types. Approximate degrees of freedom, pseudo F statistics and p-values
    were calculated by the Kenward-Roger approximation
    \cite{Kenward_Roger-1997}.}
  \label{tab:mixtures_drydown}
\centering
%% \tablesize{} %% You can specify the fontsize here, e.g., \tablesize{\footnotesize}. If commented out \small will be used.
\input{figs_tables/tab8_mixtures_drydown_anova.ltx}
\end{table}


% . The moisture content of a mixture at any given time was also dependent upon the bulk density of the mixture (anova analysis resulted in a better model if bulk density was included as a fixed effect, p=0.01135).


\begin{figure}[H]
  \centering
\includegraphics[width=16cm]{figs_tables/fig9_mixture_obs_vs_pred_mc.pdf}
\caption{Observed vs predicted fuel moisture content for litter mixtures. Predictions are based on the mean moisture content at each particualr time sinece wetting for the three constituent species in each mixture.  Codes for species in mixtures: Ab = \emph{Abies concolor}, Ca = \emph{Calocedrus decurrens}, Pi = \emph{Pinus jeffreyi} and Qu = \emph{Quercus kellogii}.}
  \label{fig:mixture_obs_pred_mc}
\end{figure}

There was evidence for non additive effects in drying rate. Mixtures tended to have lower water content than predicted from the mean of the constituent species after wetting, but higher water content than expected after long drying (Wald test shows slope of mixed linear model illustrated in Fig. \ref{fig:mixture_obs_pred_mc} less than 1, p < 0.0001).


\begin{figure}[H]
  \centering
\includegraphics[width=16cm]{figs_tables/fig10_mixture_obs_vs_pred_spread.pdf}
\caption{Observed vs predicted flame spread rate for litter mixtures. Filled circles show observed spread rate values. Open diamonds indicate predicted spread rates based on the average spread rate at each particular time since wetting for the three constituent species in each mixture. Codes for species in mixtures: Ab = \emph{Abies concolor}, Ca = \emph{Calocedrus decurrens}, Pi = \emph{Pinus jeffreyi} and Qu = \emph{Quercus kellogii}.}
  \label{fig:mixture_obs_pred_spread}
\end{figure}


% \begin{figure}[H]
%   \centering
% \includegraphics[width=16cm]{figs_tables/fig11_mixture_obs_vs_pred_consume.pdf}
% \caption{Observed vs predicted fuel consumption for litter mixtures. Predictions are based on the modeled spread rate at each particular time since wetting for the three constituent species in each mixture. Codes for species in mixtures: Ab = \emph{Abies concolor}, Ca = \emph{Calocedrus decurrens}, Pi = \emph{Pinus jeffreyi} and Qu = \emph{Quercus kellogii}.}
%   \label{fig:mixture_obs_pred_consume}
% \end{figure}


\begin{figure}[H]
  \centering
\includegraphics[width=16cm]{figs_tables/fig11_mixture_obs_vs_pred_consume.pdf}
\caption{Observed vs predicted fuel consumption for litter mixtures. Filled circles show observed fuel consumption percentages. Open diamonds indicate predicted fuel consumption based on the average fuel consumption at each particular time since wetting for the three constituent species in each mixture. Codes for species in mixtures: Ab = \emph{Abies concolor}, Ca = \emph{Calocedrus decurrens}, Pi = \emph{Pinus jeffreyi} and Qu = \emph{Quercus kellogii}.}
  \label{fig:mixture_obs_pred_consume}
\end{figure}


Flammability measures also demonstrated nonadditivity in mixtures. Flame spread rate was consistently slightly higher than predicted (Fig \ref{fig:mixture_obs_pred_spread}, Wald test of mixed linear model centered intercept for observed-predicted values greater than zero, p = 0.017) and the positive non additivity increased with time since wetting (slope greater than zero, p = 0.005). Fuel consumption, however, showed no overall bias towards positive or negative non additivity, but non addivity (difference between observed and expected values) did show a positive relationship with time since wetting (Fig. \ref{fig:mixture_obs_pred_consume}, p < 0.0001).

%%%%%%%%%%%%%%%%%%%%%%%%%%%%%%%%%%%%%%%%%%
\section{Discussion}

\subsection{Species specific moisture dynamics flammability}

The species in this study displayed different patterns of both moisture absorption and desorption, which were controlled by different litter traits (specific leaf area and litter bulk density, respectively). \emph{Q. kelloggii} was the species that absorbed the most moisture, followed distantly by the finest-needled pine, \emph{P. lambertiana}. These species that have a greater absorptive capacity are also among the most flammable in terms of spread rate when dry. One presumption of ours prior to this experiment was that the high flammability previously reported for \emph{Q. kelloggii} \cite{Magalhaes+Schwilk-2012} would disappear under higher fuel moisture contents. This prediction was partly supported: \emph{Q. kelloggii} fails to ignite until after 72 hours of drying in this study (Figs. \ref{fig:spread_time} and \ref{fig:consume_time}). However, high moisture content in this oak was unable to suppress ignition of mixtures to which it contributed (Figs. \ref{fig:mixture_obs_pred_spread} and \ref{fig:mixture_obs_pred_consume}). 

    The dry down curves we measured demonstrate that after a rain event, the litter of different species will exhibit different moisture contents over time. Our flammability trials demonstrate that that both flame spread rate and fuel consumption increased with time since wetting but that the rate of increase varied across these genera. This indicates that there is a taxon-specific response to flammability based on time since wetting.  Only the pines are able to ignite and support flame spread immediately after wetting (Fig. \ref{fig:spread_time}). \emph{Quercus} exhibited a threshold response to drying with no ignition prior to 72 hours of drying but once ignition is possible, having faster spread rates than the other genera.

    
With eight species, power to detect species-specific trait effects on these moisture patterns was weak, but we examined one litter trait known to directly influence flammability: litter bulk density and one leaf trait, specific leaf area. Litter density is largely driven by leaf size in this system  and in turns controls the major axis of flammability variation in dry litter \cite{Magalhaes+Schwilk-2012}. Our results suggest that moisture dynamics are controlled by both traits intrinsic to the leaf particles (specific leaf area) and emergent traits of the litter bed (bulk density). Specific leaf area here is probably capturing some joint effect of cuticle thickness and total surface area of litter particles.  The oak has much higher initial water retention than the other species, followed by the finest-needled pine, \emph{P. lambertiana}. With only a handful of species, however, disentangling mechanism of water absorption is not possible. Our study examines fresh litter, but we expect that decomposition will affect moisture content. For example, Van Wagner \cite{Van_Wagner-1969}, investigating fuel weathering, found that pine needles had a variation in moisture content 3 to 6 times greater when they lacked their protective wax coating when compared to intact needles.



% Therefore, mere moisture content will not inform on the fire behavior measures of a litter bed. Species composition of the litter must be considered, and time since the last rain event should allow the determination of the potential flammability of an area. 

% The pines are more flammable at higher moisture contents when compared to volatile-rich species such as \emph{C. decurrens}, \emph{Abies spp}. or \emph{S. giganteum}. This effect is influenced by litter bulk density and is likely to persist in fuel mixtures where the constituent species have different desiccation indices and therefore will have different moisture contents.


% I have determined moisture of extinction for the eight species in this study. As expected, species that have higher flammability values (Pinus spp.) had higher moisture of extinction than species with lower flammability (Abies spp. and S. giganteum). I used a higher cut-off moisture for the burn trials than the moisture of extinction found in the literature (65\% vs. 27-30\%; \citep{ Rothermel-1972} [TODO Dickinson et al. 2016] in order to be conservative, as moisture content studies as sparse. However, Q. kelloggii was still flammable at ~80-90\% moisture content (values at 72 hours used for the species mixtures), which I did not anticipate. Due to its high moisture content, Q. kelloggii was not burnt at 24 and 48 hours since wetting, which explains the gap shown in the Figures. Though I cannot comment on the exact behavior of Q. kelloggii at 24 and 48 hours, it is unlikely to be more flammable than the pines. This result indicates that though Q. kelloggii might be too wet to start a fire right after a rain event, Q. kelloggii litter dries faster than the other species in this study and will become very flammable within two days.

\subsection{Effects of moisture and flammability in litter mixtures}

We tested the dry-down process and flammability of four unique three-species litter mixtures. These mixtures contained species representative of all dry-down behavior types observed in single species trials. As expected from the single species results, mixtures that contained \emph{Q. kelloggii} absorbed more water during wetting, and the mixture without any Cupressaceae showed the most rapid dessication (Figure \ref{fig:mixture_dry_down} and Table \ref{tab:mixtures_drydown}). Drydown rates did not behave as the average of the constituent species and mixtures tended to be drier than expected early in the process and wetter than expected later (Fig. \ref{fig:mixture_obs_pred_mc}). The fact that mixtures tended to have lower water content than predicted from the mean of the constituent species after wetting, but higher water content
than expected after long drying is surprising because past work has shown that
mixtures tend to pack less densely than do litter monocultures \cite[][, Table 2]{Magalhaes+Schwilk-2012}. One would expect that the lower bulk density of mixtures would speed drying. 

% This result indicates that immediately after a rain event, the moisture content of a mixture will be lower than expected, however these high values for moisture content are unlikely to be found in natural litters. As the mixture dries, the dry-down behavior will approximate the null model, and three to four days after the rain event the dry-down behavior will become non-additive again. 


The overall ranking of flammability in litter mixtrures was consistent with the
results obtained for the individual species. Mixtures containing \emph{Q.
  kelloggii} and \emph{P. jeffreyi} had fast spread rates. These results are
likely influenced by the lower bulk densities of litter mixtures containing
those two species. Most interestingly, there were striking pattern of nan
additivity in spread rate and in fuel consumption for the four mixtures in this
study, indicating that the positive non additivity is in flammability is not
restricted to completely dry litter mixtures \citep{Magalhaes+Schwilk-2012,
  VanAltena+Logtestjin+etal-2012}. Non additivity in both spread rate fuel
consumption became more positive with time since wetting and these patterns
were consistent across mixture identities. This is a slightly different result
than that reported by Blauw et al \citeyear{Blauw+Wensink+etal-2015}. Blauw et
al found that non-addivity was greatest in moist mixtures of heathland fuels.
In their work this non addiitvity was generally negative but that positive non
addivity in mixtures was often driven by mixtures meeting ignition thresholds
not met by the constituent species alone. To limit the number of models fit, we
limited our investigation to two flammability parameters that represent two
major axes of flammability variation. Therefore, our study does not fully
capture ignition effects although we do have some results consistent with
theirs: mixtures with \emph{Q. kellogii} could ignite at times since wetting at
which the species in monoculture could not. Overall, our study does show a consistent pattern of increasingly positive non addiitvity in flammability with litter drying. This result, together with
previously published work, suggests that litter flammability is prone to
non-additive effects, reinforcing the idea that we cannot approach flammability
by lumping species into fuel categories, as is common in fire behavior models.
Community flammability is dependent on and particular to its constituent
species and cannot be addressed by simply averaging the flammability of all
species it comprises. This dynamic view is likely to be increasingly relevant
considering ongoing climate change, which predicts changes in community
assembly.

[CONCLUSIONS ?]

% I have found non-additive effects (both positive and negative) in moisture content and in the flammability of wet litter mixtures of mixed-conifer forest. These results complement the efforts of Blauw et al. (2015) for moist litter and add to the literature investigating non-additivity in litter flammability (de Magalhães and Schwilk 2012, van Altena et al. 2012). However, these studies are limited in the scope of the species studied and suggest that non-additivity needs to be investigated at a larger scale, within different systems and comparatively across a larger number of species. Moisture content is considered the single most important driver of litter flammability and the results from my study indicate how leaf and litter traits affect moisture dynamics, and how moisture effects on litter flammability are species-specific. A trait-based perspective provides a better mechanistic understanding of moisture dynamics of litter fuel flammability and fire behavior and is especially important for predicting and understanding fire in plant communities that are changing.

% %%%%%%%%%%%%%%%%%%%%%%%%%%%%%%%%%%%%%%%%%%
% \section{Conclusions}

% This section is not mandatory, but can be added to the manuscript if the discussion is unusually long or complex.

%%%%%%%%%%%%%%%%%%%%%%%%%%%%%%%%%%%%%%%%%%
% \section{Patents}
% This section is not mandatory, but may be added if there are patents resulting from the work reported in this manuscript.

%%%%%%%%%%%%%%%%%%%%%%%%%%%%%%%%%%%%%%%%%%
\vspace{6pt} 

%%%%%%%%%%%%%%%%%%%%%%%%%%%%%%%%%%%%%%%%%%
%% optional
%\supplementary{The following are available online at \linksupplementary{s1}, Figure S1: title, Table S1: title, Video S1: title.}

% Only for the journal Methods and Protocols:
% If you wish to submit a video article, please do so with any other supplementary material.
% \supplementary{The following are available at \linksupplementary{s1}, Figure S1: title, Table S1: title, Video S1: title. A supporting video article is available at doi: link.}

%%%%%%%%%%%%%%%%%%%%%%%%%%%%%%%%%%%%%%%%%%
\authorcontributions{conceptualization, RMQM and DWS; methodology,  RMQM and DWS; experiments, RMQM. coding and data analysis, DWS and RMQM; writing DWS and RMQM; writing--review and editing, DWS}

%%%%%%%%%%%%%%%%%%%%%%%%%%%%%%%%%%%%%%%%%%
\funding{This research received no external funding.}

%%%%%%%%%%%%%%%%%%%%%%%%%%%%%%%%%%%%%%%%%%
\acknowledgments{TODO.}

%%%%%%%%%%%%%%%%%%%%%%%%%%%%%%%%%%%%%%%%%%
\conflictsofinterest{The authors declare no conflict of interest. The funders had no role in the design of the study; in the collection, analyses, or interpretation of data; in the writing of the manuscript, or in the decision to publish the results.} 

\section{}
All appendix sections must be cited in the main text. In the appendixes, Figures, Tables, etc. should be labeled starting with `A', e.g., Figure A1, Figure A2, etc. 

%=====================================
% References, variant B: external bibliography
% =====================================
\externalbibliography{yes}
\bibliography{/home/schwilk/write/bib/schwilk}

%%%%%%%%%%%%%%%%%%%%%%%%%%%%%%%%%%%%%%%%%%
\end{document}
