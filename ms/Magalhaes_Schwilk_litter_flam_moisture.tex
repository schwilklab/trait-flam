\documentclass[letterpaper,12pt]{article}
\usepackage{jecol}

\title{Moisture effects on leaf litter are species-specific and result in non-additive flammability in mixed conifer forest}

\running{Leaf litter flammability}

\author{Rita Quinones de Magalhaes$^{1,2}$ and Dylan W. Schwilk$^1$}
  
\affiliations{
    \item Texas Tech University, Lubbock, Texas 79411, USA.
    \item Current address: Rochester Institute of Technology, Rochester, NY 14623, USA.
}

\nwords{7600}
\ntables{4}
\nfig{8}
\nref{39}

\corr{\url{dylan.schwilk@ttu.edu}}

\begin{document}
\maketitle

\begin{abstract}
  \noindent \begin{enumerate}
    
      \item Moisture content is a strong determinant of forest fuel
    flammability and varies on small spatial scales and over short time
    periods. Empirical modeling of equilibrium dead fine fuel moisture is well
    established, but less is known about the mechanisms by which
    species-specific litter traits influence litter moisture dynamics.
    Characterizing how species’ leaf litter retain and release moisture through
    time is critical to the assessment of fuel’s availability to burn.
      \item Natural forests are often comprised of mixed stands, leading to
    mixtures of leaf litter on the surface fuelling the fires. Multiple studies
    have established the existence of non-additive effects in the flammability
    of dry litter, but we lack an understanding of moisture synergies in mixed
    litter and of moisture effects on the flammability of mixed litter.
      \item The litter from eight species of a mixed-conifer forest was
    saturated and allowed to dry to determine moisture absorption capacity and
    drying rates. Burn trials were performed in moist litter beds of single
    species and of mixtures to establish flammability response to the dry-down
    process.
      \item Moisture dynamics varied across species with higher specific leaf
    area being associated with higher maximum absorption and lower litter bulk
    density being associated with faster drying rates. Species that produce
    more aerated litter beds (such as \emph{Q. kelloggii} and \emph{Pinus}
    species) have faster rates of litter drying.
      \item We examined two axes of flammability variation (spread rate and
    fuel consumption) and found that these differing moisture dynamics result
    in time since wetting having strongly differing effects on flammability
    across the major litter types typified by different tree taxa.
      \item We found that litter mixtures exhibited consistent non-additivity
    in flame spread rate in fuel consumption. This non-additivity became more
    positive with fuel drying. Litter mixtures burn with fire behavior more
    similar to that of the most flammable constituent species and this effect
    increases as fuels dry.
      \item \emph{Synthesis:} In this temperate mixed conifer forest, the shade
    intolerant species are favored by the increased fire to which their fuels
    contribute Such positive feedback effects increases the potential for
    species to exert community scale effects even when relatively rare and that
    can influence community assembly and the evolution of niche-constructing
    traits.
  \end{enumerate}
\end{abstract}

\noindent \textbf{Keywords:} California, flame spread rate, flammability, fuel consumption, fuel moisture, leaf litter, non-additivity

\newpage

\section*{Introduction}

Fuel moisture content is an important, and often driving, determinant of
surface fire behavior. Environmental conditions such as temperature, relative
humidity, and wind drive the moisture content of the fuel available to burn
\citep{Kreye+Varner+etal-2018}. In particular, fine dead fuels respond very
readily to environmental changes \citep{Nelson-2001}. Fuel moisture influences
flammability and fire behavior through effects on ignition, spread rate, flame
height, fuel consumption, and heat release, due to the high specific heat
content of water \citep{Rothermel-1972, Nelson-2001}. Moisture acts as a
heat-sink resulting in less energy available for propagation, and the resulting
water vapor will both cool the flaming front and dilute flammable gases
\citep{Albini-1976, Shafizadeh-1977}. Variation in moisture of living fuels has
strong effects on flammability, but moisture of dead excised litter fuels can
also vary. Moisture of litter fuels varies with climate, with seasons, and with
short term weather within a season. Therefore, the rate of drying and
flammability at any particular moisture content are key parameters that
influence fuel flammability throughout the fire season \citep{Nelson-2001}.

Fire scientists often categorize fuel by size classes that correspond to
moisture exchange rates, although such simplified categories have long been
recognized as not sufficient to describe species-specific effects on moisture
dynamics \citep{Anderson-1985}. Fuel moisture dynamics of leaf litter are
dependent upon forest structure (shading and rain throughfall effects), litter
physical structure, and chemical composition of the fuel-beds
\citep{Nelson+Hiers-2008, Matthews-2014, Kreye_Hiers_etal-2018}. A variety of
models describing moisture dynamics of fine fuels are used by fire scientists
\citep{Viney-1991, Nelson-2000, Catchpole+Catchpole+etal-2001}. These models
usually aim to describe equilibrium conditions and at least include temperature
and relative humidity although some include precipitation and solar radiation
effects as well. Explicit models of fire behavior as a function of time since
wetting could improve fire prediction, however. In many litter fuel driven
systems, natural lightning ignitions are concurrent with precipitation and,
therefore, fire behavior can depend on rate of drying following a wetting
event. If species differ in their drying rate, this may strongly influence
relatively flammability across species at any particular time following
wetting. Such differential moisture content across time, may contribute to the
often high spatial heterogeneity of surface fires
\citep{Knapp_Schwilk_etal-2006, Kreye_Hiers_etal-2018}.

Litter packing, measured as litter-bed density or packing ratio, controls the
flammability of dry litter fuels in seasonally dry climates because higher
litter density reduces flammability as a consequence of oxygen limitation. Leaf
litter density is driven by leaf size and shape
\citep{Fonda+Belanger+etal-1998, Scarff+Westoby-2006, Kane+Varner+etal-2008}.
These litter particle effects on litter bed aeration, in turn, have effects on
litter bed flammability \citep{Ganteaume+Marielle+etal-2011,
  Schwilk+Caprio-2011, Kreye+Varner+etal-2013}. These species-specific effects
are well established. However, how moisture dynamics modulate species specific
flammability is poorly known and could be driven by by leaf traits, emergent
litter litter traits, or both. Species specific effects on litter flammability
are recorded for the forests of the southeastern US \citep{Nowacki+Abrams-2008}
where species-specific litter drying rates play a role in flammability
feedbacks \citep{Kreye+Varner+etal-2013}. Similarly, in mixed-conifer forest of
the western USA, where the fire regime is characterized by surface fires fueled
by plant litter, overstory tree composition drives variation in local fire
behavior \citep{Schwilk+Caprio-2011} as a result of differential litter
flammability across tree species and as a result of synergistic interactions in
multi-species litter mixtures \citep{Magalhaes+Schwilk-2012}. These
flammability feedbacks may help explain historical fire frequency patterns, but
current estimates of litter flammability in mixed conifer forests are based
largely on burning dry litter.

Fuel moisture content through time is controlled by two main processes,
absorptive capacity and rate of moisture loss \citep{Kreye+Kobziar+etal-2013}.
Absorptive capacity represents the potential maximum fuel moisture and is
evaluated by the moisture content of litter when saturated. After saturation,
the differences in moisture content indicate different abilities to retain
moisture. Litter particle geometry may have direct effects on maximum moisture
absorption through area-mass relationships and we also expect effects of leaf
morphology, especially leaf cuticle thickness \citep{Van_Wagner-1969}. Drying
rates should also be influenced by leaf and litter particle traits:
surface-area-to-volume ratio and litter bulk density will influence air flow
through the litter bed. Litter packing therefore has two pathways of effect on
fire behavior: directly by its effect on oxygen availability for combustion,
and indirectly by its effect on dry-down rate, and both these pathways are
affected by leaf litter traits, especially leaf size
\citep{Scarff+Westoby-2006}. Fuel-beds composed of larger leaves with lower
moisture retainability and should favor drying, thus increasing the probability
of ignition and fire spread.

Flammability is a multi-dimensional characteristic of fuels
\citep{Schwilk-2015, Pausas+Keeley+etal-2017} with different traits influencing
different components of fire behavior. In litter driven fire, the two major
axes of variation are described by measurements related to flame spread rate
and measurements related to heat transfer. Flame spread rate is easy to measure
in burning trials, but heat transfer can be more difficult and often proxies
such as duration of heating are used \citep{Magalhaes+Schwilk-2012,
  Varner+Kane+etal-2015}. These different components of fire behavior have
differing ecological effects. Species traits may interact such that the
flammability of multi-species mixtures is not the average of the constituent
species. Such non-additive effects are synergistic in dry litter fuels where
the flammability of a mixture is driven by the most flammable species
\citep{VanAltena+Logtestjin+etal-2012, Magalhaes+Schwilk-2012}. It is possible
that similar non-additive effects may influence litter moisture content,
especially if different leaf traits influence absorptive capacity and drying
rate. There is some evidence for non-additive flammability in moist fuels with
one study finding increasing non-additivity with moisture
\citep{Blauw+Wensink+etal-2015}.

We investigated species specific litter moisture dynamics across eight tree
species in a temperate fire-prone forest in California. We determined the
maximum water retention and rate of moisture loss and examined how these vary
in relation to two leaf and litter bed traits. We then examined the influence
of moisture on the flammability across these species. Experimental burn trials
of leaf litter at various moisture levels determined how time since wetting
influenced two axes of flammability characterized by flame spread rate and
total fuel consumed. Previous work described the flammability of these species
in monocultures and in mixtures for oven-dry litter
\citep{Magalhaes+Schwilk-2012} and found consistent positive non-additivity of
flammability where fire behavior of mixtures was driven by the most flammable
species. Here, we examined if this positive non-additivity was maintained in
mixtures across varying levels of fuel moisture. The burn trials were also
performed with multi-species litter mixtures to evaluate if species contribute
equally to litter moisture dynamics or if there are non-additive effects as
observed in oven-dried litter. We tested if the non-additive effects on
reported in previous studies occur under ecologically-relevant moisture levels
and whether fuel moisture effects strengthen or moderate such non-additivity.


%%%%%%%%%%%%%%%%%%%%%%%%%%%%%%%%%%%%%%%%%%
\section*{Materials and Methods}

\subsection*{Site description and species selection}

Field sites for this study were located in Sequoia and Kings Canyon National
Parks, California, USA (36 36N, 118 42 W) between 1600 and 2400 m elevation in
mixed-conifer forest. We chose eight dominant tree species that are
representative of this type of forest: \emph{Pinus jeffreyi} Grev. \& Balf.,
\emph{Pinus lambertiana} Dougl., \emph{Pinus ponderosa} Dougl. ex Laws.,
\emph{Abies concolor} (Gord. \& Glend.) Lindl. ex Hildebr., \emph{Abies
  magnifica} A. Murr., \emph{Calocedrus decurrens} (Torr.) Florin,
\emph{Quercus kelloggii} Newb., and \emph{Sequoiadendron giganteum} (Lindl.) J.
Buchholz. In this study, we consider litter the top layer of leaves and small
twigs less than 0.625 cm diameter (= 1-hour fuel) that have fallen in the
previous year (mostly undecomposed). Leaf litter was collected in the summer
(mid-June to mid-July) in 2012 from 21 separate sites across the study area. We
collected from a minimum of four populations per species with each population
9--32\,km distant from its neighbors. An exception to this was \emph{A.
  magnifica} which is restricted to the higher elevations and could only be
collected from one area due to designated wilderness boundaries and logistical
constraints. Within that area, \emph{A. magnifica} litter was collected from
over 10 different individuals. For the remaining species, the collection
involved 2--4 individual trees at each site which were separated from one
another by at least 10\,m. The litter was collected 2\,m away from the trunk to
obtain a more uniform sample of leaf and fine branchlet litter, because bark
and heavier branches tend to fall closer to the tree trunk. All litter was air
dried to $< 5$\% fuel moisture by dry weight for weighing and dividing into
moisture trials. The moisture manipulations and subsequent burning trials were
conducted in April--August 2015.

\subsection*{Moisture absorption and desorption}


We produced dry-down curves to determine maximum water retainability and the
rate of moisture content loss for each species and for four distinct
three-species mixtures. These curves allowed us to determine the rate at which
a species loses moisture, approximated as exponential decay, and to determine
maximum water content following saturation and gravity draining. Samples from
the eight species (six replicates per species) and four mixtures (5 replicates
per mixture type) were placed in baskets approximately 45 x 45\,cm by 15\,cm
tall constructed of 1\,cm aluminum screen. Samples were 450\,g dry weight
litter per trial and litter depth was standardized to 10\,cm by changing the
horizontal dimensions of the basket. Weight was measured by using a balance
sensitive to 0.1\,g (model XS16001L, Mettler Toledo, Columbus, OH). Samples
were immersed in plastic storage bins filled with water for 24 hours to fully
saturate them. After saturation, the baskets were removed from the water,
allowed to drain for 3 minutes, and samples weighed to measure maximum moisture
content. The samples were then allowed to dry in a controlled environment kept
at 21\,C temperature and 30\% relative humidity, assessed via an iButton sensor
(Maxim Integrated) placed over the samples. Litter baskets were arranged in a
completely randomized design. During the dry-down process three subsamples of
5--10\,g were taken at 24 hour intervals and weighed. These subsamples were
then oven dried for 24 hours at 100\,C and re-weighed to assess moisture
content. This was repeated until the final sample from a basket was less than
10\% moisture.

With only eight species in this study, our power to test leaf trait effects on
the observed moisture dynamics was low, so we decided \emph{a priori} to test
the effect of one leaf trait likely to influence moisture absorption, specific
leaf area, and one litter trait known to directly effect flammability and
likely to influence drying rate, litter bulk density. Specific leaf area was
measured on freshly fallen litter by sampling 5 individuals per species. This
measurement included petiole and for the scale-leaved species included the full
excised branchlet. Therefore, this is not directly comparable to typical specific leaf
area measurements on fresh leaves. Litter bulk density was obtained in this
experiment for each basket by dividing the dry mass (450\,g) by the estimated
litter volume (basket width x length x litter depth).

\subsection*{Flammability assessment}

We performed flammability tests across multiple moisture levels on monospecific
litter beds and on mixed litter beds with each species or mixture replicated 5
times at each moisture level. We used the initial dry-down experiment to guide
our timing so as to attempt to burn samples over a range of moisture contents.
We aimed for moisture contents from 10\% to 100\% on a dry mass basis as
preliminary trials on damp fuels suggested that across these fuels ignition
will not occur above 100\% fuel moisture. To expand our data on the dry end of
this gradient, we supplemented these data with data from an earlier experiment
on oven-dried litter \citep{Magalhaes+Schwilk-2012}. This supplemental data
added five replicates per species at an oven dried moisture content of $<$ 5\%.
We assigned these replicates a time since wetting value of 144 hours equivalent
to the longest time since drying for the wetted samples. Flammability was
assessed using a 150\,cm long burn table, 15\,cm wide and 15\,cm tall, in which
leaf litter was placed and the table gently shaken to allow settling. Burning
trials all used 450\,g (dry weight) of litter. For three-species mixtures, this
was 150\,g per species. This design allows fire to reach constant flame spread
rates \citep{Magalhaes+Schwilk-2012}. Two graduated metal rulers equally spaced
along the apparatus allowed visual determination of maximum flame height.
Litter was ignited with a propane torch; maximum adiabatic flame temperature of
1899 C, \url{http://www.benzomatic.com/products/fuel.aspx}) at one end of the
apparatus and allowed to burn until extinction. If a sample failed to ignite
after 30 seconds, no further attempt was made and if three samples of a species
at a particular moisture level failed to ignite, no further attempts were made
on that species/moisture combination and the replicate was allowed to continue
drying. A timer was used to record total time to ignition and duration of
flaming combustion. The flammability trials followed the methodology detailed
in de Magalhães and Schwilk \citep{Magalhaes+Schwilk-2012}.
 
During flammability trials, we recorded time to ignition (s), duration of
flaming combustion (s), flame spread rate (mm/s), calculated by dividing the
length of the burned surface by the time it takes the flaming front to reach
the end of the apparatus, maximal flame height (mm), and fuel consumption (\%).
Some trials resulted only in smoldering combustion; for those trials, a time to
ignition and mass loss were recorded, but flame spread rate and flame height
were recorded as zero. Our thermocouple system failed or data was lost for two
days of burning trials, therefore we have omitted those data from this
analysis. Our past work \citep{Magalhaes+Schwilk-2012} demonstrated that flame
spread rate captures one important axis of flammability variation
\citep{Schwilk-2015, Pausas+Keeley+etal-2017} and percent fuel consumed
captures a portion of the second major axis of flammability variation (although
not as well as integrating temperature over time does). Our analysis,
therefore, focused on these two relatively orthogonal measures of flammability
to avoid examining multiple redundant dependent variables.

Burn trials occurred in a cement structure used to simulate house fires at the
Fire Department of the City of Lubbock. This structure minimized wind and
helped regulate temperature and relative humidity. Temperature and relative
humidity were measured before every trial using a Kestrel 3000
(Nielsen-Kellerman, Boothwyn, PA). All trials were conducted on clear days
between 10 am and 3 pm.

\subsection*{Statistical analyses}

Dry-down curves were created based on the drying experiment, fitted with an
exponential decay curve to each species.:

\begin{equation}
m = m_{max} e^{-\lambda t}
\end{equation}

Where $m$ is moisture content at time $t$, and $\lambda$ is the exponential
decay coefficient. To estimate these coefficients we fit a linear mixed-effects
model by first taking the natural logarithm of moisture content, with time
since wetting and species as fixed effects nested within replicate basket. We
extracted the coefficients of the fitted model to describe maximum water
retainability ($m_{max}$; g of water per g of dry mass) and desiccation rate
($\lambda$; hour$^{-1}$) for each species. All analyses were conducted in R 3.6.1
\citep{RCoreTeam-2019}. We fit models with R packages `lme4'
\citep{Bates_Machler_etal-2015} and `afex' \citep{Singmann_Bolker_etal-2017}.
Approximate degrees of freedom and p-values were calculated by the
Kenward-Roger approximation \citep{Kenward_Roger-1997} which is recommended by
Luke \citeyear{Luke-2017} as producing acceptable Type I error rates. We tested if
one leaf trait (specific leaf area) and one litter trait (litter bulk density)
influenced moisture dynamics by fitting linear models with species means as
observations. For each response variable we fit a single linear model with two
predictors and an interaction term.

Our past work \citep{Magalhaes+Schwilk-2012} demonstrated that dry litter
flammability across these eight species was driven by leaf traits and litter
packing. These previously reported flammability trials demonstrated that
species within a genus tended to behave similarly and the two scale-leaved
genera in the Cupressaceae, \emph{Calocedrus} and \emph{Sequoiadendron}, also
behaved similarly to one another. Therefore, to preserve power we used taxonomic
groups (\emph{Abies}, Cupressaceae, \emph{Quercus}, and \emph{Pinus}) as our
flammability functional groups and treated these as fixed effects in our
models. The effect of fuel moisture and time since wetting on flammability
parameters were modeled using linear models with fuel moisture or time since
wetting and taxon as fixed effects and average vapor pressure deficit in the
burning room during that trial as a nuisance covariate. We used linear mixed
models to predict spread rate, but fuel consumption tends to exhibit threshold
effects with few intermediate values. Therefore, we used mixed generalized
linear models with a beta regression link with the `glmmADMB' package
\citep{Skaug_Fournier_etal-2016} to model percent fuel consumed as a function of
moisture content and of time since wetting.

We investigated possible non-additive effects on moisture and on flammability
by testing if moisture content and flammability of mixtures were predicted by
the average of the constituent species. To choose mixtures, we first grouped
the species according to their maximum water retention and drying rate
according to estimated marginal means by species using the R package `emmeans'
\citep{Lenth-2019}. This resulted in four groups of species and we then
selected one representative species from each litter type group and produced
all four possible three-species litter mixtures using those four species. For
each mixture and time since wetting, we predicted moisture content based on the
mean content of the three constituent species at that time since wetting. We
predicted flammability parameters based on the average flammability measures
from individual species trials at each time since wetting and then calculated a
predicted spread rate and fuel consumption as the average of these three values
for the mixture. The expectation from the null model was that the difference
between observed and predicted values of flammability will be zero. We tested
if there were significant departures from the null with a mixed effects linear
model. For flammability parameters, we fit linear mixed effect models with the
observed-predicted flammability value as the dependent variable and used Wald
tests to test if the centered intercept was different than zero (a measure of
overall non-additivity) and if the slope with time since wetting was different
from zero (indicating non-additivity changing with drying).

%%%%%%%%%%%%%%%%%%%%%%%%%%%%%%%%%%%%%%%%%%
\section*{Results}

\subsection*{Species specific moisture dynamics and leaf traits}

There were marked differences in the extent of moisture species retain
initially following wetting and draining by gravity (all $P < 0.0001$, Fig.
\ref{fig:drydown} and Table \ref{tab:drydown}). \emph{Q. kelloggii}, the only
broadleaf, is the species that retained the most moisture initially.
\emph{Quercus}, like the pines, had a rapid drying rate (Fig. \ref{fig:drydown}
and SI Fig. 1). Calculating estimated marginal means following the linear
mixed-effects model discriminated the species into three different maximum
water retention groups: 1) \emph{Q. kelloggii}; 2) \emph{P. lambertiana} and
\emph{C. decurrens}; and 3) \emph{A. magnifica}, \emph{S. giganteum}, \emph{A.
  concolor}, \emph{P. jeffreyi} and \emph{P. ponderosa} (SI Fig. 1). There are
also differences in drying rates across the eight species and estimated
mixed linear model slopes discriminated species into three groups from lowest
to most rapid: 1) \emph{C. decurrens}, 2) \emph{Abies spp} and \emph{S.
  giganteum}, and 3) \emph{Q. kelloggii} and \emph{Pinus spp} (SI Fig. 2).

\begin{figure}[h]
  \centering
\includegraphics[width=8cm]{figs_tables/fig1_drydown-curves.pdf}
\caption{Dry down curves for eight litter types. Lines show best-fit
  exponential decay curves fit to moisture content on a dry mass basis. Litter
  samples we saturated by immersion for 24 hours and then allowed to drain by
  gravity for 3 min before initial weighing. Colors indicate genera and there
  were significant differences in both maximum water retention and in rate of
  drying (based linear model on log of water content). A small amount of
  horizontal jitter (0.5 hr) was added to this plot to aid in visualizing
  because, although collections were staggered, there were sometimes multiple
  samples collected at a single time point.}
 \label{fig:drydown}
\end{figure}


\begin{table}[h]
  \caption{Linear mixed model results for moisture content as a function of
    time since wetting and species. Approximate degrees of freedom (df), pseudo F
    statistics and p-values were calculated by the Kenward-Roger approximation.
    \citep{Kenward_Roger-1997}. Model coefficients reported in SI Table 1.}
  \label{tab:drydown}
\centering
%% \tablesize{} %% You can specify the fontsize here, e.g., \tablesize{\footnotesize}. If commented out \small will be used.
\begin{tabular}{llll}
  \toprule
Effect & df & $F$ & $P$ \\ 
  \midrule
  hour & 1, 183.73 & 1599.59 & $<$.0001 \\ 
  species & 7, 89.73 & 36.15 & $<$.0001 \\ 
  hour:species & 7, 184.15 & 23.06 & $<$.0001 \\ 
   \bottomrule
\end{tabular}
\end{table}

Across these eight species specific leaf area had a positive effect on maximum
water retention (SI Fig 4, linear model $P$ = 0.003) and litter
bulk density had a negative effect on desiccation rate (SI Fig 5,
linear model $P$ = 0.029). ANOVA results are shown in Tables
\ref{tab:mc_di_anova}.

\begin{table}[h]
  \caption{Linear model results for maximum water retention (\% moisture
    content on dry mass basis) and for estimated drying rate (hr$^{-1}$) as
    functions of specific leaf area and litter bulk density across eight
    species. ANOVA tables shown with degrees of freedom, sum of squares, mean
    squares, $F$, and $P$ for each effect. Model coefficients in SI tables 2 and 3}
  \label{tab:mc_di_anova}
\centering
%% \tablesize{} %% You can specify the fontsize here, e.g., \tablesize{\footnotesize}. If commented out \small will be used.

\begin{tabular}{ccrrrrr}
  \toprule
Ind. variable & Effect & df & Sum of Squares & Mean squares & $F$ & $P$ \\ 
  \midrule
Moisture content & SLA & 1 & 250834.5 & 250834.51 & 130.13 & 0.000 \\ 
  & density & 1 & 538.5 & 538.54 & 0.28 & 0.625 \\ 
  & SLA:density & 1 & 5135.2 & 5135.24 & 2.66 & 0.178 \\ 
  & Residuals & 4 & 7710.4 & 1927.61 &  &  \\ 
  \midrule
  
Drying rate & density & 1 & $3.92 \times 10^{-4}$ & $3.92 \times 10^{-4}$ & 11.01 & 0.029 \\ 
 &  SLA & 1 & $1.88 \times 10^{-5}$ & $1.88 \times 10^{-5}$ & 0.53 & 0.507 \\ 
 &  density:SLA & 1 & $6.89 \times 10^{-6}$ & $6.89 \times 10^{-6}$ & 0.19 & 0.683 \\ 
 &  Residuals & 4 & $1.42 \times 10^{-4}$ & $3.56 \times 10^{-5}$ &  &  \\ 
   \bottomrule

\end{tabular}
\end{table}


\subsection*{Moisture effects on flammability}


Flame spread rate declined with fuel moisture (Fig. \ref{fig:spread_moist}).
\emph{Quercus} had higher spread rates across all fuel moistures and had a
steeper decline in response to moisture than did the other groups (taxon by
time since wetting interaction $P$ = 0.022, Table \ref{tab:spreadrate_anova}).
Fuel moisture negatively influenced fuel consumption. Overall fuel consumption
and the rate of change with moisture content varied across the taxonomic groups
(Fig. \ref{fig:consume_moist}, Table \ref{tab:consume_anova}). Although
the overall ranking of groups was similar for fuel consumption and for flame
spread rate, in contrast to the pattern with spread rate, for fuel consumption,
\emph{Pinus} and \emph{Abies} has a steeper decline in flammability with fuel
moisture than did the other groups.

%% spread rate by moisture
\begin{figure}[h]
  \centering
\includegraphics[width=8cm]{figs_tables/fig2_spread_actualMC.pdf}
\caption{Flame spread rate by moisture content (model fit $F$= 61.15, df = 9,
  148, $P$ < 0.0001). Model ANOVA table in Table \ref{tab:spreadrate_anova} and
  coefficients in SI Table 4.}
  \label{fig:spread_moist}
\end{figure}

%% Table 3 
\begin{table}[h]
  \caption{Generalized linear mixed model results for two models predicting
    flame spread rate: the first model examined spread rate as a function of
    fuel moisture (mc) with taxonomic group as a fixed effect and species as a
    nested random effect. The second model predicted spread rate as a function
    of time since wetting (time) with taxonomic group as a fixed effect and
    species as a nested random effect. Approximate degrees of freedom (df),
    pseudo F statistics and p-values were calculated by the Kenward-Roger
    approximation \citep{Kenward_Roger-1997}. Estimated coefficients are in SI
    Tables 4 and 6}
  \label{tab:spreadrate_anova}
  
\centering
\begin{tabular}{ccrrrrr}
  \toprule
Model set & Effect & df & $F$ & & $P$ \\
  \midrule
  Effect of moisture & mc & 1, 146.20 & 21.04 & $<$.0001 \\ 
 & taxon & 3, 51.58 & 65.08 & $<$.0001 \\ 
 & vpd & 1, 148.79 & 9.62 & .002 \\ 
 & mc:taxon & 3, 147.21 & 4.83 & .003 \\ 
   
  \midrule
  
  Effect of time & hour & 1, 214.50 & 66.22 & $<$.0001 \\ 
 &  taxon & 3, 7.59 & 61.81 & $<$.0001 \\ 
 & vpd & 1, 215.00 & 13.49 & .0003 \\ 
 & hour:taxon & 3, 183.97 & 3.99 & .009 \\ 

   \bottomrule

\end{tabular}
\end{table}

%% fuel consumption by moisture:
\begin{figure}[h]
  \centering
\includegraphics[width=8cm]{figs_tables/fig3_consume_actualMC.pdf}
\caption{Percent fuel consumed by moisture content. Lines indicate best fit
  beta regression lines. Model deviance table in Table \ref{tab:consume_anova}
  and coefficients are in SI Table 5.}
  \label{fig:consume_moist}
\end{figure}


\begin{table}[h]
  \caption{Generalized linear mixed model results for two sets of models
    predicting fuel consumption. The first set modeled fuel consumption as a
    function of fuel moisture (mc) with taxonomic group as a fixed effect and
    species as a nested random effect. The second set modeled fuel consumption
    as a function of as a function of time since wetting (time) with taxonomic
    group as a fixed effect and species as a nested random effect. The analysis
    of deviance table below shows sequential chi-square tests on nested beta
    regression models: from the simplest null model within each group with only
    intercept and random effect, for a taxon effect only model, for a taxon and
    moisture content effect, and for the full model including the taxon by
    moisture or time interaction. Coefficient estimates are in SI Tables 5 and
    7.}
  \label{tab:consume_anova}
  \centering

  \begin{tabular}{llrrrr}
\toprule
Model set & model & Log Lik. & df & Deviance & $P$ \\ 
\midrule
  Effect of moisture &   \~{} 1 & 262.12 &  &  &  \\ 
   &  \~{} taxon & 268.73 & 3 & 13.23 & 0.0042 \\ 
   & \~{} taxon + mc & 289.57 & 1 & 41.68 & $<$ 0.0001 \\ 
   & \~{} taxon + mc + taxon:mc & 293.87 & 3 & 8.60 & 0.0351 \\ 
\midrule
Effect of time &  consum \~{} 1 & 140.08 &  &  &  \\ 
 & \~{} taxon & 155.84 & 4.00 & 31.51 & $<$ 0.001 \\ 
 & \~{} taxon + time & 209.84 & 1.00 & 108.01 & $<$ 0.001 \\ 
 & \~{} hour + taxon + taxon:time & 216.96 & 4.00 & 14.23 & 0.01 \\ 
    \bottomrule
    \end{tabular}
\end{table}


As a consequence of the relationships shown above, time since wetting had a
significant effect on spread rate ($P < 0.0001$) and the slope of this effect
varied by taxonomic group (Fig. \ref{fig:spread_time}, Table
\ref{tab:spreadrate_anova}). The pines were able to ignite and burn much sooner
after wetting than did the other taxa, none of which had measurable spread
rates until burning was attempted 72 hours after wetting (note that trials that
failed to even ignite smoldering combustion are omitted completely from the
model and figure). At that point, the \emph{Abies}, \emph{Calocedrus}, and
\emph{Sequoiadendron} had dryer litter than did the \emph{Quercus}, yet the
spread rates were slow. \emph{Quercus} had the steepest response to time since
wetting (Fig. \ref{fig:spread_time}, SI Table 6)).


\begin{figure}[h]
  \centering
\includegraphics[width=8cm]{figs_tables/fig4_spread_time.pdf}
\caption{Flame spread rate by time since wetting for litter of eight species.
  Lines indicate best fit linear model fits. Model ANOVA table in Table
  \ref{tab:spreadrate_anova} and coefficients in SI Table 6. Horizontal jitter
  (up to 2 hr) was added to the plot for visualization and to avoid
  overplotting. All collections were at the times labeled on the x-axis. A lack
  of data at earlier time points for a taxon indicates no ignitions occurred at
  those time points.}
\label{fig:spread_time}
\end{figure}

Fuel consumption was highly variable across trials and intermediate values were
less common than were very low or moderately high consumption (Fig.
\ref{fig:consume_time}). All parameters significantly contributed to the
generalized linear mixed beta regression model as indicated by the nested model
comparisons in Table \ref{tab:consume_anova}. As with flame spread rate
some pines had the most rapid response to drying. However, unlike in the other
flammability analyses, here the pines did not all behave similarly: \emph{P.
  lambertiana} had lower fuel consumption than did \emph{P. ponderosa} and
\emph{P. jeffreyi}. We noticed this during trials and in model fitting we
found that placing \emph{P. lambertiana} in its own group allowed model
convergence that was otherwise impossible. This was at odds with our intent to
use only \emph{a priori} taxonomic groups, but we could not obtain model
convergence otherwise. This result should therefore be seen as preliminary
because the test does not reflect an \emph{a priori} hypothesis.


\begin{figure}[h]
  \centering
\includegraphics[width=8cm]{figs_tables/fig5_consume_time.pdf}
\caption{Percent fuel consumed by time since wetting for litter of eight
  species. Curves shown indicate best fit beta regression lines. Model ANOVA
  table in Table \ref{tab:consume_anova} and coefficients in SI Table 7.
  Horizontal jitter (up to 2 hr) was added to the plot for visualization and to
  avoid overplotting. All collections were at the times labeled on the x-axis..
  A lack of data at earlier time points for a taxon indicates no ignitions
  occurred at those time points.}
  \label{fig:consume_time}
\end{figure}


\subsection*{Non-additivity in  litter mixtures}

Based on visual inspection of the dry down curves and the estimated marginal
means for maximum water retainability and desiccation rate (SI Figs 2--3), we
selected four species to represent the full range of litter moisture dynamics
for use in the litter mixtures and examined all four three-species combinations
of those four. The four species were \emph{Abies concolor}, \emph{Calocedrus
  decurrens}, \emph{Pinus jeffreyi} and \emph{Quercus kelloggii}.

Mixture dry down rates and maximum water retention showed some relationship to
the dry down curves for individual constituent species (see SI Fig. 6). For
example, the mixture that does not contain \emph{Q. kelloggii} (AbCaPi in SI
Fig. 6) has the lowest initial retention. Initial water retention and rates
differed across mixtures (linear model based on log of moisture content, SI Tables 8 and 9).


\begin{figure}[h]
  \centering
\includegraphics[width=16cm]{figs_tables/fig6_mixture_obs_vs_pred_mc.pdf}
\caption{Observed vs predicted fuel moisture content for litter mixtures.
  Predictions are based on the mean moisture content at each time since wetting
  for the three constituent species in each mixture. Codes for species in
  mixtures: Ab = \emph{Abies concolor}, Ca = \emph{Calocedrus decurrens}, Pi =
  \emph{Pinus jeffreyi} and Qu = \emph{Quercus kellogii}.}
  \label{fig:mixture_obs_pred_mc}
\end{figure}


There was evidence for non-additive effects in drying rate. Mixtures tended to
have lower water content than predicted from the mean of the constituent
species after wetting, but had higher water content than expected after long
drying (Wald test shows slope of mixed linear model illustrated in Fig.
\ref{fig:mixture_obs_pred_mc} less than 1, $P$ < 0.0001).


\begin{figure}[h]
  \centering
\includegraphics[width=16cm]{figs_tables/fig7_mixture_obs_vs_pred_spread.pdf}
\caption{Observed vs. predicted flame spread rate for litter mixtures. Filled
  circles show observed spread rate values. Open diamonds indicate predicted
  spread rates based on the average spread rate at each time since wetting for
  the three constituent species in each mixture. Codes for species in mixtures:
  Ab = \emph{Abies concolor}, Ca = \emph{Calocedrus decurrens}, Pi =
  \emph{Pinus jeffreyi} and Qu = \emph{Quercus kellogii}.}
  \label{fig:mixture_obs_pred_spread}
\end{figure}


\begin{figure}[h]
  \centering
\includegraphics[width=16cm]{figs_tables/fig8_mixture_obs_vs_pred_consume.pdf}
\caption{Observed vs predicted fuel consumption for litter mixtures. Filled
  circles show observed fuel consumption percentages. Open diamonds indicate
  predicted fuel consumption based on the average fuel consumption at each
  particular time since wetting for the three constituent species in each
  mixture. Codes for species in mixtures: Ab = \emph{Abies concolor}, Ca =
  \emph{Calocedrus decurrens}, Pi = \emph{Pinus jeffreyi} and Qu =
  \emph{Quercus kellogii}.}
  \label{fig:mixture_obs_pred_consume}
\end{figure}


Flammability measures also demonstrated non-additivity in mixtures. Flame
spread rate was consistently higher than predicted (Fig
\ref{fig:mixture_obs_pred_spread}, Wald test of mixed linear model centered
intercept for observed-predicted values greater than zero, $P$ = 0.017) and the
positive non-additivity increased with time since wetting (slope greater than
zero, $P$ = 0.005). Fuel consumption, however, showed no overall bias towards
positive or negative non-additivity, but non-additivity increased with time
since wetting (Fig. \ref{fig:mixture_obs_pred_consume}, slope not equal to
zero, $P < 0.0001$).

%%%%%%%%%%%%%%%%%%%%%%%%%%%%%%%%%%%%%%%%%%
\section*{Discussion}

\subsection*{Species-specific moisture dynamics flammability}

The species in this study displayed different patterns of both moisture
absorption and desorption, apparently controlled by different litter traits. In
this sample of species, those that have a greater absorptive capacity are also
among the most flammable in terms of spread rate when dry. This highlights the
importance of fuel moisture in modulating flammability. One shortcoming of past
work in this system \citep{Magalhaes+Schwilk-2012} was that flammability trials
tested only very dry fuels. We therefore expected that rankings of species
flammability might change under more realistic moisture levels. For example,
one presumption of ours prior to this experiment was that the high flammability
previously reported for \emph{Q. kelloggii} \citep{Magalhaes+Schwilk-2012} would
disappear under higher fuel moisture contents. This prediction was partly
supported: \emph{Q. kelloggii} fails to ignite until after 72 hours of drying
in this study (Figs. \ref{fig:spread_time} and \ref{fig:consume_time}).
However, high moisture content in this oak was unable to suppress ignition of
mixtures to which it contributed (Figs. \ref{fig:mixture_obs_pred_spread} and
\ref{fig:mixture_obs_pred_consume}).

The dry down curves we measured demonstrate that after a rain event, the litter
of different species will exhibit different moisture contents over time as a
consequence of both differential initial absorption and of differences in
drying rate. This has consequences for flammability as a function of species
and time since wetting. Our flammability trials demonstrate that that both
flame spread rate and fuel consumption increased with time since wetting but
that the rate of increase varied across these taxa. Only the pines are able to
ignite and support flame spread immediately after wetting (Fig.
\ref{fig:spread_time}). In contrast, \emph{Quercus} exhibited a threshold
response to drying with no ignition prior to 72 hours of drying but once
ignition is possible, having faster spread rates than the other taxa. The
Cupressaceae, which have dense litter but low maximum water absorption had low
flammability across moisture levels.

With eight species, power to detect species-specific trait effects on these
moisture patterns was weak, but we examined one litter trait known to directly
influence flammability: litter bulk density and one leaf trait, specific leaf
area. Litter density is largely driven by leaf size in this system and, in
turn, controls the major axis of flammability variation in dry litter
\citep{Magalhaes+Schwilk-2012}. Our results suggest that moisture dynamics are
controlled by both traits intrinsic to the leaf particles (specific leaf area)
and an emergent property of the litter bed (bulk density). Specific leaf area
here is probably capturing some joint effect of cuticle thickness and total
surface area of litter particles, but we cannot test the morphological and
physical mechanisms of water absorption with these data. The oak has much
higher initial water retention than do the other species, followed by the
finest-needled pine, \emph{P. lambertiana}. Our study examines fresh litter,
but we expect that decomposition will affect moisture dynamics. For example,
Van Wagner \citep{Van_Wagner-1969}, investigating fuel weathering, found that
pine needles had a variation in moisture content 3 to 6 times greater when they
lacked their protective wax coating when compared to intact needles.

\subsection*{Effects of moisture and flammability in litter mixtures}

We measured drying rates and flammability of four unique three-species litter
mixtures. These mixtures contained species representative of all dry-down
behavior types observed in single species trials. As expected from the single
species results, mixtures that contained \emph{Q. kelloggii} absorbed more
water during wetting, and the one mixture without any Cupressaceae had the most
rapid drying rate (SI Figure 4 and Si Table 8). Drydown rates, however, did not
behave as the average of the constituent species: mixtures tended to be drier
than expected early in the process and wetter than expected later (Fig.
\ref{fig:mixture_obs_pred_mc}). The fact that mixtures tended to have lower
water content than predicted from the mean of the constituent species after
wetting, but higher water content than expected after long drying is surprising
because past work has shown that mixtures tend to pack less densely than do
litter monocultures \citep[][, Table 2]{Magalhaes+Schwilk-2012}. If packing
density controls rate of drying, then we would expect mixtures to dry more
rapidly than predicted from monoculture trials. This pattern in moisture
content through time, however, did not directly predict flammability through
time.


The overall ranking of flammability in litter mixtures showed some consistency
with the results obtained for the individual species, but there were
significant non-additive effects. Mixtures containing \emph{Q. kelloggii} and
\emph{P. jeffreyi} tended to have fast spread rates. These results are likely
influenced by the lower bulk densities of litter mixtures containing those two
species. Most interestingly, there were striking patterns of non-additivity in
spread rate and in fuel consumption for the four mixtures, indicating that
positive non-additivity in flammability is not restricted to completely dry
litter mixtures observed in previous work \citep{Magalhaes+Schwilk-2012,
  VanAltena+Logtestjin+etal-2012}. Non-additivity in both spread rate and fuel
consumption became more positive with time since wetting and these patterns
were consistent across mixture identities. This is a slightly different result
than that reported by Blauw et al \citeyear{Blauw+Wensink+etal-2015}. Blauw et
al found that non-additivity was greater in moist mixtures of heathland fuels
than in dry fuels. In their work, non-additivity was generally negative but
cases of positive non-additivity in mixtures was often a result of mixtures
meeting ignition thresholds not met by the constituent species alone. To limit
the number of models fit, we limited our investigation to two flammability
parameters that represent two major axes of flammability variation. Therefore,
our study does not fully capture ignition effects although we do have some
results consistent with theirs: mixtures with \emph{Q. kellogii} could ignite
at times since wetting at which the species in monoculture could not.

Overall, our study demonstrates a consistent pattern of increasingly positive
non-additivity in flammability with litter drying across two largely orthogonal
measures of flammability. This result, together with previously published work,
suggests that litter flammability is prone to non-additive effects, suggesting
caution in lumping species into fuel categories. Community flammability is
dependent on and particular to its constituent species and cannot be addressed
by simply averaging the flammability of all species included. In litter-driven
fire, positive non-additivity appears to be the rule, and the relationships we
described here can form a working hypothesis and baseline expectation to adjust
fire behavior predictions in real communities: in mixtures with fuel moisture
content below 50\%, observed flame spread rates approached double the predicted
value based on monoculture trials (Fig. \ref{fig:mixture_obs_pred_spread}). In
completely dry litter from this same system, past work demonstrates observed
spread rates approached three times the expectation
\citep{Magalhaes+Schwilk-2012}.

Positive non-additivity, whereby more flammable species have effects on fire
behavior disproportionate to their contribution to the litter layer, could lead
to vegetation feedbacks. In this temperate mixed conifer forest, these effects
are likely to include positive feedbacks whereby the shade intolerant species
such as the long-needled pines and the large-leaved oak, are favored by the
increased fire to which their fuels contribute \citep{Schwilk+Caprio-2011} and
similar results are reported for southeastern US forests where species
composition shifts have led to decreases in fire activity
\citep{Nowacki+Abrams-2008}. Such positive feedback effects increases the
potential for species to exert community scale effects even when relatively
rare and that can influence community assembly and the evolutionary of
niche-constructing traits \citep{Kerr+Schwilk+etal-1999, Schwilk+Kerr-2002}.
Ongoing climate change and concurrent changes to fire regimes are linked
phenomena. Vegetation-fire feedbacks are important at a wide range of scales
from stand-level species composition shifts to the earth system
\citep{Harris+Remenyi+etal-2016, Archibald+Lehmann+etal-2018}. The patterns we
demonstrate here show to potential importance of species traits and
interactions at local and stand scales where management and fire operations
meet species' biology.


%%%%%%%%%%%%%%%%%%%%%%%%%%%%%%%%%%%%%%%%%%

\section*{Acknowledgments}

The authors thank the National Park Service and the employees of Sequoia and
Kings Canyon National Parks. This work was carried out with permission from NPS
under permit \# SEKI-2014-SCI-0040. The authors thank Russell Lackey who
contributed essential assistance with the field work and Gabrielle Plata and
Leticia Plata who assisted with the burning trials. We thank the Lubbock Fire
Department for their ongoing help and cooperation.

This research received no external funding.

\section*{Author contributions}

conceptualization, RMQM and DWS; methodology, RMQM and DWS; experiments, RMQM;
coding and data analysis, DWS and RMQM; writing DWS and RMQM; writing--review
and editing, DWS

\section*{Data and code accessibility}
All data and code for this project is at \url{https://github.com/schwilklab/trait-flam}

\newpage
\bibliography{Magalhaes_Schwilk_litter_flam_moisture.bib}

% %%%%%%%%%%%%%%%%%%%%%%%%%%%%%%%%%%%%%%%%%%
% \conflictsofinterest{The authors declare no conflict of interest. The funders had no role in the design of the study; in the collection, analyses, or interpretation of data; in the writing of the manuscript, or in the decision to publish the results.} 

% References, variant B: external bibliography
% =====================================

%%%%%%%%%%%%%%%%%%%%%%%%%%%%%%%%%%%%%%%%%%
\end{document}

