%% SI
\documentclass[letterpaper]{article}

\usepackage{booktabs}
\usepackage{graphicx}
\usepackage{fullpage}


\title{Magalhaes and Schwilk Supplementary Materials}
\author{Rita Quinones de Magalhaes and Dylan W. Schwilk}
%\date{\today}

% Hint: \title{what ever}, \author{who care} and \date{when ever} could stand 
% before or after the \begin{document} command 
% BUT the \maketitle command MUST come AFTER the \begin{document} command! 

\begin{document}

\maketitle

\section{Tables}

\begin{table}[h]
  \caption{Mixed model coefficients for drydown curves. Table shows linear model results of model fit to  natural log of dry-mass based fuel moisture content as a function of time and species.} 
  \label{tabS1}
\centering
\input{figs_tables/drydown-coef-tab.ltx}
\end{table}

\begin{table}[h]
  \caption{Mixed model coefficients for flame spread rate. Table shows mixed linear model results  TODO.} 
  \label{tabS3}
\centering
\input{figs_tables/spreadrate-coef-tab.ltx}
\end{table}

\section{Figures}

\begin{figure}[h]
  \centering
  \label{fig-S1}

  \includegraphics[width=17cm]{figs_tables/figS1_max_water_emmeans.pdf}
\caption{Estimated marginal means by species for maximum water retainability based on mixed model results. Shaded regions indicate 95\% confidence intervals for the marginal means.}

\end{figure}


\begin{figure}[h]
  \centering
  \label{fig-S2}
\includegraphics[width=17cm]{figs_tables/figS2_dessication_emmeans.pdf}
\caption{Estimated marginal means by species for dessication rate based on mixed model results. Shaded regions indicate 95\% confidence intervals for the marginal means.}
\end{figure}



\end{document}
