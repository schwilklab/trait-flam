%% SI
\documentclass[letterpaper]{article}

\usepackage{booktabs}
\usepackage{graphicx}
\usepackage{fullpage}
\usepackage{float}


\title{Magalhaes and Schwilk Supplementary Materials}
\author{Rita Quinones de Magalhaes and Dylan W. Schwilk}
\date{}

% Hint: \title{what ever}, \author{who care} and \date{when ever} could stand 
% before or after the \begin{document} command 
% BUT the \maketitle command MUST come AFTER the \begin{document} command! 

\begin{document}

\maketitle

\section{Tables}

\begin{table}[H]
  \caption{Mixed model coefficients for drydown curves. Table shows linear model results of model fit to  natural log of dry-mass based fuel moisture content as a function of time and species.} 
  \label{tabS1}
\centering
\input{figs_tables/SI_tab1_drydown_coef.ltx}
\end{table}

\begin{table}[H]
  \caption{ANOVA model coefficients for moisture content as a function of species traits} 
  \label{tabS2}
\centering
\input{figs_tables/SI_tab2_mc_coef.ltx}
\end{table}

\begin{table}[H]
  \caption{ANOVA model coefficients for drying rate as a function of species traits} 
  \label{tabS3}
\centering
\input{figs_tables/SI_tab3_di_coef.ltx}
\end{table}

\begin{table}[H]
  \caption{Mixed model coefficients for flame spread rate as a function of taxon and moisture content. Table shows mixed linear model results.} 
  \label{tabS4}
\centering
\input{figs_tables/SI_tab4_spread_moist_coef.ltx}
\end{table}

\begin{table}[H]
  \caption{Mixed model coefficients for fuel consumption as a function of taxon and moisture content. Table shows mixed linear model results.} 
  \label{tabS5}
\centering
\input{figs_tables/SI_tab5_consume_moist_coef.ltx}
\end{table}


\begin{table}[H]
  \caption{Mixed model coefficients for flame spread rate as a function of taxon and time since wetting. Table shows mixed linear model results.} 
  \label{tabS6}
\centering
\input{figs_tables/SI_tab6_spread_time_coef.ltx}
\end{table}

\begin{table}[H]
  \caption{Mixed model coefficients for fuel consumption as a function of taxon and time since wetting. Table shows mixed linear model results.} 
  \label{tabS6}
\centering
\input{figs_tables/SI_tab7_consume_time_coef.ltx}
\end{table}

% table 7
\begin{table}[H]
  \caption{Linear mixed model results for moisture content as a function of
    time since wetting for four different litter mixture types. Approximate
    degrees of freedom (df), pseudo F statistics and p-values were calculated
    by the Kenward-Roger approximation}.
  \label{tab:mixtures_drydown}
\centering

\begin{tabular}{llll}
  \toprule
Effect & df & $F$ & $P$ \\ 
  \midrule
time & 1, 96.19 & 565.28 & $<$0.0001 \\ 
  mixture & 3, 96.08 & 23.63 & $<$0.0001 \\ 
  time:mixture & 3, 95.71 & 8.70 & $<$0.0001 \\ 
   \bottomrule
\end{tabular}
\end{table}

\section{Figures}

\begin{figure}[H]
  \centering
  \includegraphics[width=8cm]{figs_tables/SI_fig1_drydown-curves_logged.pdf}
   \label{SI_fig1}
\caption[Semi-log scale dry down curves for eight litter types.]{Lines show best-fit lines fit to the natural log of moisture content on a dry mass basis.  The slope of the line shows the dessication rate ($hr^{-1}$). Litter samples we saturated by immersion for 24 hours and then allowed to drain by gravity for 3 min before initial weighing. Colors indicate genera and there were significant differences in both }

\end{figure}



\begin{figure}[H]
  \centering
  \label{SI_fig2}
  \includegraphics[width=17cm]{figs_tables/SI_fig2_max_water_emmeans.pdf}
\caption{Estimated marginal means by species for maximum water retainability based on mixed model results. Shaded regions indicate 95\% confidence intervals for the marginal means.}
\end{figure}


\begin{figure}[H]
  \centering
  \label{SI_fig3}
\includegraphics[width=17cm]{figs_tables/SI_fig3_dessication_emmeans.pdf}
\caption{Estimated marginal means by species for dessication rate based on mixed model results. Shaded regions indicate 95\% confidence intervals for the marginal means.}
\end{figure}



\begin{figure}[H]
  \centering
\includegraphics[width=8cm]{figs_tables/SI_fig4_mixture_drydown-curves.pdf}
\caption{Fuel moisture by time since saturation for four litter mixtures.  Codes for species in mixtures: Ab = \emph{Abies concolor}, Ca = \emph{Calocedrus decurrens}, Pi = \emph{Pinus jeffreyi} and Qu = \emph{Quercus kellogii}. Moistures were measured every 24 hours but data are shown with slight horizontal jitter to reduce point overlap. Mixed linear model results shown in SI Table 7.}
  \label{SI_fig4}
\end{figure}

% SI fig 5
\begin{figure}[H]
  \centering
\includegraphics[width=8cm]{figs_tables/fig2_SLA_maxMC.pdf}
\caption{Maximum water retention by specific leaf area across eight species.
  % ANOVA table in Table \ref{tab:mc_di_anova}.
}
  \label{fig:maxmc-di}
\end{figure}

% SI fig 6
\begin{figure}[H]
  \centering
\includegraphics[width=8cm]{figs_tables/fig3_di_bd.pdf}
\caption{Drying rate rate as a function of specific leaf area and litter bulk
  density across eight species.
  % ANOVA table in Table \ref{tab:mc_di_anova}
}.
  \label{fig:bd-di}
\end{figure}


\end{document}
