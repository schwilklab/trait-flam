 
\documentclass[letterpaper, 12pt]{letter}
\usepackage{times, fullpage}
\usepackage{url}

\address{Department of Biological Sciences\\Flint and Main\\Texas Tech University\\
  Lubbock, TX 79409}

\name{Dylan W. Schwilk}

\begin{document}
\begin{letter}{}

\opening{Dear Dr. David Gibson,}

Enclosed is our revised manuscript research article, ``''.

We thank the editor and reviewer for their helpful comments and suggestions. We
have itemized each comment below and followed each with our reply and a list of
changes we made in response to the comment. 

{\bf Responses to Senior Editor's comments:}

\begin{quote}
I support the AE's recommendation to reject but allow resubmission. However, the bar for a resubmission will be high. My feeling is that your study confirms the essential findings from your 2012 study published in Journal of Ecology with the addition that your methodology of drying down adds some realism. Indeed, the confirmatory nature of this study was pointed out by reviewer 2 in their point 6. In addition to the points raised by the AE and reviewers, I will be looking in any resubmission for more evidence that this study advances your earlier work (apart from your last paragraph, your discussion is largely an explanation of your results with little context from other studies), that the findings have real ecological relevance (i.e., does it matter to the mixed-conifer communities?) and how is the new dry down treatment related to dry down conditions in litter mixtures in the field. If you can adequately address all of these concerns then I'll be happy to consider a resubmission.
\end{quote}

We agree that our prior Discussion was weak and have extensively rewritten it.
We argue that this work is not merely confirmatory. Our revised Introduction
argues for the ecological significance of dry down rates. The importance of
fuel moisture is not under debate. Furthermore, fine scale heterogeneity in
fire behavior has been statistically associated wwith variation in fuel
moisture (Hille and Stephens 2005) and fire history can vary over short spatial
scales as a consequence of species composition (Stephens 2001). Although the
importance of dry-down rates is less well-established, prescribed burns in
Sierra Nevada mixed conifer forest demonstrate that fuel moisture likely drives
differences in fire behavior depending on the timing of fire throughout the
season (Kaufman and Martin 1989, Knapp and Keeley 2006, Hille and Stephens
2005, Knapp et al 2006). These differences in fire behavior in turn control
tree mortality ((Stephens et all 2002, Schwilk et al 2006). Despite this, we
know of only one other study specifically investigating differential drying
behavior (Kreye et al 2013).

TODO
Add context also suggested by reviewer 1.

\begin{quote}
 
\end{quote}

TODO


{\bf Responses to Associate Editor's comments:}

\begin{quote}
First, apologies to the authors for the long delay in the review process. Because of the rather contrasting assessments of the first two reviewers, I was waiting for the third reviewer to produce his/her review, but unfortunately we waited in vain. So I have to make a recommendation based on these two contrasting assessments plus my own judgement. On the one hand, in agreement with reviewer 1, I can see the novelty of this study. So far non-additivity in litter mixture flammability has been studied either at air-dry litter moisture or at fixed levels of litter moisture (e.g. Blauw et al 2015) and I do applaud the approach of bringing more field realism into these studies by allowing mixtures to dry out at different rates from expected values based on their single components. Having said that, there are also major problems with the current manuscript, as reported especially by reviewer 2 (and some additional ones by myself, see below). Besides these major problems there is a rather high degree of inaccuracy in the text.  Therefore, given the high rejection rates of often even good manuscripts to J. Ecol., I have to make a negative recommendation for this manuscript. However, given the novelty and principally nice design of the experiment, I would allow the authors to submit a strongly revised new manuscript, addressing all major and minor comments by the reviewers and myself, as a new submission. (At least one of the reviewers to check the revision would be new.) In addition to the points raised by the reviewers, I like to add some of my own:
\end{quote}

Thank you. Several comments converged on the absolutely correct criticism that our Discussion did not engage adequately with the literature. We have made substantial changes to the Introduction and the Discussion adding more context and citing more of the recent literature. We reorganized the Introduction and now do not rush the reader so quickly through the ideas. We have expanded our justification for the study as argued for by the Senior Editor.



\begin{quote}
The authors mention leaf size explicitly as an important parameter in non-additivity (with some references including their own 2012 paper) but they do not include it in the current study. Especially at low moisture, the small individual needles of Abies should cause low ventilation and a high likelihood of negative non-additivity on flammability, as was shown by Zhao et al 2016 Ecol. Evol. Figure 4). In the Methods, as observed also by reviewer 2, there is very little detail on the extent to which litter of the respective species was picked up as individual leaves, branchlets without leaves versus branchlets with leaves still attached. This is really important, also because positive versus negative non-additivity of leaf-twig mixtures have been reported recently (Zhao et al. 2019 Plant and Soil). I feel that data and/or photos indicating the size and shape of the different litter types (and if possible variability in these) are important to add and to be used in the interpretation of the results.
\end{quote}

We have added information to the methods explaining the litter collection in more detail. All material was naturally dropped litter which could include fine twigs and leaves. Litter in these two \emph{Abies} species is a combination of branchlets with still attached needles and individual needles that have detached from the branchlet.  We now cite the Zhao et al (2019) results when discussing the mixture results.

Yes, when deciding how to limit the number of tests at the comparitive scale (species as data points), we decided to not include size, because we believed its influence on flammability is primarily through litter packing and was therefore captured by litterbed density. We have added a figure (SI fig. 1) to the SI materials showing litter particle size (the containing prism volumes) distributions for each of these species. These data were collected from the stored litter. Therefore, the data show the particle size distributions of the material that was burned. We did not add these data to the main figures as they are not part of any statistical test or main conclusions, but we are happy to do so if the editor suggests.


\begin{quote}
In the Methods, it was unclear how the various treatments and replicates were grouped over time, given that temperature and humidity in the fire facility could vary. Were there blocks in time with one replicate for each of the treatments (in random burning order) within a block? Or was there a different statistical design in time? 
\end{quote}

Guided by the drying rates measured in the initial trials, we staggered starting saturation times across species. This was aimed at allowing a completely blocked design. However, because some ignition attempts were unsuccessful, some days lacked the full complement of species while others, later in the process, sometimes had more than one replicate per species burned. We've added this information to the text [Lines X-X].

\begin{quote}
Line 196 (see also reviewer 2): how can we assume that the quality of the stored litter is the same as that used in the previous study 3 years earlier, i.e. can these two datasets be safely combined?
\end{quote}

We have added detail as requested by reviewer 2. See lines X-X. All litter was
initially oven dried in large paper bags. The litter was then stored in the lab
at low humidity for several weeks on open racks. The litter was then bagged and
sealed and stored within cardboard boxes in a lab storage area. We saw no signs
of mold nor ongoing decomposition. We minimized handling of the bags to avoid
mechanical damage and changes to litter particle size.


\begin{quote}
Line 365 (see also reviewer 1): on the other hand Cornwell et al. 2015 New Phyt reported mostly high flammability in a range of Cupressaceae.
\end{quote}

This is a good point and another place where we did not discuss the literature fully. We
have now added this to the discussion and made the conjecture that these
differing patterns may result from different structure across the
Cupressaceae with Australian \emph{Callitris} and the two Californian species
packing more tightly and resulting in less flammable litter than the Asian
Cupressaceae in Cornwell et al (2015).


{\bf Responses to the Reviewer 1's comments}

\begin{quote}
This work compares mixtures of fuels, relative to their individual components, in terms of moisture retention and flammability. Moisture dynamics were also investigated in relation to properties of leaves of the species forming litter. This information is important in understanding flammability of surface fuels in the field. 

The study has been well thought through and well executed, and the ms is generally well written. Stylistically, the main problem is that commas are missing in many sentences. This might sound trivial, but will really help the reader.
\end{quote}

We thank the reviewer for their comments and apologize for the failure in
proofreading. The reviewer should not have had to conduct copy editing and we
sincerely apologize for that. We have taken heed of the reviewers' comments and
fixed all specific grammar errors and missing symbols. For brevity, we've
omitted responses to the very simple suggested grammar fixes and simply made all
requested changes. We respond to other points individually.

\begin{quote}
line 72 - the comment about natural lightning ignitions comes from nowhere, and is a  bit of a jolt for the reader - rephrase slightly, e.g. many fires result from lightning ignitions, which are concurrent with ....
\end{quote}

This portion was poorly written and this was also mentioned by Reviewer 2. We have rewritten this in response to reviewer 2's comment that most fires result from dry lightning strikes across the west. See our responses to Reviewer 2, below.

\begin{quote}
line 103 - 'litter packing therefore affects fire behaviour in two ways' is easier to read
\end{quote}

Yes, agreed. We have tried to clean up the writing throughout and make the prose more active and precise.

\begin{quote}
line 133 - 'in mixtures of litter with varying levels..' is more explicit
\end{quote}

Agreed and fixed.

\begin{quote}
line 182 - affect, not effect
\end{quote}

Yes, an embarrassing error.

\begin{quote}
line 220 - see also Prior et al (2018)  Conceptualizing ecological flammability ....Fire 1, article 14
\end{quote}

Thank you. This was an oversight. We have also added more context from the
literature to our Discussion. See lines XX-XX.


\begin{quote}
lines 243/244 I was a bit puzzled here --- It would help to be more specific here --- which aspects of moisture dynamics? Which response variables did you use?  (or 'as described below')
\end{quote}

We have rewritten that section to be more precise.  See lines XX.

\begin{quote}
line 290 - you show an association, not necessarily an effect
\end{quote}

We have edited this as suggested.

\begin{quote}
line 307 and line 326 - a minor inconsistency is that you analyse present species groups, yet discuss the results in terms of individual species
\end{quote}

TODO!

Yes. We have made edits to

TODO. Check wording.



\begin{quote}
lines 348-349 - logically, this does not necessarily follow - factors such as large leaves or high SLA could independently affect both absorptive capacity and flammability - although fuel moisture does of course have a strong effect 
\end{quote}

Yes, we think that our logic must not have been clear here as we don't see the contradiction. We  have rewritten these sentences and we hope they are no longer confusing. See lines X-X.


\begin{quote}
Table 4 - get rid of the unnecessary zeroes in the df column
\end{quote}

Fixed.

\begin{quote}
line 365 - wording is problematic - change to '...ignition is possible, it had faster spread rates ...'
\end{quote}

Agreed. Fixed. 

\begin{quote}
line 365 - The low flammability of litter from your Cupressaceae species is interesting - Scarff and Westoby (2006) also found this for the Australian Cupressaceae genus Callitris, as did Bowman et al. (2018) J. Ecol 106, 1010 -1022
\end{quote}

We have added some better discussion of past results and included discussion of these papers, as well as contrasting with the Cornwell et al 2015 result as pointed out by the Associate Editor. See lines XX-XX.

\begin{quote}
line 387 - capitalise 'I' in SI Table  8
\end{quote}

Fixed.

\begin{quote}
line 403 - change to 'to the completely dry litter mixtures studied in previous work'
\end{quote}

Done. See line XX.


\begin{quote}
Figures 7 and 8 - is there a reason that you don't present the information in the same way as in Figure 6, if your main point is the discrepancy between the mixtures and the averaged individual values?
\end{quote}

We agree this does lack consistency and an early draft used the same concise format of Fig 6 for all three figures. However, we decided that having time on the x-axis was worth the space and allows the reader more insight into the flammability effects of drying over time. Reviewer 2 pointed out some exceptions to the overall statistical pattern and those exceptions would not have been easily visible without an explicit time axis. There is a trade-off here between illustrating simply the results of the statistical model and allowing the reader to get a better idea of variation in flammability consequences of drying time.

\begin{quote}
line 416 - Prior et al (2018) also show that rate of spread and consumability are largely independent in litter fuels
\end{quote}

Yes. That was an oversight. We have added citations here. See lines XX-XX

{\bf Responses to the Reviewer 2's comments}

\begin{quote}

My major concerns are:
1. One motivation for the study is not valid. I do not completely agree with the authors that natural lightning ignitions are concurrent with precipitation and, therefore, fire behavior can depend on rate of drying following a wetting event. The efficiency of individual lighting strikes in igniting a fire is affected by the moisture properties of forest fuel at the moment of lighting (Krawchuket et al. 2006, Ecology). Dry lighting that occurs without any accompanied rainfall nearby is most likely to cause wildfire because the fuel is dry enough to be ignited.
\end{quote}

This sentence was poorly written and oversimplified. We thank the reviewer for
pointing this out and we have rewritten this section. However, our badly made
point was: lighting ignitions can occur during the same general period as
precipitation; therefore, drying rates are important. As the reviewer points
out, dry lightning strikes that find dry fuel are more likely to lead to fires
than are strikes that find damp fuels (Abatzoglou et al 2016). The fact that
the total number lightning strikes does not well predict burned area and that
ignitions occur often from dry strikes argues for the importance of fuel
moisture. Although synoptic weather conditions that create dry and wet storms
differ, the potential for precipitation and lightning during the same overall
period of thunderstorm activity often exists (eg wet and dry convection days
can occur under overlapping conditions, Rorig and Ferguson 1999). We've
rewritten this section, see lines XX-XX.


Cite:

Article{rorig1999characteristics,
  title={Characteristics of lightning and wildland fire ignition in the Pacific Northwest},
  author={Rorig, Miriam L and Ferguson, Sue A},
  journal={Journal of Applied Meteorology},
  volume={38},
  number={11},
  pages={1565--1575},
  year={1999}
}


@article{abatzoglou2016controls,
  title={Controls on interannual variability in lightning-caused fire activity in the western US},
  author={Abatzoglou, John T and Kolden, Crystal A and Balch, Jennifer K and Bradley, Bethany A},
  journal={Environmental Research Letters},
  volume={11},
  number={4},
  pages={045005},
  year={2016},
  publisher={IOP Publishing}
}

http://www.publish.csiro.au/wf/WF08117

@article{van2008temporal,
  title={Temporal and spatial distribution of lightning strikes in California in relation to large-scale weather patterns},
  author={Van Wagtendonk, Jan W and Cayan, Daniel R},
  journal={Fire Ecology},
  volume={4},
  number={1},
  pages={34--56},
  year={2008},
  publisher={Springer}
}



\begin{quote}
2. The research question of the research is not explicitly stated both in the abstract and in the introduction. In the end of the introduction, the authors focus on specific data-collection goals, while leave the reader to figure out what the research question is. 
\end{quote}

We have edited the abstract to make the research questions clearer. Yes, our
first submission jumped from stating current reserach gaps to specific methods
too quickly. See also our response to the Senior Editor, above.


\begin{quote}
3. The authors concluded that positive non-additivity appeared to be the rule in litter-driven fire and this non-additivity became more positive with fuel drying. But if we look closely at figure 8, non-additivity was negative within 50 hours since wetting, non-additivity was both positive and negative at about 75 hours since wetting, and non-additivity was positive at about 100 hours since wetting. The high variability of non-additive effects at about 75 hours since wetting might have important ecological effects on fire ignition and behavior, and are important information cannot be ignored.
\end{quote}

Yes, this is a good point and we have added some discussion of this. We tried to limit our statements to those supported by the statistical tests (eg, slope of nonaddivity with drying was greater than 1), and therefore focused on the positive effect of drying and the overall trend for positive nonadditivity. This, of course, obscures some details which are worth mentioning and we have edited this section in response to the reviewer's criticism. We agree that the high variability in nonadditivity across mixtures deserves highlighting and that is the reason we presented the actual predictions and departures from prediction across time. Please see our changes on lines X-X.


\begin{quote}
4. The writing of this manuscript needs to be improved for clarity and accuracy. There are many typos in the text, and the using to scientific terms lack consistency through the text.
\end{quote}

We have made extensive edits in response to both reviewer's suggestions.

\begin{quote}
5. The title does not precisely communicate the key message of the study. I suggest it is more precise to say “moisture effect … and alter non-additive flammability…”
\end{quote}

We have changed the title.

\begin{quote}
6. The last paragraph of the discussion has a significant overlap with the last paragraph of the discussion in the authors’ previous paper (de Magalhães \& Schwilk 2012) in terms of the main content.
\end{quote}


TODO

We have rewritten much of the discussion to better include findings from the literature and rewritten the concluding paragraph. 

\begin{quote}
The abstract can be shorter. It exceeds 350 words limit. 
\end{quote}

We have editing the abstract to more clearly state the research questions reduce its length.

\begin{quote}
Line 50: add reference to the first sentence.
\end{quote}

We thought this was an established enough fact to not need citation but have added one.

\begin{quote}
Lines 50-61: in this paragraph, after the first sentence I think it is more logic to first explain the mechanism of moisture’s negative effects on fire, then explain the reason of the driving importance of dead fuel moisture content to surface fire ignition and behavior because fine dead fuels respond more readily to environmental changes than living fuels.
\end{quote}

We have rearranged the logic of the introduction as suggested.

\begin{quote}
Line 72: revise the description “ litter fuel driven system”.
\end{quote}

We have rewritten this section and removed that phrase.

\begin{quote}
Line 75: “at any particular time”?
\end{quote}

We have rewritten this for clarity. See line XX.

\begin{quote}
Line 85: revise “driven by by” and “emergent litter litter”.
\end{quote}

Fixed.

\begin{quote}
Line 88: “similarly”?
\end{quote}

We were not sure what the reviewer's criticism was here but have rewritten this system to be more detailed and precise.

\begin{quote}
Line 95: “absorption capacity” and “rate of moisture loss” are not correct descriptions of processes. Better say “absorption and retention”.
\end{quote}

Good point. We fixed that here and in another sentence elsewhere. Furthermore, we've edited to be more precise about what we mean by ``absorption'' because we are referring to a litter bed and we include adsorption to litter surfaces in that definition. We had originally tried to avoid the chemical terms because we are observing a higher-level phenomenon.

\begin{quote}
Line 98: revise “after saturation” to “dried certain time after saturation” for clarity.
\end{quote}

Fixed.

\begin{quote}
Line 101: delete “also”.
\end{quote}

Fixed.

\begin{quote}
Lines 102-103: litter bulk density is not a litter particle trait. Delete “will influence air flow through the litter bed”.
\end{quote}

Fixed.

\begin{quote}
Lines 103-105: leaf size is not leaf litter trait.
\end{quote}

We have rewritten this. See line X.

\begin{quote}
Line 107: “fuel-beds composed of larger leaves with lower moisture retainability”?
\end{quote}

We edited to ``composed of larger leaves or leaves with lower moisture retainability''

\begin{quote}
Line 115: start a new paragraph here.
\end{quote}

We have broken these ideas into multiple paragraphs.

\begin{quote}
Lines 117-119: better specify the system, because negative non-additive effects on litter
flammability were also found.
\end{quote}

Yes, excellent point. We have added detail and citations here (lines X-X).

\begin{quote}
Line 120: “especially if different leaf traits influence absorptive capacity and drying rate”?
\end{quote}

TODO?

\begin{quote}
Line 125: revise “two leaf and litter bed traits”, and better directly say “SLA and packing density”.
\end{quote}

Yes, this was incorrect. Fixed.

\begin{quote}
Line 133: delete “also”.
\end{quote}

Fixed.

\begin{quote}
Line 134: “if species contribute equally to litter moisture dynamics”?
\end{quote}

TODO

We have rewritten this. See line X.

\begin{quote}
Line 136: delete “on”. Revise “ecologically-relevant”.
\end{quote}

TODO

We assume that the reviewer's complaint is that ``ecologically-relevant'' is vague.

\begin{quote}
Line 141: revise “(36 36N, 118 42 W)”.
\end{quote}

Symbols fixed throughout.

\begin{quote}
Line 147: at least for Quercus, leaves fallen in the previous year will be partly decomposed in the summer.
\end{quote}

Yes. We have added some detail to this and also now provide particle size distributions for each species measured on the actual stored litter (SI fig 1).

\begin{quote}
Line 155: add more details about how the litter was collected?
\end{quote}

We have added information to the methods explaining the collection in more detail. See lines XX and our response to the Associate Editor, above. 

\begin{quote}
Line 157: is it possible that all litter has less than 5\% fuel moisture when air-dried?
\end{quote}

We have corrected the methods because the litter was initially oven dried in large paper bags, then stored on racks in the low humidity lab. Our subsample measurements showed less than 5\% moisture.  See lines X-X.

\begin{quote}
Line 158: the collecting of the leaf litter is in 2012. The conduction of the experiment is in 2015. Can the author explain how do they save the litter?
\end{quote}

We have added detail on the long term dry storage. See lines X-X.

\begin{quote}
Line 167: 1 cm ? aluminum screen.
\end{quote}

Yes, this was a typo. Thanks. The aluminum mesh openings were 0.203 inches. We have clarified and corrected to ``0.52\,cm'' 

\begin{quote}
Line 173: revise “21C temperature”.
\end{quote}

Symbols fixed.

\begin{quote}
Line 182: revise “effect” to “affect”. 	Revise “litter bulk density” to “litter packing density”
\end{quote}

Fixed.

\begin{quote}
Line 185: I do not think this sentence is necessary because whether or not petioles should be included in SLA measurements depends on the research question at hand.
\end{quote}

Agreed, fixed.

\begin{quote}
Line 201: the fuel depth was controlled or not?
\end{quote}

No, mass was fixed, width and length of the burning tray was fixed (15\,cm by 1.5\,m), and depth was allowed to vary.  We edited this for clarity. See lines X-X

\begin{quote}
Line 205: revise “1899C”.
\end{quote}


Symbols fixed.

\begin{quote}
Line 208: revise “species/moisture combination” to “species and moisture combination”.
\end{quote}

Fixed.

\begin{quote}
Line 209: “the replicate was allowed to continue drying for …”
\end{quote}

TODO check

\begin{quote}
Line 216: revise “mass loss” to “fuel consumption”.
\end{quote}

Fixed.

\begin{quote}
Line 224: if not all flammability data was used in the analysis, then why measure them?
\end{quote}

Flammability is a multivariate trait. To avoid collinearity and to simplify
models, two general approaches are possible: either reduce and transform the
response variables and create synthetic variables through a method such as PCA,
or pick a few variables representative of the axes of variation. When the
experiment was designed, we had intended to use the first method. The loss of
several days of thermocouple data made that an unattractive and unnecessary
option, so we opted for the simpler to explain method of picking two traits
already known to correlate with different axes of litter flammability. I envy
the scientist who has never dealt with unforeseen events.

\begin{quote}
Line 242: revise “one leaf trait and one litter trait” to “one litter particle trait and one litter bed traits”.
\end{quote}

TODO


\begin{quote}
Line 307: revise “at that point”.
\end{quote}

Although we are unsure of the criticism here, we have rewritten the sentence.

\begin{quote}
Lines 314-320: this indicates that even in the same genus, species vary in flammability.
\end{quote}

Yes.  This is not surprising in the case of these three pines where one has quite different needles. But our \emph{a priori} groups were based on genus.

\begin{quote}
Line 330: “rates”?
\end{quote}

TODO

\begin{quote}
Line 344: “moisture dynamics flammability” is difficult to understand.
\end{quote}

Yes, oops a typo. We have rewritten the heading. Line X

\begin{quote}
Line 346: revise “apparently controlled”.
\end{quote}

Fixed.

\begin{quote}
Line 350: revise “very dry”.
\end{quote}

TODO

\begin{quote}
Line 351: revise “more realistic moisture levels”.
\end{quote}

Agreed. That was vague. Fixed.

\begin{quote}
Line 361: revise “that that”.
\end{quote}

Fixed.

\begin{quote}
Line 366: “but”?
\end{quote}

TODO.

\begin{quote}
Line 401: “striking”?
\end{quote}

We removed the offending word.

\begin{quote}
Line 410: “not met by the constituent species alone”?
\end{quote}

We don't understand the criticism. Perhaps it is the word ``alone'' that is ambiguous?  We changed ``alone'' to ``in monoculture'' although we thought the original wording was clear.

\begin{quote}
Line 415: “demonstrates”?
\end{quote}

We don't understand the criticism. Academic writing commonly uses present tense to describe results that still apply. There is no grammatical error.

\begin{quote}
Line 439: revise “to”.
\end{quote}

TODO

\closing{Sincerely,}

\end{letter}
\end{document}
