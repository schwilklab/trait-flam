% Response letter 2019-12-20

\documentclass[letterpaper, 12pt]{letter}
\usepackage{times, fullpage}
\usepackage[font=itshape]{quoting}
\usepackage{url}

\address{Department of Biological Sciences\\Flint and Main\\Texas Tech University\\
  Lubbock, TX 79409}

\name{Dylan W. Schwilk}

\begin{document}
\begin{letter}{}

\opening{Dear Dr. David Gibson,}

Enclosed is our revised manuscript research article, ``Moisture absorption and drying alter nonadditive litter flammability in a mixed conifer forest''.

We thank the editors and reviewers for their substantial comments and
suggestions. We have itemized each comment below and followed each with our
reply and a description of changes we made in response to the comment.

{\bf Responses to the Senior Editor's comments:}

\begin{quoting}
  I support the AE's recommendation to reject but allow resubmission. However,
  the bar for a resubmission will be high. My feeling is that your study
  confirms the essential findings from your 2012 study published in Journal of
  Ecology with the addition that your methodology of drying down adds some
  realism. Indeed, the confirmatory nature of this study was pointed out by
  reviewer 2 in their point 6. In addition to the points raised by the AE and
  reviewers, I will be looking in any resubmission for more evidence that this
  study advances your earlier work (apart from your last paragraph, your
  discussion is largely an explanation of your results with little context from
  other studies), that the findings have real ecological relevance (i.e., does
  it matter to the mixed-conifer communities?) and how is the new dry down
  treatment related to dry down conditions in litter mixtures in the field. If
  you can adequately address all of these concerns then I'll be happy to
  consider a resubmission.
\end{quoting}

We agree that our Discussion was weak and we have rewritten it. We argue,
however, that this work is not merely confirmatory. Our revised Introduction
argues for the ecological significance of drying rates. The importance of fuel
moisture is well established. At large scales, fire in the western US is often
limited by fuel moisture (Parks et al 2014). At fine scales, variation in fuel
moisture influences heterogeneity in fire behavior (Hille and Stephens 2005)
and may contribute to fire history variation over short distances (Stephens
2001). Although the importance of drying rate is less well-established,
prescribed burns in Sierra Nevada mixed conifer forest demonstrate that fuel
moisture drives differences in fire behavior depending on the timing of fire
throughout the season (Kaufman and Martin 1989, Knapp and Keeley 2006, Hille
and Stephens 2005, Knapp et al 2006). These differences in fire behavior cause
spatial variation in tree mortality (Stephens et al 2002, Schwilk et al 2006).
Despite this, we know of only one other study specifically investigating
differential drying behavior (Kreye et al 2013b) and none that included drying
rates with flammability measurement.

The editor asks that we compare fuel moisture with field measurements. Litter
fuels can vary from single digits to over 100\% fuel moisture throughout the
fire season. For example, relatively fine fuels in northern California varied
from less than 10\% to over 80\% in response to repeated precipitation events
throughout the fire season (Estes et al 2012). Early season burns in this mixed
conifer system show evidence of moisture limiting fire behavior at 12-14\% fuel
moisture (Knapp et al 2005). We have added this information to the Discussion
(lines 396--404).

{\bf Responses to the Associate Editor's comments:}

\begin{quoting}
  First, apologies to the authors for the long delay in the review process.
  Because of the rather contrasting assessments of the first two reviewers, I
  was waiting for the third reviewer to produce his/her review, but
  unfortunately we waited in vain. So I have to make a recommendation based on
  these two contrasting assessments plus my own judgement. On the one hand, in
  agreement with reviewer 1, I can see the novelty of this study. So far
  non-additivity in litter mixture flammability has been studied either at
  air-dry litter moisture or at fixed levels of litter moisture (e.g. Blauw et
  al 2015) and I do applaud the approach of bringing more field realism into
  these studies by allowing mixtures to dry out at different rates from
  expected values based on their single components. Having said that, there are
  also major problems with the current manuscript, as reported especially by
  reviewer 2 (and some additional ones by myself, see below). Besides these
  major problems there is a rather high degree of inaccuracy in the text.
  Therefore, given the high rejection rates of often even good manuscripts to
  J. Ecol., I have to make a negative recommendation for this manuscript.
  However, given the novelty and principally nice design of the experiment, I
  would allow the authors to submit a strongly revised new manuscript,
  addressing all major and minor comments by the reviewers and myself, as a new
  submission. (At least one of the reviewers to check the revision would be
  new.) In addition to the points raised by the reviewers, I like to add some
  of my own:
\end{quoting}

Thank you for the consideration and comments. Several reviewer comments
converged on the correct criticism that our Discussion did not engage
adequately with the literature. We have made substantial changes to the
Introduction and the Discussion. We added context and cited more of the
recent literature. We reorganized the Introduction and now do not rush the
reader so quickly through the ideas. We have expanded our justification for the
study as argued for by the Senior Editor and others.


\begin{quoting}
  The authors mention leaf size explicitly as an important parameter in
  non-additivity (with some references including their own 2012 paper) but they
  do not include it in the current study. Especially at low moisture, the small
  individual needles of Abies should cause low ventilation and a high
  likelihood of negative non-additivity on flammability, as was shown by Zhao
  et al 2016 Ecol. Evol. Figure 4).
\end{quoting}

We have added a new figure in response to this comment (SI Fig 1).

When considering how to limit the number of tests at the comparative scale
(species as data points), we decided to not include size, because it affects
flammability primarily through litter packing and was captured by litterbed
density. However, in response to the request to show particle sizes, we have
added a figure (SI Fig. 1) to the SI materials showing litter particle size
distributions for each of these species. These data were collected from the
stored litter. We did not add these data to the main figures as they are not
part of any statistical test or main conclusions, but we are happy to do so if
the editor suggests.


\begin{quoting}
  In the Methods, as observed also by reviewer 2, there is very little detail
  on the extent to which litter of the respective species was picked up as
  individual leaves, branchlets without leaves versus branchlets with leaves
  still attached. This is really important, also because positive versus
  negative non-additivity of leaf-twig mixtures have been reported recently
  (Zhao et al. 2019 Plant and Soil). I feel that data and/or photos indicating
  the size and shape of the different litter types (and if possible variability
  in these) are important to add and to be used in the interpretation of the
  results.
\end{quoting}

We have added information to the methods explaining the litter collection in
more detail. All material was naturally dropped litter which could include
twigs and leaves. Litter in these two \emph{Abies} species is a combination of
branchlets with still attached needles and individual needles that have
detached. We now cite the Zhao et al (2019) results when discussing the mixture
results (lines 466--470).

\begin{quoting}
  In the Methods, it was unclear how the various treatments and replicates were
  grouped over time, given that temperature and humidity in the fire facility
  could vary. Were there blocks in time with one replicate for each of the
  treatments (in random burning order) within a block? Or was there a different
  statistical design in time?
\end{quoting}

Guided by the drying rates measured in the initial trials, we staggered
starting saturation times across species. This was aimed at allowing a
completely blocked design. This was largely successful, but because some
ignition attempts failed, some days lacked the full complement of species.
We've added this information to the text (lines 218-221 and see lines 230-232
for our explanation of how we used weather as a covariate).

\begin{quoting}
  Line 196 (see also reviewer 2): how can we assume that the quality of the
  stored litter is the same as that used in the previous study 3 years earlier,
  i.e. can these two datasets be safely combined?
\end{quoting}

We have added detail as requested by reviewer 2. See lines 170-177. All litter
was initially oven dried in large paper bags. The litter was then stored in the
lab at low humidity for several weeks on open racks. The litter was then bagged
and sealed and stored within cardboard boxes in a lab storage area. We saw no
signs of mold nor ongoing decomposition. We minimized handling of the bags to
avoid mechanical damage and changes to litter particle size.


\begin{quoting}
  Line 365 (see also reviewer 1): on the other hand Cornwell et al. 2015 New
  Phyt reported mostly high flammability in a range of Cupressaceae.
\end{quoting}

This is a good point and another place where we did not discuss the literature
fully. We have now added this to the Discussion and made the conjecture that
these differing patterns may result from different structure across the
Cupressaceae with Australian \emph{Callitris} and the two Californian species
packing more tightly and resulting in less flammable litter than the Asian
Cupressaceae in Cornwell et al (2015). See lines 409-415.


{\bf Responses to the Reviewer 1's comments}

\begin{quoting}
  This work compares mixtures of fuels, relative to their individual
  components, in terms of moisture retention and flammability. Moisture
  dynamics were also investigated in relation to properties of leaves of the
  species forming litter. This information is important in understanding
  flammability of surface fuels in the field.

  The study has been well thought through and well executed, and the ms is
  generally well written. Stylistically, the main problem is that commas are
  missing in many sentences. This might sound trivial, but will really help the
  reader.
\end{quoting}

We thank the reviewer for their comments and apologize for our failure in
proofreading. We have taken heed of the reviewers' comments and fixed all
specific grammar errors and missing symbols. For brevity, we've omitted
responses to the straightforward grammar fixes and simply made all requested
changes. We respond to other points individually below.

\begin{quoting}
  line 72 - the comment about natural lightning ignitions comes from nowhere,
  and is a bit of a jolt for the reader - rephrase slightly, e.g. many fires
  result from lightning ignitions, which are concurrent with ....
\end{quoting}

This portion was poorly written and this was also mentioned by Reviewer 2. We
have rewritten this in response to Reviewer 2's comment that most fires that
spread result from dry lightning strikes across the west. See our responses to
Reviewer 2, below and the paragraph now at lines 100-108.

\begin{quoting}
  line 103 - 'litter packing therefore affects fire behaviour in two ways' is
  easier to read
\end{quoting}

Done. We have tried to clean up the writing throughout and make the prose more
active and precise.

\begin{quoting}
  line 133 - 'in mixtures of litter with varying levels..' is more explicit
\end{quoting}

Agreed and fixed.

\begin{quoting}
line 182 - affect, not effect
\end{quoting}

Yes, an embarrassing error.

\begin{quoting}
  line 220 - see also Prior et al (2018) Conceptualizing ecological
  flammability ....Fire 1, article 14
\end{quoting}

Thank you. This was an oversight. Fixed on line 260. We have also added more
context from the literature to our Discussion. See the paragraph that begins on
line 462.


\begin{quoting}
  lines 243/244 I was a bit puzzled here --- It would help to be more specific
  here --- which aspects of moisture dynamics? Which response variables did you
  use? (or 'as described below')
\end{quoting}

We have rewritten that section to be more precise (lines 276-278). 

\begin{quoting}
line 290 - you show an association, not necessarily an effect
\end{quoting}

We have edited as suggested (line 324)

\begin{quoting}
  line 307 and line 326 - a minor inconsistency is that you analyse present
  species groups, yet discuss the results in terms of individual species
\end{quoting}

We've tried to increase consistency throughout. Yes, we examined dry down
curves by species because we did not have prior hypothesis about groups and
because we had decided to find moisture dynamic groups empirically for deciding
on mixtures. For the flammability analyses, we had decided to examine patterns
by taxonomic group because we thought that would allow clearer presentation and
aid generalizability and understanding. We now refer to taxonomic groups when
appropriate and have standardized the subsection headings. We could do all
analyses on a species basis, but that would be re-running new models on those
data.

\begin{quoting}
  lines 348--349 - logically, this does not necessarily follow - factors such as
  large leaves or high SLA could independently affect both absorptive capacity
  and flammability - although fuel moisture does of course have a strong effect
\end{quoting}

We have rewritten these sentences and we hope they are no longer confusing. See
lines 384--388.

\begin{quoting}
Table 4 - get rid of the unnecessary zeroes in the df column
\end{quoting}

Done.

\begin{quoting}
  line 365 - wording is problematic - change to '...ignition is possible, it
  had faster spread rates ...'
\end{quoting}

Done. 

\begin{quoting}
  line 365 - The low flammability of litter from your Cupressaceae species is
  interesting - Scarff and Westoby (2006) also found this for the Australian
  Cupressaceae genus Callitris, as did Bowman et al. (2018) J. Ecol 106, 1010
  -1022
\end{quoting}

We have added discussion of past results and included these papers. We also now
contrast our result with that of Cornwell et al (2015) as suggested out by the
Associate Editor. See lines 409--415.

\begin{quoting}
line 387 - capitalise 'I' in SI Table  8
\end{quoting}

Done.

\begin{quoting}
line 403 - change to 'to the completely dry litter mixtures studied in previous work'
\end{quoting}

Done.

\begin{quoting}
  Figures 7 and 8 - is there a reason that you don't present the information in
  the same way as in Figure 6, if your main point is the discrepancy between
  the mixtures and the averaged individual values?
\end{quoting}

We agree this does lack consistency and an early draft used the same concise
format of Fig 6 for all three figures. However, we decided that having time on
the x-axis was worth the space and allows the reader more insight into the
flammability effects of drying over time. Reviewer 2 pointed out some
exceptions to the overall statistical pattern and those exceptions would not
have been easily visible without an explicit time axis. There is a trade-off
here between only illustrating the results of the statistical model and
allowing the reader to get a better idea of variation in flammability
consequences of drying time.

\begin{quoting}
  line 416 - Prior et al (2018) also show that rate of spread and consumability
  are largely independent in litter fuels
\end{quoting}

Yes. That omission was an oversight. We have added citations here. See lines
462--465

{\bf Responses to the Reviewer 2's comments}

\begin{quoting}

  My major concerns are:
  
  1. One motivation for the study is not valid. I do not completely agree with
  the authors that natural lightning ignitions are concurrent with
  precipitation and, therefore, fire behavior can depend on rate of drying
  following a wetting event. The efficiency of individual lighting strikes in
  igniting a fire is affected by the moisture properties of forest fuel at the
  moment of lighting (Krawchuk et al. 2006, Ecology). Dry lighting that occurs
  without any accompanied rainfall nearby is most likely to cause wildfire
  because the fuel is dry enough to be ignited.
\end{quoting}

Our sentence was poorly written and oversimplified. We thank the reviewer for
pointing this out and we have rewritten this section. However, our badly made
point was: lighting ignitions can occur during the same general period as
precipitation; therefore, drying rates are important. As the reviewer points
out, dry lightning strikes that find dry fuel are more likely to lead to fires
than are strikes that find damp fuels (Abatzoglou et al 2016). The fact that
the total number lightning strikes does not well predict burned area and that
observed ignitions occur most often from dry strikes argues for the importance
of fuel moisture. Although synoptic weather conditions that create dry and wet
storms differ, the potential for precipitation and lightning during the same
overall period of thunderstorm activity often exists (eg wet and dry convection
days can occur under overlapping conditions, Rorig and Ferguson 1999). We've
rewritten this section and added details justifying the importance of drying
rates in fine fuels as suggested by the Senior Editor. See lines 100--108 and
other changes to the Introduction.

\begin{quoting}
  2. The research question of the research is not explicitly stated both in the
  abstract and in the introduction. In the end of the introduction, the authors
  focus on specific data-collection goals, while leave the reader to figure out
  what the research question is.
\end{quoting}

We have edited the abstract to make the research questions clearer. We agree
that our first submission jumped from stating current reserach gaps to specific
methods too quickly. See also our response to the Senior Editor, above.


\begin{quoting}
  3. The authors concluded that positive non-additivity appeared to be the rule
  in litter-driven fire and this non-additivity became more positive with fuel
  drying. But if we look closely at figure 8, non-additivity was negative
  within 50 hours since wetting, non-additivity was both positive and negative
  at about 75 hours since wetting, and non-additivity was positive at about 100
  hours since wetting. The high variability of non-additive effects at about 75
  hours since wetting might have important ecological effects on fire ignition
  and behavior, and are important information cannot be ignored.
\end{quoting}

This is a good point and we have added some discussion of this pattern. We
tried to limit our statements to those supported by the statistical tests.
This, of course, obscures some details which are worth mentioning and we have
edited this section in response to the reviewer's criticism. We agree that the
high variability in nonadditivity across mixtures deserves highlighting and
that is the reason we presented the actual predictions and departures from
prediction across time. Please see our revised Discussion and lines 462--474.

\begin{quoting}
  4. The writing of this manuscript needs to be improved for clarity and
  accuracy. There are many typos in the text, and the using to scientific terms
  lack consistency through the text.
\end{quoting}

We have made extensive edits in response to both reviewer's suggestions.
Furthermore, we have standardized our usage of some terms. We use the term
``maximum water absorption'' to indicate water content following saturation. Of
course, this is slightly inaccurate and includes both absorption into particles
and adsorption to surfaces. Our earlier draft used the term ``maximum water
retention''. But we think this earlier wording was more confusing (and we were
inconsistent in usage as the reviewer points out).

\begin{quoting}
  5. The title does not precisely communicate the key message of the study. I
  suggest it is more precise to say “moisture effect … and alter non-additive
  flammability…”
\end{quoting}

We have changed the title.

\begin{quoting}
  6. The last paragraph of the discussion has a significant overlap with the
  last paragraph of the discussion in the authors’ previous paper (de Magalhães
  \& Schwilk 2012) in terms of the main content.
\end{quoting}

We have rewritten much of the discussion to better include findings from the
literature and rewritten the concluding paragraph.

\begin{quoting}
The abstract can be shorter. It exceeds 350 words limit. 
\end{quoting}

We have editing the abstract to more clearly state the research questions
reduce its length.

\begin{quoting}
Line 50: add reference to the first sentence.
\end{quoting}

We have added one.

\begin{quoting}
  Lines 50-61: in this paragraph, after the first sentence I think it is more
  logic to first explain the mechanism of moisture’s negative effects on fire,
  then explain the reason of the driving importance of dead fuel moisture
  content to surface fire ignition and behavior because fine dead fuels respond
  more readily to environmental changes than living fuels.
\end{quoting}

We have rearranged the logic of the Introduction as suggested.

\begin{quoting}
Line 72: revise the description ``litter fuel driven system''.
\end{quoting}

We have rewritten this section and removed that phrase. See lines 100--108.

\begin{quoting}
Line 75: “at any particular time”?
\end{quoting}

We have rewritten this for clarity. See line response above.

\begin{quoting}
Line 85: revise ``driven by by'' and ``emergent litter litter''.
\end{quoting}

Fixed.

\begin{quoting}
Line 88: “similarly”?
\end{quoting}

We were not sure what the reviewer's criticism was here but have rewritten the
Introduction.

\begin{quoting}
  Line 95: “absorption capacity” and “rate of moisture loss” are not correct
  descriptions of processes. Better say “absorption and retention”.
\end{quoting}

Good point. We fixed that here and in another sentence elsewhere. Furthermore,
we've edited to explain what we mean by ``absorption.'

\begin{quoting}
  Line 98: revise “after saturation” to “dried certain time after saturation”
  for clarity.
\end{quoting}

Edited. See paragraph that begins on line 109.

\begin{quoting}
  Line 101: delete “also”.
\end{quoting}

Fixed.

\begin{quoting}
  Lines 102-103: litter bulk density is not a litter particle trait. Delete
  “will influence air flow through the litter bed”.
\end{quoting}

Fixed.

\begin{quoting}
  Lines 103-105: leaf size is not leaf litter trait.
\end{quoting}

We have rewritten this. See lines 113--125.

\begin{quoting}
  Line 107: “fuel-beds composed of larger leaves with lower moisture
  retainability”?
\end{quoting}

We edited this section. See above.

\begin{quoting}
  Line 115: start a new paragraph here.
\end{quoting}

We have broken these ideas into multiple paragraphs.

\begin{quoting}
  Lines 117-119: better specify the system, because negative non-additive
  effects on litter flammability were also found.
\end{quoting}

Yes, excellent point. We have added detail and citations here (lines 128--131).

\begin{quoting}
  Line 120: “especially if different leaf traits influence absorptive capacity
  and drying rate”?
\end{quoting}

We rewrote these sentences.

\begin{quoting}
  Line 125: revise “two leaf and litter bed traits”, and better directly say
  “SLA and packing density”.
\end{quoting}

Done. Lines 145--146.

\begin{quoting}
  Line 133: delete “also”.
\end{quoting}

Done.

\begin{quoting}
  Line 134: “if species contribute equally to litter moisture dynamics”?
\end{quoting}

We have rewritten this. See lines 148--151.

\begin{quoting}
  Line 136: delete “on”. Revise “ecologically-relevant”.
\end{quoting}

We assume that the reviewer's complaint is that ``ecologically-relevant'' is
vague. We have revised this and removed that term.

\begin{quoting}
Line 141: revise “(36 36N, 118 42 W)”.
\end{quoting}

Symbols fixed throughout.

\begin{quoting}
  Line 147: at least for Quercus, leaves fallen in the previous year will be
  partly decomposed in the summer.
\end{quoting}

We have added some detail to this and also now provide particle size
distributions for each species measured on the actual stored litter (SI Fig.~1).

\begin{quoting}
  Line 155: add more details about how the litter was collected?
\end{quoting}

We have added information to the methods explaining the collection in more
detail. See lines 169--177 and our response to the Associate Editor, above.

\begin{quoting}
  Line 157: is it possible that all litter has less than 5\% fuel moisture when
  air-dried?
\end{quoting}

We have corrected the methods because the litter was initially oven dried in
large paper bags, then stored on racks in the low humidity lab. Our subsample
measurements showed less than 5\% moisture. See lines 173--177.

\begin{quoting}
  Line 158: the collecting of the leaf litter is in 2012. The conduction of the
  experiment is in 2015. Can the author explain how do they save the litter?
\end{quoting}

We have added detail on the long term dry storage. See lines 173--177.

\begin{quoting}
Line 167: 1 cm ? aluminum screen.
\end{quoting}

Yes, this was a mistake. Thanks. The aluminum mesh openings were 0.203 inches.
We have clarified and corrected to ``0.52\,cm''

\begin{quoting}
  Line 173: revise “21C temperature”.
\end{quoting}

Symbols fixed.

\begin{quoting}
  Line 182: revise “effect” to “affect”. Revise “litter bulk density” to
  “litter packing density”
\end{quoting}

Fixed.

\begin{quoting}
  Line 185: I do not think this sentence is necessary because whether or not
  petioles should be included in SLA measurements depends on the research
  question at hand.
\end{quoting}

Agreed and done.

\begin{quoting}
  Line 201: the fuel depth was controlled or not?
\end{quoting}

In flammability trials mass was fixed, width and length of the burning tray was
fixed (15\,cm by 1.5\,m), and depth was allowed to vary. We edited this section
for clarity. This decision was based on discussion with materials engineers who
study fire that recommend a burning tray length of 10 times width to best
measure steady flame spread. See lines 236--250.

\begin{quoting}
Line 205: revise “1899C”.
\end{quoting}

Symbols fixed.

\begin{quoting}
  Line 208: revise “species/moisture combination” to “species and moisture
  combination”.
\end{quoting}

Done.

\begin{quoting}
  Line 209: “the replicate was allowed to continue drying for …”
\end{quoting}

Edited for clarity (Lines 248--249).

\begin{quoting}
  Line 216: revise “mass loss” to “fuel consumption”.
\end{quoting}

Done.

\begin{quoting}
  Line 224: if not all flammability data was used in the analysis, then why
  measure them?
\end{quoting}

Flammability is a multivariate trait. To avoid collinearity and to simplify
models, two general approaches are possible: either reduce and transform the
response variables and create synthetic variables through a method such as PCA,
or pick a few variables representative of the axes of variation. When the
experiment was designed, we had intended to use the first method. The loss of
several days of thermocouple data made that an unattractive and unnecessary
option, so we opted for the simpler method of picking two traits already known
to correlate with different axes of litter flammability.

We could eliminate any mention of the other measurements if the reviewer and
editors wish. In this draft have erred on the side of full disclosure.

\begin{quoting}
  Line 242: revise “one leaf trait and one litter trait” to “one litter
  particle trait and one litter bed traits”.
\end{quoting}

Done.

\begin{quoting}
  Line 307: revise “at that point”.
\end{quoting}

We have rewritten that paragraph (lines 338--345)

\begin{quoting}
  Lines 314-320: this indicates that even in the same genus, species vary in
  flammability.
\end{quoting}

Yes. This is not surprising in the case of these three pines where one has
quite different needles. But our \emph{a priori} groups were based on genus. 

\begin{quoting}
Line 330: “rates”?
\end{quoting}

Corrected.

\begin{quoting}
  Line 344: “moisture dynamics flammability” is difficult to understand.
\end{quoting}

Yes, oops a typo. We have revised the headings.

\begin{quoting}
  Line 346: revise “apparently controlled”.
\end{quoting}

Done.

\begin{quoting}
  Line 350: revise “very dry”.
\end{quoting}

Done.

\begin{quoting}
Line 351: revise “more realistic moisture levels”.
\end{quoting}

We agree such wording was vague. Fixed.

\begin{quoting}
  Line 361: revise “that that”.
\end{quoting}

Fixed.

\begin{quoting}
  Line 366: “but”?
\end{quoting}

We've rewritten these sentences as ``Our two Cupressaceae, which have dense
litter and low maximum water absorption, had low flammability across moisture
levels.'' Lines 409--411.

\begin{quoting}
  Line 401: “striking”?
\end{quoting}

We removed the word.

\begin{quoting}
  Line 410: “not met by the constituent species alone”?
\end{quoting}

Perhaps the word ``alone'' is ambiguous? We edited these sentences (lines 459--461).

\begin{quoting}
  Line 415: “demonstrates”?
\end{quoting}

We rewrote this sentence.

\begin{quoting}
  Line 439: revise “to”.
\end{quoting}

Done.

\closing{Sincerely,}

\end{letter}
\end{document}
