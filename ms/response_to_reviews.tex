 
\documentclass[letterpaper, 12pt]{letter}
\usepackage{times, fullpage}
\usepackage{url}

\address{Department of Biological Sciences\\Flint and Main\\Texas Tech University\\
  Lubbock, TX 79409}

\name{Dylan W. Schwilk}

\begin{document}
\begin{letter}{}

\opening{Dear Dr. David Gibson,}

Enclosed is our revised manuscript research article, ``''.

We thank the editor and reviewer for their helpful comments and suggestions. We
have itemized each comment below and followed each with our reply and a list of
changes we made in response to the comment. 


{\bf Responses to Senior Editor's comments:}

\begin{quote}
 I support the AE's recommendation to reject but allow resubmission. However, the bar for a resubmission will be high. My feeling is that your study confirms the essential findings from your 2012 study published in Journal of Ecology with the addition that your methodology of drying down adds some realism. Indeed, the confirmatory nature of this study was pointed out by reviewer 2 in their point 6. In addition to the points raised by the AE and reviewers, I will be looking in any resubmission for more evidence that this study advances your earlier work (apart from your last paragraph, your discussion is largely an explanation of your results with little context from other studies), that the findings have real ecological relevance (i.e., does it matter to the mixed-conifer communities?) and how is the new dry down treatment related to dry down conditions in litter mixtures in the field. If you can adequately address all of these concerns then I'll be happy to consider a resubmission.
\end{quote}

TODO.

Add context also suggested by reviewer 1.

\begin{quote}
 
\end{quote}

TODO


{\bf Responses to Associate Editor's comments:}

\begin{quote}
I support the AE's recommendation to reject but allow resubmission. However, the bar for a resubmission will be high. My feeling is that your study confirms the essential findings from your 2012 study published in Journal of Ecology with the addition that your methodology of drying down adds some realism. Indeed, the confirmatory nature of this study was pointed out by reviewer 2 in their point 6. In addition to the points raised by the AE and reviewers, I will be looking in any resubmission for more evidence that this study advances your earlier work (apart from your last paragraph, your discussion is largely an explanation of your results with little context from other studies), that the findings have real ecological relevance (i.e., does it matter to the mixed-conifer communities?) and how is the new dry down treatment related to dry down conditions in litter mixtures in the field. If you can adequately address all of these concerns then I'll be happy to consider a resubmission.
\end{quote}

Thank you. Add context. Yes, several comments converged on the absolutely correct criticism that our discussion did not engage adequately with the literature.  Wer have added X and X...

inclduing cite Zhoa 2019

\begin{quote}
The authors mention leaf size explicitly as an important parameter in non-additivity (with some references including their own 2012 paper) but they do not include it in the current study. Especially at low moisture, the small individual needles of Abies should cause low ventilation and a high likelihood of negative non-additivity on flammability, as was shown by Zhao et al 2016 Ecol. Evol. Figure 4). In the Methods, as observed also by reviewer 2, there is very little detail on the extent to which litter of the respective species was picked up as individual leaves, branchlets without leaves versus branchlets with leaves still attached. This is really important, also because positive versus negative non-additivity of leaf-twig mixtures have been reported recently (Zhao et al. 2019 Plant and Soil). I feel that data and/or photos indicating the size and shape of the different litter types (and if possible variability in these) are important to add and to be used in the interpretation of the results.
\end{quote}

We have added a figure showing particle size distributions for each of these species. Yes, Abies litter is a a combination of still attached needes and those that have detached from the branchlet.

We 

\begin{quote}
In the Methods, it was unclear how the various treatments and replicates were grouped over time, given that temperature and humidity in the fire facility could vary. Were there blocks in time with one replicate for each of the treatments (in random burning order) within a block? Or was there a different statistical design in time? 
\end{quote}

Guided by the drying rates measured in the initail trials, we staggered saturation times across species aiming to allow completely blocked design in time. This staggering was mostly successful but Because some ignition attempts were unsuccessful, some days lacked the full complemnt of species while others later in the porcess sometimes had more than one replicate per species burned.

TODO

\begin{quote}
Line 196 (see also reviewer 2): how can we assume that the quality of the stored litter is the same as that used in the previous study 3 years earlier, i.e. can these two datasets be safely combined?
\end{quote}

We have added detail as requested by reviewer 2. See lines (X-X). The litter was dried in the lab (very low hmidity, but not oven dried), for several weeks on open racks. Then the litter was bagged and sealed and stored within garbage bags inside cardboard boxes in a lab storage area.  We saw no signs of mold or ongoing decomposition. We minimized handling of the bags to avoid mechanical damage and changes to little particle size.


\begin{quote}
Line 365 (see also reviewer 1): on the other hand Cornwell et al. 2015 New Phyt reported mostly high flammability in a range of Cupressaceae.
\end{quote}

TODO.  Mention.  Maybe due to very small samples that miminimize effects of packing in Cornwell et al 2015?



{\bf Responses to the Reviewer 1's comments}


\begin{quote}
This work compares mixtures of fuels, relative to their individual components, in terms of moisture retention and flammability. Moisture dynamics were also investigated in relation to properties of leaves of the species forming litter. This information is important in understanding flammability of surface fuels in the field. 

The study has been well thought through and well executed, and the ms is generally well written. Stylistically, the main problem is that commas are missing in many sentences. This might sound trivial, but will really help the reader.
\end{quote}

We thank the reviewer for their comments and apologize for the failure in proofreading. The reviewer should not have had to do such copy editing and I I sincerely apologize for that. We have taken heed of both reviewers' comments and fixed all specific grammar errors and fixed the issue we had with missing symbols. For brevity, we've omited response to the very simple suggested grammar fixes and simply made all requested changes. We respond to other points individually, below


\begin{quote}
line 72 - the comment about natural lightning ignitions comes from nowhere, and is a  bit of a jolt for the reader - rephrase slightly, e.g. many fires result from lightning ignitions, which are concurrent with ....
\end{quote}

TODO.

We have rewritten this in response to reviewer 2s point that dry lightning is also amajor ignition source.

TODO


\begin{quote}
line 103 - 'litter packing therefore affects fire behaviour in two ways' is easier to read
\end{quote}

Yes, agreed. We have tried to clean up the writing throughout and make the prose more active and precise.

\begin{quote}
line 133 - 'in mixtures of litter with varying levels..' is more explicit
\end{quote}

Agreed. fixed.

\begin{quote}
line 182 - affect, not effect
\end{quote}

Yes, an embarrassing error.


\begin{quote}
line 220 - see also Prior et al (2018)  Conceptualizing ecological flammability ....Fire 1, article 14
\end{quote}

Thank you. This was an oversight. We have also added more context from the
literature to our Discussion

TODO

\begin{quote}
lines 243/244 I was a bit puzzled here -It would help to be more specific here - which aspects of moisture dynamics? Which response variables did you use?  (or 'as described below')
\end{quote}

TODO. We have rewritten that section to be more precise.

\begin{quote}
line 290 - you show an association, not necessarily an effect
\end{quote}

True. We have edited this as suggested.

\begin{quote}
line 307 and line 326 - a minor inconsistency is that you analyse present species groups, yet discuss the results in terms of individual species
\end{quote}

TODO

Yes. We have made edits to

TODO. Check wording.



\begin{quote}
lines 348-349 - logically, this does not necessarily follow - factors such as large leaves or high SLA could independently affect both absorptive capacity and flammability - although fuel moisture does of course have a strong effect 
\end{quote}

TODO.


\begin{quote}
Table 4 - get rid of the unnecessary zeroes in the df column
\end{quote}

Fixed.

\begin{quote}
line 365 - wording is problematic - change to '...ignition is possible, it had faster spread rates ...'
\end{quote}

Agreed. Fixed.

\begin{quote}
line 365 - The low flammability of litter from your Cupressaceae species is interesting - Scarff and Westoby (2006) also found this for the Australian Cupressaceae genus Callitris, as did Bowman et al. (2018) J. Ecol 106, 1010 -1022
\end{quote}

We have added some better discussion of past results and included discussion of these papers, as well as contrasting with the Cornwell et al 2015 result as pointed out by the Associate Editor.

\begin{quote}
line 387 - capitalise 'I' in SI Table  8
\end{quote}

Fixed.

\begin{quote}
line 403 - change to 'to the completely dry litter mixtures studied in previous work'
\end{quote}

TODO


\begin{quote}
Figures 7 and 8 - is there a reason that you don't present the information in the same way as in Figure 6, if your main point is the discrepancy between the mixtures and the averaged individual values?
\end{quote}

We agree this does lack consistency and an early draft used the same concise Fig 6 format for all of them. But we decided that having time on the x-axis was worth the space and allows the reader more insight into the

Reviewer 2 pointed out some exceptions to the overall statistical pattern and those exceptions would not have been easily visible without an explicit time axis.


\begin{quote}
line 416 - Prior et al (2018) also show that rate of spread and consumability are largely independent in litter fuels
\end{quote}

TODO

Yes. We have added citations here.



{\bf Responses to the Reviewer 2's comments}

\begin{quote}

My major concerns are:
1. One motivation for the study is not valid. I do not completely agree with the authors that natural lightning ignitions are concurrent with precipitation and, therefore, fire behavior can depend on rate of drying following a wetting event. The efficiency of individual lighting strikes in igniting a fire is affected by the moisture properties of forest fuel at the moment of lighting (Krawchuket et al. 2006, Ecology). Dry lighting that occurs without any accompanied rainfall nearby is most likely to cause wildfire because the fuel is dry enough to be ignited.
\end{quote}


TODO.  Get data.


\begin{quote}
2. The research question of the research is not explicitly stated both in the abstract and in the introduction. In the end of the introduction, the authors focus on specific data-collection goals, while leave the reader to figure out what the research question is. 
\end{quote}

TODO.


\begin{quote}
3. The authors concluded that positive non-additivity appeared to be the rule in litter-driven fire and this non-additivity became more positive with fuel drying. But if we look closely at figure 8, non-additivity was negative within 50 hours since wetting, non-additivity was both positive and negative at about 75 hours since wetting, and non-additivity was positive at about 100 hours since wetting. The high variability of non-additive effects at about 75 hours since wetting might have important ecological effects on fire ignition and behavior, and are important information cannot be ignored.
\end{quote}


Yes, this is a good point and we have added some discussion of this.  We tried to lmit our statements, hwoever, to those supported by the statistical tests, and .....

TODO.

\begin{quote}
4. The writing of this manuscript needs to be improved for clarity and accuracy. There are many typos in the text, and the using to scientific terms lack consistency through the text.
\end{quote}

We have made extensive edits in response to both reviewer's suggestions

\begin{quote}
5. The title does not precisely communicate the key message of the study. I suggest it is more precise to say “moisture effect … and alter non-additive flammability…”
\end{quote}

TODO

\begin{quote}
6. The last paragraph of the discussion has a significant overlap with the last paragraph of the discussion in the authors’ previous paper (de Magalhães \& Schwilk 2012) in terms of the main content.
\end{quote}

We have rewritten much of the discussion to better include findings from the literature.....

\begin{quote}
The abstract can be shorter. It exceeds 350 words limit. 
\end{quote}

TODO

\begin{quote}
Line 50: add reference to the first sentence.
\end{quote}

TODO

We disagree in this case. 

\begin{quote}
Lines 50-61: in this paragraph, after the first sentence I think it is more logic to first explain the mechanism of moisture’s negative effects on fire, then explain the reason of the driving importance of dead fuel moisture content to surface fire ignition and behavior because fine dead fuels respond more readily to environmental changes than living fuels.
\end{quote}

TODO

\begin{quote}
Line 72: revise the description “ litter fuel driven system”.
\end{quote}

TODO

\begin{quote}
Line 75: “at any particular time”?
\end{quote}

\begin{quote}
Line 85: revise “driven by by” and “emergent litter litter”.
\end{quote}

\begin{quote}
Line 88: “similarly”?
\end{quote}

\begin{quote}
Line 95: “absorption capacity” and “rate of moisture loss” are not correct descriptions of processes. Better say “absorption and retention”.
\end{quote}

\begin{quote}
Line 98: revise “after saturation” to “dried certain time after saturation” for clarity.
\end{quote}

\begin{quote}
Line 101: delete “also”.
\end{quote}

\begin{quote}
Lines 102-103: litter bulk density is not a litter particle trait. Delete “will influence air flow through the litter bed”.
\end{quote}

\begin{quote}
Lines 103-105: leaf size is not leaf litter trait.
\end{quote}

\begin{quote}
Line 107: “fuel-beds composed of larger leaves with lower moisture retainability”?
\end{quote}

\begin{quote}
Line 115: start a new paragraph here.
\end{quote}

\begin{quote}
Lines 117-119: better specify the system, because negative non-additive effects on litter
flammability were also found.
\end{quote}

\begin{quote}
Line 120: “especially if different leaf traits influence absorptive capacity and drying rate”?
\end{quote}

\begin{quote}
Line 125: revise “two leaf and litter bed traits”, and better directly say “SLA and packing density”.
\end{quote}

\begin{quote}
Line 133: delete “also”.
\end{quote}

\begin{quote}
Line 134: “if species contribute equally to litter moisture dynamics”?
\end{quote}

\begin{quote}
Line 136: delete “on”. Revise “ecologically-relevant”.
\end{quote}

\begin{quote}
Line 141: revise “(36 36N, 118 42 W)”.
\end{quote}

\begin{quote}
Line 147: at least for Quercus, leaves fallen in the previous year will be partly decomposed in the summer.
\end{quote}

\begin{quote}
Line 155: add more details about how the litter was collected?
\end{quote}

\begin{quote}
Line 157: is it possible that all litter has less than 5% fuel moisture when air-dried?
\end{quote}

\begin{quote}
Line 158: the collecting of the leaf litter is in 2012. The conduction of the experiment is in 2015. Can the author explain how do they save the litter?
\end{quote}

\begin{quote}
Line 167: 1 cm ? aluminum screen.
\end{quote}

Yes, this was a typo. Thanks. The aluminum mesh openings were 0.203 inches. We have clarified and corrected to ``0.52\,cm'' 

\begin{quote}
Line 173: revise “21C temperature”.
\end{quote}

\begin{quote}
Line 182: revise “effect” to “affect”. 	Revise “litter bulk density” to “litter packing density”
\end{quote}

\begin{quote}
Line 185: I do not think this sentence is necessary because whether or not petioles should be included in SLA measurements depends on the research question at hand.
\end{quote}

\begin{quote}
Line 201: the fuel depth was controlled or not?
\end{quote}

\begin{quote}
Line 205: revise “1899C”.
\end{quote}

\begin{quote}
Line 208: revise “species/moisture combination” to “species and moisture combination”.
\end{quote}

\begin{quote}
Line 209: “the replicate was allowed to continue drying for …”
\end{quote}

\begin{quote}
Line 216: revise “mass loss” to “fuel consumption”.
\end{quote}

\begin{quote}
Line 224: if not all flammability data was used in the analysis, then why measure them?
\end{quote}

\begin{quote}
Line 242: revise “one leaf trait and one litter trait” to “one litter particle trait and one litter bed traits”.
\end{quote}

\begin{quote}
Line 307: revise “at that point”.
\end{quote}

\begin{quote}
Lines 314-320: this indicates that even in the same genus, species vary in flammability.
\end{quote}

\begin{quote}
Line 330: “rates”?
\end{quote}

\begin{quote}
Line 344: “moisture dynamics flammability” is difficult to understand.
\end{quote}

\begin{quote}
Line 346: revise “apparently controlled”.
\end{quote}

\begin{quote}
Line 350: revise “very dry”.
\end{quote}

\begin{quote}
Line 351: revise “more realistic moisture levels”.
\end{quote}

\begin{quote}
Line 361: revise “that that”.
\end{quote}

\begin{quote}
Line 366: “but”?
\end{quote}

\begin{quote}
Line 401: “striking”?
\end{quote}

\begin{quote}
Line 410: “not met by the constituent species alone”?
\end{quote}

\begin{quote}
Line 415: “demonstrates”?
\end{quote}

\begin{quote}
Line 439: revise “to”.
\end{quote}

\closing{Sincerely,}

\end{letter}
\end{document}
