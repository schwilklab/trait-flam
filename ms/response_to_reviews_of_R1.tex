 
\documentclass[letterpaper, 12pt]{letter}
\usepackage{times, fullpage}
\usepackage{url}

\address{Department of Biological Sciences\\Flint and Main\\Texas Tech University\\
  Lubbock, TX 79409}

\name{Dylan W. Schwilk}

\begin{document}
\begin{letter}{}

\opening{Dear Dr. David Gibson,}

Enclosed is our revised manuscript research article, ``Moisture absorption and
drying alter nonadditive litter flammability in a mixed conifer forest''.

The editors listed two substantial concerns with our revised manuscript: 1) we
did not adequately articulate the novelty of the study and 2) that we had too
strong an emphasis and ``\ldots and large amounts of text and graphics about
single species results.'' We have tried to rectify these shortcomings with this
submission. Our most substantial change was to combine figures and reorganize
the results around these new figures as the editors suggested.

Additionally, to answer the first complaint, we have slightly revised the
Introduction. We have emphasized that this study is the first to examine
ecologically relevant drying rates and their interaction with non additive
moisture effects on flammability. Yes, Blauw et al (2015) report moisture and
flammability interactions but their study did not investigate the drying
process which our results demonstrate is key to understanding flammability
changes with time since wetting. Furthermore, their study investigated a very
different system. The Sierran mixed confier forests in which our study takes
place include large areas where surface fire driven by leaf litter is the
dominant ecological disturbance and many fires are completely decoupled from
canopy fuels. I think extensive quibbles with Blauw et al would be out of place
in our Introduction. Our ability to investigate the mechanisms underlying our
results are limited by the diversity of traits in our species pool, but theses
consistent patterns within groups suggest mechanisms which we explore in the
Discussion. No previous work has been able to address these questions --- with
four species in Blauew et al 2015 there was even less ability to provide
information on potential trait effects.

The editors' complaint regarding ``single species'' results must refer to the
dry down curves which do show each species and to other figures which show
results by genus. We have combined the figures as suggested. This allows a more
concise and consistent presentation of the results and I hope satisfies the
suggestion to reduce emphasis on these results by taxon. We feel that the
differences across taxa are important, however, and should be presented before
the mixture results. The consistency across groups (not necessarily species)
points to potential generalizability of the results and these families and
genera are important components of many forests. Additionally, some results
necessitated slight modifications to the planned analyses (eg those reported on
lines 350-360). My preference is to be exact and completely upfront on these
changes, but that does add some length to the text. Is this something the
editors wanted further cut down? I am loathe to do so but would consider it. I
worry that to cut the text further here would rob some context from the mixture
results we later describe. We have made cuts and other edits to the Discussion
to increase focus on the mixture results.

The decision letter requested I mark new changes to the text in color.
Therefore, I will submit a second pdf alongside the main submission showing
highlighted changes to the main text (not the figures). This colorized pdf was
created with latexdiff, but editors and reviewers are also welcome to examine
the diffs of the text source directly in git as we have made the repository
public.

% git diff --word-diff --color-words R1 Magalhaes_Schwilk_litter_flam_moisture.tex | aha > worddiff.html
% or see latexdiff --flatten

{\bf Additional specific comments and our responses:}

\begin{quote}
  Line 124 typo in moisture
\end{quote}

Thank you. fixed

\begin{quote}
  Line 140 delete "on"
\end{quote}

We disagree because the sentence would then no longer make sense. ``Based
largely on'' is grammatically correct in at least American English usage of the
phrasal verb.

\begin{quote}
  Line 173: what was the temperature of oven-drying? This is important to know.
\end{quote}

Oven temperature was $60^\circ$C. We've added that to the methods (line 178).
That is an interesting point about the potential importance of this. I've seen
some data showing that drying temperature speed will effect volatile retention,
but I think that was from live collected foliage. I wonder if there is much
effect on litter that has gone through natural wetting/drying cycles?

\begin{quote}
Line 344: typo in drier
\end{quote}

Oops! Yes, not the clothes dryer. Fixed!


\closing{Sincerely,}

\end{letter}
\end{document}
