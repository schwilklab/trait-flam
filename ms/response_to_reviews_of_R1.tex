 
\documentclass[letterpaper, 12pt]{letter}
\usepackage{times, fullpage}
\usepackage{url}

\address{Department of Biological Sciences\\Flint and Main\\Texas Tech University\\
  Lubbock, TX 79409}

\name{Dylan W. Schwilk}

\begin{document}
\begin{letter}{}

\opening{Dear Dr. David Gibson,}

Enclosed is our revised manuscript research article, ``Moisture absorption and drying alter nonadditive litter flammability in a mixed conifer forest''.

The editors listed two substantial concerns with our revised manuscript: 1) we did not adequately articulate the novelty of the study and 2) that we had too strong an emphasis and ``\ldots and large amounts of text and graphics about single species results.'' We have tried to rectify both of these shortcomings with this submission. 

To answer the first complaint, we have slightly revised the Introduction. We have emphasized that this study is the first to examine ecologically relevant drying rates and their interaction with non additive moisture effects on flammability. Yes, Blauw et al (2015) report moisture and flammability interactions but their study did not investigate the drying process which our results demonstrate is key to understanding flammability changes with time since wetting. Furthermore, their study investigated a very different system. The Sierran mixed confier forests in which our study takes place include large areas where surface fire driven by leaf litter is the dominant ecological disturbance. I think extensive quibbles with Blauw et al would be out of place in our Introduction.

The editor's complaint regarding ``single species'' results must refer to graphs which show results by genus. We have combined the figures as suggested. This allows a more concise and consistent presentation of the results and I hope satisfies the suggestion to reduce emphasis on these results by taxon. We feel that the differences across taxa are important, however. Our ability to investigate the mechanisms underlying our results are limited by the diversity of traits in our species pool, but theses consistent patterns within groups suggest mechanisms which we explore in the Discussion. No previous work has been able to address these questions (eg with four species in Blauew et al 20156 there was even less ability to provide information on potential trait effects).


I ahve porvided a word-based diff of our current version source text against our previous submission.  I hope this is what the jhournal wants when it asks for color-marked changes.  


{\bf Additional specific comments and our responses:}

\begin{quote}
  Line 124 typo in moisture
\end{quote}

Thank you. fixed

\begin{quote}
  Line 140 delete "on"
\end{quote}

We disagree as the sentence would no longer make sense. Can the editor state the objection?

\begin{quote}
Line 173:  what was the temperature of oven-drying? This is important to know.
\end{quote}

Oven temperature was 50C. Our lab humidity is very low (low single digits pretty much all year).

\begin{quote}
Line 344: typo in drier
\end{quote}

Oops! Yes, not the clothes dryer. Fixed!

  
\end{quote}


\closing{Sincerely,}

\end{letter}
\end{document}

from Rita:

Arguments in favor of resubmission:

The work we did and published in 2012 was novel in many ways. We used a novel
approach to litter flammability methodology which allowed us to 1) approximate
realistic litter assemblages in the field (litter was not manipulated/cut) and
2) approximate realistic fire behavior with the design of our apparatus. We
were also the first to demonstrate non-additive effects in litter flammability
in the mixed-conifer forest. However, this original study was done with
artificially dried litter. All samples were dried to <
5% moisture content, which wasn’t realistic of most days in the mixed-conifer.
 % It became clear that, while we had made a significant advance in the
 % understanding of litter behavior, we lacked knowledge of other litter
 % characteristics that, in the field, play a major role in determining litter
 % fire behavior.

At the time of our work, and still now, only one paper (Blauw et al. 2015) had established non-additive behavior of litter under wet conditions. We were glad to see this was an area of interest to the community but it was clear a lot of work still needed to be done. Our work differs from this paper in a number of relevant ways. First, the particulars of the experiments are very different. Our work was done for a community unlike that of Blauw et al. and we used a different apparatus for the burning experiments. More importantly, our approach to this work was not just ecological, but also realistic. We wanted to be able to approximate as much as possible field conditions. In this aspect, the work follows from our previous work. We used the same species, the same community, the same burning apparatus. We view this as a strength. We tested hypotheses and a novel way to burn litter and we were convinced by our results that this was a solid method. Infusing more realism was the natural next step. Though we were interested in knowing the moisture content of our species, the measure that we considered most relevant was time since wetting and how the two measures were linked, which is not often investigated. 

Understanding the mechanism behind the wetting and drying process (moisture absorption and desorption) brings us closer to understanding how community assembly will influence fire behavior, both for current assemblages and in light of changing species compositions due to ongoing climate change. We demonstrated that the flammability ranking of our species changed with higher moisture contents, and therefore, understanding the process of moisture absorption and desorption, and time since wetting, while also quantifying a species contribution to the litter throughout the community, can help predict flammability of that community. This is especially relevant since litter drying rates are non-linear. We showed non-additive effects, but unlike our original paper, these effects were not all positive. And while these results are generally in line with results found in Blauw et al. they differ in significant ways: we found that flammability non-additive effects were more positive as mixtures dried, and we also demonstrated complex non-additivity in drying rates of litter mixtures. 

Our work aligning generally with that of Blauw et al. is hardly indicative of a trend, when these are only two papers showing this result. More work is necessary to explore this complex non-additive behavior of litter under moisture conditions. With our work, we offered the mechanism controlling moisture content in the 8 species that represent mixed-conifer forest in our site, but we also demonstrated that knowledge of single species behavior will not be representative of the behavior of the more realistic mixtures. The results for our oak are evidence of this. The integration of realistic litter samples, using time since wetting, and delineating the mechanism is novel in the field of litter flammability and establishes a direct link to the ecology of the mixed-conifer forest in Northern California. 
